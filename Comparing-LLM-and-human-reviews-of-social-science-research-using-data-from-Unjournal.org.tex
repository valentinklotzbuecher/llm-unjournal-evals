% Options for packages loaded elsewhere
\PassOptionsToPackage{unicode}{hyperref}
\PassOptionsToPackage{hyphens}{url}
\PassOptionsToPackage{dvipsnames,svgnames,x11names}{xcolor}
%
\documentclass[
  11pt,
  letterpaper,
  DIV=11,
  numbers=noendperiod]{scrreprt}

\usepackage{amsmath,amssymb}
\usepackage{iftex}
\ifPDFTeX
  \usepackage[T1]{fontenc}
  \usepackage[utf8]{inputenc}
  \usepackage{textcomp} % provide euro and other symbols
\else % if luatex or xetex
  \usepackage{unicode-math}
  \defaultfontfeatures{Scale=MatchLowercase}
  \defaultfontfeatures[\rmfamily]{Ligatures=TeX,Scale=1}
\fi
\usepackage{lmodern}
\ifPDFTeX\else  
    % xetex/luatex font selection
\fi
% Use upquote if available, for straight quotes in verbatim environments
\IfFileExists{upquote.sty}{\usepackage{upquote}}{}
\IfFileExists{microtype.sty}{% use microtype if available
  \usepackage[]{microtype}
  \UseMicrotypeSet[protrusion]{basicmath} % disable protrusion for tt fonts
}{}
\makeatletter
\@ifundefined{KOMAClassName}{% if non-KOMA class
  \IfFileExists{parskip.sty}{%
    \usepackage{parskip}
  }{% else
    \setlength{\parindent}{0pt}
    \setlength{\parskip}{6pt plus 2pt minus 1pt}}
}{% if KOMA class
  \KOMAoptions{parskip=half}}
\makeatother
\usepackage{xcolor}
\usepackage[margin=1in]{geometry}
\setlength{\emergencystretch}{3em} % prevent overfull lines
\setcounter{secnumdepth}{5}
% Make \paragraph and \subparagraph free-standing
\ifx\paragraph\undefined\else
  \let\oldparagraph\paragraph
  \renewcommand{\paragraph}[1]{\oldparagraph{#1}\mbox{}}
\fi
\ifx\subparagraph\undefined\else
  \let\oldsubparagraph\subparagraph
  \renewcommand{\subparagraph}[1]{\oldsubparagraph{#1}\mbox{}}
\fi


\providecommand{\tightlist}{%
  \setlength{\itemsep}{0pt}\setlength{\parskip}{0pt}}\usepackage{longtable,booktabs,array}
\usepackage{calc} % for calculating minipage widths
% Correct order of tables after \paragraph or \subparagraph
\usepackage{etoolbox}
\makeatletter
\patchcmd\longtable{\par}{\if@noskipsec\mbox{}\fi\par}{}{}
\makeatother
% Allow footnotes in longtable head/foot
\IfFileExists{footnotehyper.sty}{\usepackage{footnotehyper}}{\usepackage{footnote}}
\makesavenoteenv{longtable}
\usepackage{graphicx}
\makeatletter
\def\maxwidth{\ifdim\Gin@nat@width>\linewidth\linewidth\else\Gin@nat@width\fi}
\def\maxheight{\ifdim\Gin@nat@height>\textheight\textheight\else\Gin@nat@height\fi}
\makeatother
% Scale images if necessary, so that they will not overflow the page
% margins by default, and it is still possible to overwrite the defaults
% using explicit options in \includegraphics[width, height, ...]{}
\setkeys{Gin}{width=\maxwidth,height=\maxheight,keepaspectratio}
% Set default figure placement to htbp
\makeatletter
\def\fps@figure{htbp}
\makeatother
% definitions for citeproc citations
\NewDocumentCommand\citeproctext{}{}
\NewDocumentCommand\citeproc{mm}{%
  \begingroup\def\citeproctext{#2}\cite{#1}\endgroup}
\makeatletter
 % allow citations to break across lines
 \let\@cite@ofmt\@firstofone
 % avoid brackets around text for \cite:
 \def\@biblabel#1{}
 \def\@cite#1#2{{#1\if@tempswa , #2\fi}}
\makeatother
\newlength{\cslhangindent}
\setlength{\cslhangindent}{1.5em}
\newlength{\csllabelwidth}
\setlength{\csllabelwidth}{3em}
\newenvironment{CSLReferences}[2] % #1 hanging-indent, #2 entry-spacing
 {\begin{list}{}{%
  \setlength{\itemindent}{0pt}
  \setlength{\leftmargin}{0pt}
  \setlength{\parsep}{0pt}
  % turn on hanging indent if param 1 is 1
  \ifodd #1
   \setlength{\leftmargin}{\cslhangindent}
   \setlength{\itemindent}{-1\cslhangindent}
  \fi
  % set entry spacing
  \setlength{\itemsep}{#2\baselineskip}}}
 {\end{list}}
\usepackage{calc}
\newcommand{\CSLBlock}[1]{\hfill\break\parbox[t]{\linewidth}{\strut\ignorespaces#1\strut}}
\newcommand{\CSLLeftMargin}[1]{\parbox[t]{\csllabelwidth}{\strut#1\strut}}
\newcommand{\CSLRightInline}[1]{\parbox[t]{\linewidth - \csllabelwidth}{\strut#1\strut}}
\newcommand{\CSLIndent}[1]{\hspace{\cslhangindent}#1}

\KOMAoption{captions}{tableheading,figureheading}
\makeatletter
\@ifpackageloaded{tcolorbox}{}{\usepackage[skins,breakable]{tcolorbox}}
\@ifpackageloaded{fontawesome5}{}{\usepackage{fontawesome5}}
\definecolor{quarto-callout-color}{HTML}{909090}
\definecolor{quarto-callout-note-color}{HTML}{0758E5}
\definecolor{quarto-callout-important-color}{HTML}{CC1914}
\definecolor{quarto-callout-warning-color}{HTML}{EB9113}
\definecolor{quarto-callout-tip-color}{HTML}{00A047}
\definecolor{quarto-callout-caution-color}{HTML}{FC5300}
\definecolor{quarto-callout-color-frame}{HTML}{acacac}
\definecolor{quarto-callout-note-color-frame}{HTML}{4582ec}
\definecolor{quarto-callout-important-color-frame}{HTML}{d9534f}
\definecolor{quarto-callout-warning-color-frame}{HTML}{f0ad4e}
\definecolor{quarto-callout-tip-color-frame}{HTML}{02b875}
\definecolor{quarto-callout-caution-color-frame}{HTML}{fd7e14}
\makeatother
\makeatletter
\@ifpackageloaded{bookmark}{}{\usepackage{bookmark}}
\makeatother
\makeatletter
\@ifpackageloaded{caption}{}{\usepackage{caption}}
\AtBeginDocument{%
\ifdefined\contentsname
  \renewcommand*\contentsname{Table of contents}
\else
  \newcommand\contentsname{Table of contents}
\fi
\ifdefined\listfigurename
  \renewcommand*\listfigurename{List of Figures}
\else
  \newcommand\listfigurename{List of Figures}
\fi
\ifdefined\listtablename
  \renewcommand*\listtablename{List of Tables}
\else
  \newcommand\listtablename{List of Tables}
\fi
\ifdefined\figurename
  \renewcommand*\figurename{Figure}
\else
  \newcommand\figurename{Figure}
\fi
\ifdefined\tablename
  \renewcommand*\tablename{Table}
\else
  \newcommand\tablename{Table}
\fi
}
\@ifpackageloaded{float}{}{\usepackage{float}}
\floatstyle{ruled}
\@ifundefined{c@chapter}{\newfloat{codelisting}{h}{lop}}{\newfloat{codelisting}{h}{lop}[chapter]}
\floatname{codelisting}{Listing}
\newcommand*\listoflistings{\listof{codelisting}{List of Listings}}
\makeatother
\makeatletter
\makeatother
\makeatletter
\@ifpackageloaded{caption}{}{\usepackage{caption}}
\@ifpackageloaded{subcaption}{}{\usepackage{subcaption}}
\makeatother
\ifLuaTeX
  \usepackage{selnolig}  % disable illegal ligatures
\fi
\usepackage{bookmark}

\IfFileExists{xurl.sty}{\usepackage{xurl}}{} % add URL line breaks if available
\urlstyle{same} % disable monospaced font for URLs
\hypersetup{
  pdftitle={Comparing LLM and human reviews of social science research using data from Unjournal.org},
  pdfauthor={David Reinstein; Valentin Klotzbücher; Tianmai Michael Zhang},
  colorlinks=true,
  linkcolor={blue},
  filecolor={Maroon},
  citecolor={Blue},
  urlcolor={Blue},
  pdfcreator={LaTeX via pandoc}}

\title{Comparing LLM and human reviews of social science research using
data from Unjournal.org}
\author{David Reinstein \and Valentin Klotzbücher \and Tianmai Michael
Zhang}
\date{2025-11-30}

\begin{document}
\maketitle
\begin{abstract}
We will build and refine LLM tools to generate peer-reviews and ratings
of impactful research, and compare these with human experts' work
(esp.~from Unjournal.org): to benchmark performance, understand AI's
research taste, and develop tools to improve research evaluation and
dissemination.
\end{abstract}

\renewcommand*\contentsname{Table of contents}
{
\hypersetup{linkcolor=}
\setcounter{tocdepth}{2}
\tableofcontents
}
\bookmarksetup{startatroot}

\chapter{Introduction}\label{introduction}

\begin{tcolorbox}[enhanced jigsaw, colback=white, colbacktitle=quarto-callout-warning-color!10!white, bottomtitle=1mm, bottomrule=.15mm, opacitybacktitle=0.6, rightrule=.15mm, title=\textcolor{quarto-callout-warning-color}{\faExclamationTriangle}\hspace{0.5em}{Work in progress}, left=2mm, opacityback=0, colframe=quarto-callout-warning-color-frame, toptitle=1mm, arc=.35mm, titlerule=0mm, breakable, coltitle=black, leftrule=.75mm, toprule=.15mm]

Pages, metrics, and comparisons are under active development. Expect
rough edges and frequent updates.

\end{tcolorbox}

Is AI good at peer-reviewing? Does it offer useful and valid feedback?
Can it predict how human experts will rate research across a range of
categories? How can it help academics do this ``thankless'' task better?
Is it particularly good at spotting errors? Are there specific
categories, e.g.~spotting math errors or judging real-world relevance,
where it does surprisingly well or poorly? How does its ``research
taste'' compare to humans?

If AI research-evaluation works it could free up a lot of scientific
resources -- perhaps \$1.5 billion/year in the US alone Aczel, Szaszi,
and Holcombe (\citeproc{ref-aczel2021billion}{2021})) -- and offer more
continual and detailed review, helping improve research. It could also
help characterize methodological strengths/weaknesses across papers,
aiding training and research direction-setting. Furthermore, a key
promise of AI is to directly improve science and research. Understanding
how AI engages with research evaluations may provide a window into its
values, abilities, and limitations.

In this project, we are testing the capabilities of current large
language models (LLMs), illustrating whether they can generate research
paper evaluations comparable to expert human reviews. The Unjournal
systematically prioritizes `impactful' research and pays for
high-quality human evaluations, structured quantified ratings, claim
identification and assessment, and predictions. In this project, we use
an AI (OpenAI's \texttt{GPT-5\ Pro} model) to review social science
research papers under the same criteria used by human reviewers for The
Unjournal.

Each paper is assessed on specific dimensions -- for example, the
strength of its evidence, rigor of methods, clarity of communication,
openness/reproducibility, relevance to global priorities, and overall
quality. The LLM will provide quantitative scores (with uncertainty
intervals) on these criteria and produce a written evaluation

Our initial dataset will include the 5 research papers that have
existing Unjournal human evaluations. For each paper, the AI will
generate: (1) numeric ratings on the defined criteria, (2)
identification of the paper's key claims, and (3) a detailed review
discussing the paper's contributions and weaknesses. We will then
compare the AI-generated evaluations to the published human evaluations.

In the next phase, we will focus on papers currently under evaluation,
i.e., where no human evaluation has been made public, to allow us to
rule out any contamination.

\subsection{Our work in context}\label{our-work-in-context}

Luo et al. (\citeproc{ref-Luo2025}{2025}) survey LLM roles from idea
generation to peer review, including experiment planning and automated
scientific writing. They highlight opportunities (productivity, coverage
of long documents) alongside governance needs (provenance, detection of
LLM-generated content, standardizing tooling) and call for reliable
evaluation frameworks.

Eger et al. (\citeproc{ref-eger2025}{2025}) provide a broad review of
LLMs in science and a focused discussion of AI‑assisted peer review.
They argue: (i) peer‑review data is scarce and concentrated in
CS/OpenReview venues; (ii) targeted assistance that preserves human
autonomy is preferable to end‑to‑end reviewing; and (iii) ethics and
governance (bias, provenance, detection of AI‑generated text) are
first‑class constraints.

Zhang and Abernethy (\citeproc{ref-Zhang2025}{2025}) propose deploying
LLMs as quality checkers to surface critical problems instead of
generating full narrative reviews. Using papers from WITHDRARXIV and an
automatic evaluation framework that leverages ``LLM-as-judge,'' they
find the best performance from top reasoning models but still recommend
human oversight.

Pataranutaporn et al. (\citeproc{ref-Pataranutaporn2025}{2025}) asked
four nearly state-of-the-art LLM models (GPT-4o mini, Claude 3.5 Haiku,
Gemma 3 27B, and LLaMA 3.3 70B) to consider 1220 unique papers ``drawn
from 110 economics journals excluded from the training data of current
LLMs''. They prompted the models to act ``in your capacity as a reviewer
for {[}a top-5 economics journal{]}'' and make a publication
recommendation using a 6-point scale ranging from ``1 = Definite
Reject\ldots{}'' to ``6. Accept As Is\ldots{}''. They asked it to
evaluate each paper on a 10-point scale for originality, rigor, scope,
impact, and whether it was `written by AI'. They also (separately) had
LLMs rate 330 papers with the authors' identities removed, or replacing
the names with fake male/female names and real elite or non-elite
institutions (check this) or with prominent male or female economists
attached.

They compare the LLMs' ratings with the RePEC rankings for the journals
the papers were published in, finding general alignment. They find mixed
results on detecting AI-generated papers. In the names/institutions
comparisons, they also find the LLMs show biases towards named
high-prestige male authors relative to high-prestige female authors, as
well as biases towards elite institutions and US/UK universities.
(Doublecheck the details here).

There have been several other empirical benchmarking projects, including
work covered in LLM4SR: A Survey on Large Language Models for Scientific
Research and \href{https://arxiv.org/abs/2502.05151}{Transforming
Science with Large Language Models: A Survey on AI-assisted Scientific
Discovery, Experimentation, Content Generation, and Evaluation}. (We
will discuss these here.)

Zhang et al. (\citeproc{ref-Zhang2025Replication}{2025})

\begin{itemize}
\item
  AI conference paper data
\item
  ``employs LLM agents to perform pairwise comparisons among
  manuscripts''
\item
  ``significantly outperforms traditional rating-based methods in
  identifying high-impact papers'' {[}by citation metrics{]}
\item
  Some evidence of biases/\textasciitilde statistical discrimination
  based on characteristics like `papers from established research
  institutions'
\end{itemize}

Our project distinguishes itself in its use of \emph{actual} human
evaluations of research in economics and adjacent fields, past and
\emph{prospective}, including both reports, ratings, and
predictions.\footnote{Other work has relied on collections of research
  and grant reviews, including NLPEER, SubstanReview, and the Swiss
  National Science Foundation. That data has a heavy focus on
  computer-science adjacent fields, and iss less representative of
  mainstream research peer review practices in older, established
  academic fields. Note that The Unjournal commissions the evaluation of
  impactful research, often from high-prestige working paper archives
  like NBER, and makes all evaluations public, even if they are highly
  critical of the paper.} The Unjournal's 50+ evaluation packages enable
us to train and benchmark the models. Their pipeline of future
evaluations allow for clean out-of-training-data predictions and
evaluation. Their detailed written reports and multi-dimensional ratings
also allows us to compare the `taste', priorities, and comparative
ratings of humans relative to AI models across the different criteria
and domains. The `journal tier prediction' outcomes also provides an
external ground-truth\footnote{About verifiable publication outcomes,
  not about the `true quality' of the paper of course.} enabling a
human-vs-LLM horse race. We are also planning multi-armed trials on
these human evaluations (cf.~Brodeur et al, 2025 and Qazi et al, 2025)
to understand the potential for \emph{hybrid} human-AI evaluation in
this context.

Footnote, a fancier way to say this, from a grant application? Or from
chatGPT?\footnote{Our approach differs from prior work by (i) focusing
  on structured, percentile-based quantitative ratings with credible
  intervals that map to decision-relevant dimensions used by The
  Unjournal; (ii) comparing those ratings to published human evaluations
  rather than using LLM-as-judge; and (iii) curating contamination-aware
  inputs (paper text extraction with reference-section removal and token
  caps), with a roadmap to add multi-modal checks when we score figure-
  or table-dependent criteria.}

\bookmarksetup{startatroot}

\chapter{Data and methods}\label{data-and-methods}

We draw on two main sources:

\begin{enumerate}
\def\labelenumi{\arabic{enumi})}
\tightlist
\item
  Human evaluations from
  \href{https://unjournal.github.io/unjournaldata/index.html}{The
  Unjournal's public evaluation data} (PubPub reports and the Coda
  evaluation form export).\\
\item
  LLM‑generated evaluations using a structured JSON‑schema prompt with
  \texttt{gpt-5-pro-2025-10-06} (knowledge cut-off: 30 September 2024).
\end{enumerate}

\section{Unjournal.org evaluations}\label{unjournal.org-evaluations}

We use The Unjournal's public data for a baseline comparison. At The
Unjournal each paper is typically evaluated (aka `reviewed') by two
expert evaluators\footnote{Occasionally they use 1 or 3 evaluators.} who
provide quantitative ratings on a 0--100 percentile scale for each of
seven criteria (with 90\% credible intervals),\footnote{See their
  guidelines
  \href{https://globalimpact.gitbook.io/the-unjournal-project-and-communication-space/policies-projects-evaluation-workflow/evaluation/guidelines-for-evaluators\#quantitative-metrics}{here};
  these criteria include ``Overall assessment'', ``Claims, strength and
  characterization of evidence'', ``Methods: Justification,
  reasonableness, validity, robustness'', ``Advancing knowledge and
  practice'', ``Logic and communication'', ``Open, collaborative,
  replicable science'', and ``Relevance to global priorities, usefulness
  for practitioners''} two ``journal tier'' ratings on a 0.0 - 5.0
scale,\footnote{``a normative judgment about `how well the research
  should publish'\,'' and ``a prediction about where the research will
  be published''} a written evaluation (resembling a referee report for
a journal), and identification and assessment of the paper's ``main
claim''. For our initial analysis, we extracted these human ratings and
aggregated them, taking the average score per criterion across
evaluators (and noting the range of individual scores).

All papers have completed The Unjournal's evaluation process (meaning
the authors received a full evaluation on the Unjournal platform, which
has been publicly posted at unjournal.pubpub.org). The sample includes
papers spanning 2017--2025 working papers in development economics,
growth, health policy, environmental economics, and related fields that
The Unjournal identified as high-impact. Each of these papers has
quantitative scores from at least one human evaluator, and many have
multiple (2-3) human ratings.

\section{LLM-based evaluation}\label{llm-based-evaluation}

\subsection{Quantitative ratings and journal-ranking
tiers}\label{quantitative-ratings-and-journal-ranking-tiers}

Following The Unjournal's
\href{https://globalimpact.gitbook.io/the-unjournal-project-and-communication-space/policies-projects-evaluation-workflow/evaluation/guidelines-for-evaluators}{standard
guidelines for evaluators} and their
\href{https://coda.io/form/Unjournal-Evaluation-form-academic-stream-Coda-updated-version_dGjfMZ1yXME}{academic
evaluation form}, evaluators are asked to consider each paper along the
following dimensions: \textbf{claims \& evidence}, \textbf{methods},
\textbf{logic \& communication}, \textbf{open science}, \textbf{global
relevance}, and an \textbf{overall} assessment. Ratings are interpreted
as percentiles relative to serious recent work in the same area. For
each metric, evaluators are asked for the midpoint of their beliefs and
their 90\% credible interval, to communicate their uncertainty. For the
journal rankings measure, we ask both ``what journal ranking tier should
this work be published in? (0.0-5.0)'' and ``what journal ranking tier
will this work be published in? (0.0-5.0)'', with some further
explanation.The full prompt can be seen in the code below -- essentially
copied from the Unjournal's guidelines page.

We captured the versions of each paper that was evaluated by The
Unjournal's human evaluators, downloading from the links provided in The
Unjournal's Coda database.

We evaluate each paper by passing the PDF directly to the model and
requiring a strict, machine‑readable JSON output. This keeps the
assessment tied to the document the authors wrote. Direct ingestion
preserves tables, figures, equations, and sectioning, which ad‑hoc text
scraping can mangle. It also avoids silent trimming or segmentation
choices that would bias what the model sees.

We enforce a JSON Schema for the results. The model must return one
object for each of the named criteria including a midpoint rating and a
90\% interval for each rating. This guarantees that every paper is
scored on the same fields with the same types and bounds. It makes the
analysis reproducible and comparisons clean.

We request credible intervals (as we do for human evaluators) to allow
the model to communicate its uncertainty rather than suggest false
precision; these can also be incorporated into our metrics, penalizing a
model's inaccuracy more when it's stated with high confidence.

Relying on GPT-5 Pro, we use a single‑step call with a reasoning model
that supports file input. One step avoids hand‑offs and summary loss
from a separate ``ingestion'' stage. The model reads the whole PDF and
produces the JSON defined above. We do not retrieve external sources or
cross‑paper material for these scores; the evaluation is anchored in the
manuscript itself.

The Python pipeline uploads each PDF once and caches the returned file
id keyed by path, size, and modification time. We submit one background
job per PDF to the OpenAI Responses API with ``high'' reasoning effort
and server‑side JSON‑Schema enforcement. Submissions record the response
id, model id, file id, status, and timestamps.

A separate script polls job status and, for each completed job,
retrieves the raw response, extracts the first balanced top‑level JSON
object, and writes both the raw response and parsed outputs to disk.

\bookmarksetup{startatroot}

\chapter{Results}\label{results}

Here we present preliminary results, starting with a comparison of the
LLM‑generated quantitative ratings (model: \texttt{gpt-5-pro}, see
the\href{methods.qmd}{(previous section}) with human evaluations across
\href{https://globalimpact.gitbook.io/the-unjournal-project-and-communication-space/policies-projects-evaluation-workflow/evaluation/guidelines-for-evaluators\#undefined-1}{the
Unjournal's criteria}.

\section{Quantitative comparison: human vs.~GPT‑5
Pro}\label{quantitative-comparison-human-vs.-gpt5-pro}

We first use the earlier GPT‑5 Pro evaluation run that covered all
papers in our Unjournal sample with a simpler JSON‑schema prompt.
Figure~\ref{fig-forest-overall} shows the overall percentile ratings
from this initial run, averaged across human evaluators and compared to
the LLM's ``overall'' scores for each paper.

\begin{figure}

\caption{\label{fig-forest-overall}Comparison of Human vs LLM overall
percentile ratings}

\includegraphics{results_files/figure-pdf/fig-forest-overall-1.pdf}

\end{figure}%

Figure~\ref{fig-heatmap-human-minus-llm} shows a heatmap of the
differences between human and LLM mean ratings across all evaluation
criteria. Positive values (in green) indicate that humans rated the
paper higher than the LLM, while negative values (in orange) indicate
the opposite.

\begin{figure}

\caption{\label{fig-heatmap-human-minus-llm}Heatmap of Human minus LLM
mean ratings across evaluation criteria}

\includegraphics{results_files/figure-pdf/fig-heatmap-human-minus-llm-1.pdf}

\end{figure}%

\begin{figure}

\caption{\label{fig-forest-tiers-should}Comparison of Human vs LLM
journal tier ratings (should be published in)}

\includegraphics{results_files/figure-pdf/fig-forest-tiers-should-1.pdf}

\end{figure}%

\clearpage

\section{Qualitative comparison: detailed GPT‑5 Pro
evaluations}\label{qualitative-comparison-detailed-gpt5-pro-evaluations}

To understand what GPT‑5 Pro is actually responding to, we re‑ran the
model on four focal papers (\citeproc{ref-Adena2024}{Adena and Hager
2024}; \citeproc{ref-Peterman2024}{Peterman et al. 2024};
\citeproc{ref-Williams2024}{Williams et al. 2024};
\citeproc{ref-Green2025}{Green, Smith, and Mathur 2025}) using a refined
prompt.

This second run keeps the same quantitative metrics but additionally
requires a diagnostic summary of about 1,000 words and high‑effort
reasoning, with the full reasoning trace returned by the ``thinking''
model. For each paper we can therefore inspect:

\begin{itemize}
\tightlist
\item
  the LLM's quantitative scores and journal‑tier predictions,
\item
  the hidden reasoning steps used to arrive at those scores, and
\item
  the token usage and approximate API cost of the evaluation.
\end{itemize}

We start by examining the Williams et al.~(2024) evaluation in detail
and then show the analogous summaries for the other four focal papers.
In the next step we will juxtapose these LLM assessments with the human
evaluators' written reports.

\subsection{Qualitative comparison: Williams et
al.~(2024)}\label{qualitative-comparison-williams-et-al.-2024}

In the refined run, GPT‑5 Pro reads about 28,704 input tokens and
produces 6,327 visible output tokens plus 5,120 reasoning tokens.\\
At current API prices this evaluation costs roughly \$1.80.

The table below shows the model's percentile ratings and 90\% credible
intervals for the Unjournal criteria.

\begin{longtable}[t]{lccc}

\caption{\label{tbl-llm-williams-metrics}GPT-5 Pro percentile ratings
for Williams et al.~(2024)}

\tabularnewline

\toprule
Criterion & Midpoint & Lower 90\% & Upper 90\%\\
\midrule
Overall assessment & 86 & 86 & 86\\
Claims \& evidence & 78 & 78 & 78\\
Methods & 74 & 74 & 74\\
Advancing knowledge and practice & 92 & 92 & 92\\
Logic and communication & 84 & 84 & 84\\
\addlinespace
Open, collaborative, replicable science & 63 & 63 & 63\\
Relevance to global priorities & 94 & 94 & 94\\
\bottomrule

\end{longtable}

\begin{longtable}[t]{lccc}

\caption{\label{tbl-llm-williams-tiers}GPT-5 Pro journal tier ratings
for Williams et al.~(2024)}

\tabularnewline

\toprule
Measure & Score & Lower 90\% & Upper 90\%\\
\midrule
Deserved journal tier (should) & 4.4 & 4.4 & 4.4\\
Predicted journal tier (will) & 4.8 & 4.8 & 4.8\\
\bottomrule

\end{longtable}

For Williams et al.~(2024), GPT‑5 Pro assigns a high overall percentile
score (86/100), with particularly strong ratings for advancing knowledge
(92) and global relevance (94), and a relatively favourable view of
methods (74) and claims and evidence (78). It judges the paper as
deserving publication in a high‑tier journal (tier‑should 4.4/5,
tier‑will 4.8/5).

In its diagnostic summary and reasoning trace (printed below), the model
identifies many of the same issues highlighted by the human evaluators:
heavy reliance on a regrowth dataset with low producer accuracy and
substantial omission error; temporal leakage from contemporaneous
predictors; uncalibrated random‑forest probabilities used to derive the
215 Mha estimate; unrealistically narrow confidence intervals; coarse
predictors driving 30‑m predictions; a liberal definition of land
``available for restoration''; and incomplete uncertainty quantification
for the carbon overlay and permanence. It also notes that code is not
fully open, limiting replication despite open data and published maps.

By contrast, both human evaluators at The Unjournal assign much lower
overall ratings (50/100) and are substantially more critical of methods
and claims: methods scores of 20 and 5, and claims scores of 20 and 5,
respectively. They explicitly argue that methodological problems
``fundamentally challenge the validity and utility of the central 215
Mha estimate of regeneration potential,'' even while rating the paper's
contribution to knowledge and global relevance as high and considering
it suitable for high‑tier journals conditional on major revisions.

Qualitatively, the LLM and human experts are strikingly aligned on what
the main problems are: they converge on concerns about biased input
data, temporal leakage, incomplete probability calibration, under‑stated
uncertainty, domain definition, and the gap between ``biophysical
potential'' and realistic policy use. However, they diverge sharply on
how serious these problems are. The human evaluators treat them as
sufficient to render the headline estimates low‑credibility and heavily
qualified for policy use; GPT‑5 Pro instead regards these issues as
important but ultimately compatible with a high overall rating, robust
methods, and near top‑journal quality.

This case suggests that, at least in this configuration, the LLM can
reproduce sophisticated methodological critiques and uncertainty
language but tends to under‑penalise these shortcomings in its
quantitative ratings, especially for high‑profile, high‑impact work.

\begin{tcolorbox}[enhanced jigsaw, colback=white, colbacktitle=quarto-callout-tip-color!10!white, bottomtitle=1mm, bottomrule=.15mm, opacitybacktitle=0.6, rightrule=.15mm, title=\textcolor{quarto-callout-tip-color}{\faLightbulb}\hspace{0.5em}{Model reasoning trace -- Williams et al.~2024}, left=2mm, opacityback=0, colframe=quarto-callout-tip-color-frame, toptitle=1mm, arc=.35mm, titlerule=0mm, breakable, coltitle=black, leftrule=.75mm, toprule=.15mm]

\textbf{Analyzing model limitations}

I see another limitation in the model's assumptions. They assumed
stationarity from 2000-2016 when predicting for 2015 to 2030. They
included present conditions like 2018 tree cover but didn't factor in
future climate or socio-economic changes. By excluding socio-economic
predictors, they reduced potential model improvements slightly, but this
could bias predictions in areas where socio-economic factors play a
significant role, potentially leading to unrealistic outcomes in
``biophysical potential.'' They clarify it's about modeling potential,
not feasibility.

\textbf{Examining dataset limitations}

I notice another issue with the primary regrowth dataset used, which
came from Global Forest Watch to map regrowth over 5 m. In humid
tropics, the producer's accuracy is low at just 18.7\% because of
omission errors, although user accuracy is high at 85.1\%. The detection
biases may skew the training sample and affect model calibration. Their
conservative approach to mapping regrowth likely favors more accessible
and larger areas, and many small regrowth patches are excluded. It's
unclear if the training sample is representative across different biomes
without proper stratification.

\textbf{Clarifying domain definitions}

I need to define the domain used in the study. They included all parts
of forest biomes within ±25° but excluded water, bare land, urban areas,
and so on, while purposely opting for a liberal domain. They used 2018
tree cover to exclude forests but may misclassify certain savannas or
drylands as part of forest biomes. While they mention not including
savannas, their definition could still capture mosaic-savanna zones.
Additionally, there's a scale mismatch since predictions at a finer 30 m
resolution are computed alongside coarser predictors, which could impact
accuracy.

\textbf{Assessing the weighted area approach}

I see another issue with the weighted area approach, where the
probability of regrowth times pixel area is calculated. However, the
random forest classification is trained on presence/absence from
2000-2016, and the output probabilities aren't properly calibrated.
Calibration techniques like Platt scaling or isotonic regression are
necessary to interpret results as probabilities. They also didn't
present AUC, only accuracy, which may mislead since balanced sampling
artificially sets prevalence to 0.5. Without proper probability
calibration, the estimate of expected area could be questionable and not
reflect actual conditions.

\textbf{Reviewing definitions and calibration}

They used specific definitions, claiming that ``the continuous
potential'' can be seen as the probability of natural regeneration per
pixel, with the weighted-area value representing the expected area for
regeneration. However, they didn't perform any calibration on this.
Their accuracy and out-of-bag accuracy are reported as 0.878, but they
failed to discuss probability calibration, which is important. Regarding
carbon sequestration potentials, they calculated totals and ranges from
previous datasets, but it's unclear where the range comes
from---possibly from Monte Carlo methods or carbon dataset quartile
variation, yet they didn't show how they computed uncertainty.

\textbf{Identifying dataset uncertainties}

The dataset presents uncertainties by grid cell, but their method
multiplies 30-year totals without mentioning how uncertainty is handled.
The ``range'' they provide likely comes from minimum and maximum values
across these uncertainties, which is unclear. They need to clarify how
uncertainty propagates and address independence and omitted below-ground
carbon, since they reference adding 22--28\% from literature.
Additionally, they didn't consider factors like degradation baseline,
leakage, or permanence. While they frame the ``biophysical potential,''
the abstract could mislead policy implications with its stated
above-ground carbon sequestration potential of 23.4 Gt C over 30 years.

\textbf{Checking carbon sequestration claims}

I noticed they mentioned that ``new carbon sinks could increase global
sequestration by 14.3\% per year'' but also stated ``mitigating current
pantropical forest carbon losses by 90.5\% per year,'' which might be
inconsistent. They claim assisted natural regeneration can sequester
significant carbon but didn't show how they calculated this. It seems to
derive from 23.4 Gt over 30 years, equating to 0.78 Gt C per year. With
a net carbon loss of about 0.86 Gt C per year, I need to verify Baccini
2017's values, which reported a net carbon source of 1.1 Gt C per year.
My memory on this is a bit fuzzy!

\textbf{Evaluating carbon loss mitigation claims}

They state that ``mitigating current pantropical forest carbon losses by
approximately 90.5\% per year,'' referencing Baccini, raises some
questions. If regeneration can offset about 0.8 Gt per year, that
suggests total losses of around 0.9 Gt per year. While this might be
plausible, there are logical leaps involved since their potential
benefits might not actually be realized. It's a bit tricky to interpret,
yet they present it confidently as potential. I need to consider these
assumptions carefully!

\textbf{Analyzing dataset issues}

They wisely exclude plantation regrowth from their analysis, but it
seems their training dataset might still contain misclassified
plantations. They reported a previous algorithm with 90.6\% accuracy;
however, an area-based producer accuracy of just 18.7\% in humid tropics
suggests significant under-detection. This non-random classification
error could undermine generalizability. While they argue for the
robustness of the detected patches, the validation accuracy of 87.9\%
relies on potentially biased training dataset presence points. Field
validation is unlikely, and poor accuracy in Southeastern Asia raises
another red flag.

\textbf{Evaluating modeling assumptions}

The modeling approach assumes that predicted `potential' is valid from
2015 to 2030 without considering changes, which is questionable since
the climate is constantly shifting. Additionally, the ``distance to
forest'' will likely change as regrowth occurs, yet they didn't model
this dynamic feedback. They mentioned underestimating impacts like
positive feedback and the exclusion of savanna/dryland. Climate change
could also hinder regrowth, leading to ambiguity. They present a figure
showing ``available for restoration'' area by country, but they don't
clarify how this was calculated, leaving some mystery around the
methodology used.

\textbf{Analyzing definitions and reproducibility}

They define the study's domain as all tropical forest biomes, excluding
areas like water, urban, and existing forests, while including factors
like `distance to water' as covariates. It seems that ``available for
restoration'' refers to the total area of this domain, but their
disclaimers indicate some harvest/regrowth classifications might be
included. Although they mention using a forestry dataset, they didn't
clearly define ``available for restoration.'' Additionally, requesting
code via email isn't ideal; it should be publicly accessible. While they
use accessible datasets, the lack of code hampers reproducibility, and
the confidence intervals aren't credible.

\textbf{Evaluating accuracy and methodology}

They mention the model overcame potential overfitting with an out-of-bag
(OOB) accuracy of 87.8\%. They used a balanced sample and omitted
socioeconomic features due to similar accuracy, but I wonder if the
limited role of socioeconomic data in predicting presence versus
absence, as indicated by remote sensing, could affect results. They also
disclaim that the maps are only one input, which is good. Since this is
published in Nature, it seems high quality, but I believe we should look
beyond that. We need to propose credible metrics.

\textbf{Ranking evaluation criteria}

Based on the content, I would rank high---maybe around the 85th
percentile overall.

For claims and evidence, they articulate strong claims and provide
robust mapping and a predictive model. However, there are issues like
reliance on one training dataset with major omissions and unrealistic
confidence intervals. I'd place this at about 75.

For methods, they've implemented random forest modeling and accounted
for autocorrelation, yet there's heavy reliance on coarse data. This
leads me to perhaps a 70.

Regarding advancing knowledge, their maps on natural regeneration
potential are valuable, globally relevant for restoration planning, so
I'd mark this as a 90.

On logic and communication, while the writing is clear and disclaimers
are present, there are leaps in concluding about carbon losses and some
misuses of ``CI.'' I'd score this around 82.

For open science, while the data is available on Zenodo and the input
data is open, the code is only available on request, which limits
replicability. I'd give a score of 65.

Finally, in terms of global relevance, this work ties in well with
global priorities and quantifies information by country, so I'd rate
this a solid 92.

\textbf{Assigning tier ratings}

I need to assign ``tier\_should'' and ``tier\_will'' ratings from 0 to
5. Based on quality, I think ``should'' deserves a high score, perhaps
around 4.5 or 4.7, due to some identified issues. For ``will,'' since
it's already published in Nature, I'd predict around 4.6 to 4.8.
However, they want continuous numbers, so I'll set ``should'' at 4.4 and
``will'' at 4.7.

If I consider possible underestimation due to method-related issues,
perhaps ``should'' could be 4.2 or 4.4, while ``will'' could be 4.8
because of strong interest. Still, the dataset not being open-coded is a
consideration, but it's published in Nature, so I want to remain fair.

Now, I need to craft a summary, aiming for around 1000 words. I'll start
by summarizing the approach and identifying major issues.

\textbf{Highlighting dataset issues}

I'm noticing that the training dataset has biases, especially regarding
natural regrowth mapping, which shows significant omissions in humid
biomes. This can lead to a non-random sample of presences, and the model
might learn from incomplete detections, distorting variable
relationships and potentially mispredicting areas.

Next, regarding probability calibration and the ``expected area''
method, they treat random forest outputs as probabilities, but without
proper calibration and baseline prevalence, the resulting estimates of
215 million hectares could be inaccurate. They present highly precise
confidence intervals that don't seem credible because they didn't
account for uncertainty across inputs.

Lastly, concerning spatial autocorrelation, while they tried to evaluate
it, accuracy tends to decline with distance, so their optimism might not
be justified. They didn't apply block cross-validation and reported
overall accuracy, avoiding metrics like AUC. The balanced classification
gives an unrealistic sense of accuracy, and there are no confusion
matrices or biome-specific metrics presented beyond overall accuracy.

\textbf{Addressing scale and domain issues}

I see that there's a scale mismatch with the 30 m predictions derived
from 300 m to 1 km predictors, which could cause smoothing and
artifacts. They've acknowledged this with visual evidence in Fig. 1,
suggesting it could impact local planning as they consider these results
to be preliminary.

The domain definition appears overly broad, including areas where
natural regeneration may be undesirable---like non-forest ecosystems,
productive farmland, or communities---which could potentially mislead
their ``available for restoration'' metrics.

Lastly, they assume conditions from 2000 to 2016 without modeling
climate change or socioeconomic factors, which could really undermine
their 2030 predictions. Also, the carbon accumulation overlay from the
Cook-Patton dataset has a scale mismatch issue, with 1 km data being
downscaled to 30 m without considering edges and heterogeneity.

\textbf{Evaluating carbon accounting and code accessibility}

They don't initially account for below-ground carbon, indicating an
additional 22--28\% should be included, and they don't consider the risk
of reclearance. Their discussions on leakage and additionality seem
inadequate since their yearly rates are used to support claims of
significant offsetting of losses, which likely exaggerates realized
mitigation.

While they exclude plantations, algorithmic misclassification might
still exist, and they haven't incorporated a classification probability
step.

Regarding code accessibility, it's not publicly available, making
replication difficult. Although they provide predicted maps on Zenodo,
the training and modeling code needing to be requested further hampers
replicability. Additionally, using 2018 tree cover for distance
predictions while training on 2000 cover could lead to variable shifts,
potentially degrading the predictions since they haven't cross-validated
changes.

\textbf{Observing assumptions and suggestions for improvement}

The weighted area assumption sums across all pixels to determine
``expected area,'' but it overlooks spatial dependence and positive
feedback, which are mentioned but not integrated. The country-level
summarization is good, although it uses administrative boundaries
without addressing ecoregion differences, and there's a lack of error
budget for these country-level sums.

They note that predictions may degrade in Southeastern Asia, but don't
confirm this for later years. For improvements, I should suggest block
cross-validation, probability calibration via isotonic regression, and
incorporating socioeconomic variables.

Overall, I thought the score should be around 85, but I'm reconsidering
an 86.

\textbf{Finalizing ranking and summary considerations}

I'm setting the overall score to 86, considering it's intended to
reflect percentiles within {[}0, 100{]}. For research in remote sensing
global mapping, I'd lean toward an 88, but due to certain issues, I
might adjust this to 82. However, advancing knowledge could be rated
higher at 90, leading me to average around 86, so choosing 87 seems
fair.

I also need to fill out the JSON carefully, ensuring that the assessment
summary is a concise paragraph of about 800-1100 words and omitting any
author names or external knowledge outside the provided PDF. The
confidence intervals will match the midpoint of 86. For claims and
evidence, I'm thinking a score of 79?

\textbf{Setting scores and producing JSON details}

I've decided to set the score of 78 for overall assessment, with
specifics for other categories: 73 for methods, 92 for advancing
knowledge, 84 for logic communication, 62 for open science, and 94 for
global relevance. For tier should, I'll use a score of 4.4, and for tier
will, I'll set it at 4.8, all while keeping consistency in mind with
overall scores at 86, advancing at 92, and global relevance at 94.

The JSON needs to include an assessment summary of around 1,000 words
and metrics for each score calculated with midpoint, lower, and upper
bounds. I'll also avoid naming authors or their institutions, but I can
reference previous analyses without using specific names or citation
numbers.

\textbf{Crafting the summary points}

I'm going to paraphrase the training dataset description, ensuring I
accurately reference the analysis of natural regrowth from 2000 to 2016
without including citation numbers. I should say, ``The study builds on
an existing pantropical remote sensing analysis that mapped natural
regrowth and distinguished it from plantations.''

Next, I want to finalize the summary by outlining key points. For data,
I'll mention potential mapping across tropical forests at a 30 m
resolution, calculating weighted areas for 215 Mha, and noting
variations across ecoregions.

For methods, I'll detail the balanced random forest approach, variable
selection, the training dataset derived from regrowth mapping, and the
overall accuracy. I'll also highlight potential issues like omissions in
the training dataset and domain definition relative to thresholds.

\textbf{Assessing model limitations}

I need to go over some aspects of the model's predictive capabilities.
First, I see that there's a lack of probability calibration in the
balanced training; the predicted probabilities aren't calibrated, which
might lead to bias in the weighted area approach and unrealistic
confidence intervals.

Next, there's the issue of carbon overlay and uncertainty propagation;
downscaling seems to be done naively. The coarser covariate resolution
could degrade the 30 m predictions, causing blockiness. Socio-economic
variables are omitted in terms of feasibility versus potential, and
there's a failure to integrate dynamic processes like climate change
impacts.

I should also note that accuracy in Southeastern Asia is poor, and the
code's availability is limited, impacting replicability due to the heavy
dataset.

Lastly, I'll mention that the additional risk of reclearance isn't
accounted for in the carbon measurements. I need to finalize this while
including suggestions and discussing uncertainties qualitatively as we
maintain strict JSON formatting.

\end{tcolorbox}

\begin{tcolorbox}[enhanced jigsaw, colback=white, colbacktitle=quarto-callout-note-color!10!white, bottomtitle=1mm, bottomrule=.15mm, opacitybacktitle=0.6, rightrule=.15mm, title=\textcolor{quarto-callout-note-color}{\faInfo}\hspace{0.5em}{Diagnostic assessment -- Williams et al.~2024}, left=2mm, opacityback=0, colframe=quarto-callout-note-color-frame, toptitle=1mm, arc=.35mm, titlerule=0mm, breakable, coltitle=black, leftrule=.75mm, toprule=.15mm]

This paper delivers a high-resolution, pantropical map of where
deforested lands have biophysical potential to undergo natural forest
regeneration, and it aggregates this to country and biome summaries and
an overlay-based carbon removal potential. The approach is transparent
in intent---predicting potential, not feasibility---but several
methodological choices and data limitations affect inference strength
and precision. The dependent variable (presence/absence of natural
regeneration during 2000--2016) is derived from a prior global
remote-sensing classification that conservatively emphasized omission
errors in humid regions and required a minimum patch size; area-based
producer accuracy for natural regrowth in the humid biome was low while
user accuracy for detected patches was high. This means the positive
training class is a non-random subset of true regrowth, with systematic
under-detection likely concentrated in particular forest types,
geographies, and patch sizes. The predictive model then inherits these
detection and sampling biases, potentially distorting learned
relationships and shifting mapped potential away from contexts where
regrowth was common but poorly detected. Model fitting uses random
forests on millions of stratified points, with a sensible
variable-selection procedure; predictions are ultimately based on
biophysical predictors alone because a combined socioecological model
yielded very similar accuracy. Omitting socioeconomic predictors makes
the maps more temporally stable and widely available, but it also
reframes outputs as ``biophysical potential'' and risks overpredicting
potential in places where human pressure or governance constraints are
binding. Validation emphasizes overall accuracy around 88\% using a
balanced validation set; however, accuracy declines at moderate
distances from training locations, indicating residual spatial
autocorrelation and some optimism in headline accuracy. The use of
accuracy rather than threshold-free metrics, and the balancing of
classes (prevalence set to 50\%) during training and validation, further
complicates interpretation of both accuracy and the calibration of
predicted probabilities. The paper interprets the continuous model
output as a probability and converts this to ``expected area'' by
multiplying by pixel area and summing, yielding 215 Mha. Yet the
calibration of random-forest scores to true probabilities is not
established; without calibration (e.g., isotonic or Platt scaling) under
the real-world prevalence, the expected-area identity may be biased. The
extraordinarily narrow ``confidence intervals'' reported for area (±0.22
Mha at global scale) are not credible statistical uncertainty; they
reflect computational or rounding artifacts, not propagation of
uncertainty from training data errors, cross-validated prediction error,
input-layer uncertainty, or probability calibration. A threshold
(\textgreater0.5) sensitivity shows a substantially larger area,
highlighting dependence on interpretation choices. Many predictors are
coarse (250 m--1 km) while predictions are at 30 m; this scale mismatch
introduces blockiness and may overstate local precision. The domain of
``available for restoration'' is defined liberally by excluding only
forests, water, urban, bare, and forestry areas; while appropriate for
modeling, its presentation alongside country totals risks being misread
as areas that are socially or legally available for reforestation.
Temporal assumptions also matter: the model projects to 2015/2030 by
assuming that 2000--2016 relationships hold and by updating some
forest-context layers with 2018 data; it does not incorporate climate
change trajectories, shifting fire regimes, or evolving socioeconomics
that strongly affect regeneration and permanence. Carbon estimates are
calculated by downscaling a 1 km accumulation dataset to 30 m,
multiplying by modeled potential, and summing; the stated range lacks a
clearly defined uncertainty model and appears not to propagate
classification, calibration, or permanence risks. The paper is
commendably explicit that realized sequestration will be lower due to
leakage, reclearance, and policy constraints, yet some statements
comparing potential annual removals to current pantropical losses could
be interpreted too strongly given these omitted risks and uncertainties.
Spatial performance is uneven, with lower accuracy in parts of Southeast
Asia; country and biome summaries lack uncertainty decomposition, which
would be useful for decision makers. On replicability, large gridded
outputs are publicly archived and inputs are open, but code is available
only upon request; combined with the volume of preprocessing, this
raises barriers to full reproduction, code review, and extension.
Strengths include a clearly articulated objective focused on actionable
planning, a very large training sample, careful variable selection,
explicit treatment of autocorrelation effects (even if not using full
spatial block cross-validation), and extensive caveats about use and
governance. The resulting layers will be highly valuable as a first-pass
screen for assisted natural regeneration opportunities and for national
restoration dialogues. The main improvements needed relate to
uncertainty characterization (probability calibration, spatially
explicit error propagation, and country-level error budgets), spatial
cross-validation that reduces spatial leakage, clearer separation of
statistical precision from conceptual ranges, dynamic scenario analysis
including plausible climate and human-pressure changes, and full open
release of code. Overall, this is an influential, practically useful
mapping contribution with limitations that mainly concern probabilistic
interpretation, uncertainty, and temporal and socioecological external
validity.

\end{tcolorbox}

\begin{longtable}[]{@{}
  >{\raggedright\arraybackslash}p{(\columnwidth - 10\tabcolsep) * \real{0.2234}}
  >{\raggedleft\arraybackslash}p{(\columnwidth - 10\tabcolsep) * \real{0.1383}}
  >{\raggedleft\arraybackslash}p{(\columnwidth - 10\tabcolsep) * \real{0.1489}}
  >{\raggedleft\arraybackslash}p{(\columnwidth - 10\tabcolsep) * \real{0.1809}}
  >{\raggedleft\arraybackslash}p{(\columnwidth - 10\tabcolsep) * \real{0.1383}}
  >{\raggedleft\arraybackslash}p{(\columnwidth - 10\tabcolsep) * \real{0.1702}}@{}}
\toprule\noalign{}
\begin{minipage}[b]{\linewidth}\raggedright
Paper
\end{minipage} & \begin{minipage}[b]{\linewidth}\raggedleft
Input tokens
\end{minipage} & \begin{minipage}[b]{\linewidth}\raggedleft
Output tokens
\end{minipage} & \begin{minipage}[b]{\linewidth}\raggedleft
Reasoning tokens
\end{minipage} & \begin{minipage}[b]{\linewidth}\raggedleft
Total tokens
\end{minipage} & \begin{minipage}[b]{\linewidth}\raggedleft
Est. cost (USD)
\end{minipage} \\
\midrule\noalign{}
\endhead
\bottomrule\noalign{}
\endlastfoot
Peterman et al.~2025 & 18762 & 7617 & 6208 & 26379 & 1.94 \\
Adena and Hager 2024 & 24234 & 7019 & 5312 & 31253 & 1.84 \\
Williams et al.~2024 & 28704 & 6327 & 5120 & 35031 & 1.80 \\
Kudymowa et al.~2023 & 65096 & 4147 & 2432 & 69243 & 1.77 \\
Green et al.~2025 & 22904 & 5884 & 3904 & 28788 & 1.52 \\
\end{longtable}

\bookmarksetup{startatroot}

\chapter{Discussion}\label{discussion}

\section{Limitations}\label{limitations}

\emph{Sample size and representativeness:} We evaluated on only
\textasciitilde40--50 papers, all in the social sciences and
specifically chosen by The Unjournal for evaluation (which means they
were considered high-impact or interesting). This is not a random sample
of research literature. The papers also skew toward empirical and
policy-relevant topics. The AI's performance and alignment might differ
in other fields (e.g., pure theory, biology) or on less polished papers.

\emph{Human agreement as a moving target:} The Unjournal human
evaluations themselves are not a single ground truth. As evidence of
this, we note substantial variability between reviewers.

\emph{Potential AI knowledge contamination:} We attempted to prevent
giving the AI any information about the human evaluations, but we cannot
be 100\% sure that the model's training data didn't include some
fragment of these papers, related discussions, or even The Unournal
evaluations. We will be able to exclude this for the evaluations of
\emph{future} Unjournal evaluations.

Model limitations and ``alignment'' issues: While powerful, is not a
domain expert with judgment honed by years of experience. It might be
overly influenced by how a paper is written (fluency) or by irrelevant
sections. It also tends to avoid extremely harsh language or low scores
unless there is a clear reason, due to its alignment training to be
helpful/polite -- this could explain the general score inflation we
observed. The model might fail to catch subtle methodological flaws that
a field expert would notice, or conversely it might ``hallucinate'' a
concern that isn't actually a problem. Without ground truth about a
paper's actual quality, we used human consensus as a proxy; if the
humans overlooked something, the AI could appear to ``disagree'' but
possibly be pointing to a real issue.

(There is also evidence, e.g. Pataranutaporn et al.
(\citeproc{ref-Pataranutaporn2025}{2025}) that LLMs show biases towards
more prestigious author names, institutions, and towards male
prestigious authors. We will provide further evidence on this in the
next iterations, de-identifying the work under LLM evaluation.)

Scoring calibration: The AI was prompted to use the 0--100 percentile
scale, but calibrating that is hard. Humans likely had some calibration
from guidelines or community norms (e.g.~perhaps very few papers should
get above 90). The AI might have been more liberal in using the high end
of the scale (hence higher means). In future, a different prompt or
examples could calibrate it to match the distribution of human scores
more closely. We also only took one run from the AI for each paper; LLM
outputs can have randomness, so a different run might vary slightly. (To
do: aggregate across multiple runs.)

Small differences and rounding: Our analysis treated the AI's numeric
outputs at face value. Small differences (e.g.~AI 85 vs human 82) might
not be meaningful in practice -- both indicate a similar qualitative
assessment (``very good''). Some of our metrics (like kappa) penalize
any difference, even if minor. Thus, the ``low agreement'' statistics
might sound worse than the reality where in many cases AI and humans
were only off by a few points. We intend to analyze the distribution of
absolute differences: a large portion might be within say ±5 points
which could be considered essentially agreement in practice. The
credible intervals add another layer: sometimes an AI's score fell
outside a human's interval, but overlapping intervals could still mean
they agree within uncertainty. We did observe that AI's intervals were
often narrower than humans' (LLM tended to be confident, giving
\textasciitilde10-point spreads, whereas some human evaluators gave
20-point or left some intervals blank), which is another aspect of
calibration.

\bookmarksetup{startatroot}

\chapter*{References}\label{references}
\addcontentsline{toc}{chapter}{References}

\markboth{References}{References}

\phantomsection\label{refs}
\begin{CSLReferences}{1}{0}
\bibitem[\citeproctext]{ref-aczel2021billion}
Aczel, Balazs, Barnabas Szaszi, and Alex O Holcombe, {``A billion-dollar
donation: Estimating the cost of researchers' time spent on peer
review,''} \emph{Research integrity and peer review}, 6 (2021), 1--8
(Springer).

\bibitem[\citeproctext]{ref-Adena2024}
Adena, Maja, and Anselm Hager, {``Does online fundraising increase
charitable giving? A nationwide field experiment on facebook,''} 2024
(WZB Discussion Paper Discussion Paper SP II 2020--302r2 (2nd revision
February 2024)).

\bibitem[\citeproctext]{ref-eger2025}
Eger, Steffen, Yong Cao, Jennifer D'Souza, Andreas Geiger, Christian
Greisinger, Stephanie Gross, Yufang Hou, Brigitte Krenn, Anne Lauscher,
Yizhi Li, Chenghua Lin, Nafise Sadat Moosavi, Wei Zhao, and Tristan
Miller, {``\href{https://arxiv.org/abs/2502.05151}{Transforming science
with large language models: A survey on AI-assisted scientific
discovery, experimentation, content generation, and evaluation},''}
\emph{arXiv preprint arXiv:2505.05151}, (2025).

\bibitem[\citeproctext]{ref-Green2025}
Green, Seth Ariel, Benny Smith, and Maya Mathur,
{``\href{https://Elsevier}{Meaningfully reducing consumption of meat and
animal products is an unsolved problem: A meta-analysis},''}
\emph{Appetite}, (2025), 108233.

\bibitem[\citeproctext]{ref-Luo2025}
Luo, Ziming, Zonglin Yang, Zexin Xu, Wei Yang, and Xinya Du,
{``\href{https://arxiv.org/abs/2501.04306}{LLM4SR: A survey on large
language models for scientific research},''} \emph{arXiv preprint
arXiv:2501.04306}, (2025).

\bibitem[\citeproctext]{ref-Pataranutaporn2025}
Pataranutaporn, Pat, Nattavudh Powdthavee, Chayapatr Achiwaranguprok,
and Pattie Maes,
{``\href{https://api.semanticscholar.org/CorpusID:277510314}{Can AI
solve the peer review crisis? A large scale cross model experiment of
LLMs' performance and biases in evaluating over 1000 economics
papers},''} 2025.

\bibitem[\citeproctext]{ref-Peterman2024}
Peterman, Amber, Jingying Wang, Kevin Kamto Sonke, and Janina Steinert,
{``Social safety nets, women's economic achievements and agency: A
systematic review and meta-analysis,''} \emph{Center for Global
Development Working Papers}, (2024).

\bibitem[\citeproctext]{ref-Williams2024}
Williams, Brooke A, Hawthorne L Beyer, Matthew E Fagan, Robin L Chazdon,
Marina Schmoeller, Starry Sprenkle-Hyppolite, Bronson W Griscom, James
EM Watson, Anazélia M Tedesco, Mariano Gonzalez-Roglich, and others,
{``\href{https://doi.org/10.1038/s41586-024-08106-4}{Global potential
for natural regeneration in deforested tropical regions},''}
\emph{Nature}, 636 (2024), 131--137.

\bibitem[\citeproctext]{ref-Zhang2025}
Zhang, Tianmai M, and Neil F Abernethy,
{``\href{https://arxiv.org/abs/2505.23824v2}{Reviewing scientific papers
for critical problems with reasoning LLMs: Baseline approaches and
automatic evaluation},''} \emph{arXiv preprint arXiv:2505.23824},
(2025).

\bibitem[\citeproctext]{ref-Zhang2025Replication}
Zhang, Yaohui, Haijing Zhang, Wenlong Ji, Tianyu Hua, Nick Haber,
Hancheng Cao, and Weixin Liang,
{``\href{https://arxiv.org/abs/2506.11343v1}{From replication to
redesign: Exploring pairwise comparisons for LLM-based peer review},''}
\emph{arXiv preprint arXiv:2506.11343}, (2025).

\end{CSLReferences}

\cleardoublepage
\phantomsection
\addcontentsline{toc}{part}{Appendices}
\appendix

\chapter{LLM evaluation summaries and reasoning
traces}\label{llm-evaluation-summaries-and-reasoning-traces}

\renewcommand\normalsize{\small}

\section{Adena and Hager 2024}\label{adena-and-hager-2024}

\begin{tcolorbox}[enhanced jigsaw, colback=white, colbacktitle=quarto-callout-note-color!10!white, bottomtitle=1mm, bottomrule=.15mm, opacitybacktitle=0.6, rightrule=.15mm, title=\textcolor{quarto-callout-note-color}{\faInfo}\hspace{0.5em}{Model assessment summary}, left=2mm, opacityback=0, colframe=quarto-callout-note-color-frame, toptitle=1mm, arc=.35mm, titlerule=0mm, breakable, coltitle=black, leftrule=.75mm, toprule=.15mm]

This is a large, well-conceived geo-randomized field experiment
assessing whether Facebook video ads increase charitable giving. The
intervention randomized 7,686 of roughly 8,181 German postal codes into
control or treatment over a 14-day campaign, with a 2x2 factorial on ad
content (empathy vs effectiveness) and impression allocation (Facebook
free allocation vs fixed per--postal-code budgets). Outcomes were
measured at the postal-code-by-day level and crucially cover Save the
Children's total donations across all channels, not just click-throughs,
with 31 days pre, 14 during, and 38 post (and some robustness using 10
days of January). The main results show statistically meaningful
increases in donation frequency both during and long after the campaign;
revenue effects are positive and borderline-significant over the
combined during+post window, and not significant in the short term. The
paper also provides evidence of crowd-out for other charities (negative
effects in competitor datasets), documents spillovers to nearby
untreated postal codes, and shows that short-run click/attention metrics
favor the empathy video but do not translate into superior long-run
giving. The design and execution are careful and impressively scaled,
but several limitations and interpretive cautions remain.

Key identification and measurement issues: - SUTVA/spillovers: The
authors detect positive spillovers from treated to untreated areas
(within 30 km), implying the main ITT estimates are lower bounds for
total effects on giving to the focal charity. This strengthens the claim
that ads raise giving to the focal charity but complicates the
interpretation of magnitudes, especially for the decomposition and
competitor analyses. The spillover specification is cross-sectional and
may still pick up spatial confounding; it would benefit from
pre-registered spatial models and placebo checks. - Partial exposure and
dose: Only about one in ten Facebook users in treated postal codes
received at least one impression; intensity varies across postal codes
and treatments, and click data are semi-aggregated for the
free-allocation arm. The study reports ITT effects but provides limited
exploitation of variation in impressions to recover dose--response or
treatment-on-the-treated effects. Given available impression counts, an
IV or continuous treatment analysis could have triangulated effect sizes
and validated the ITT magnitudes. - Aggregation and winsorization:
Outcomes are aggregated to PLZ-day, with winsorization at €1,000 for
PLZ-day sums. Heavy-tailed gifts can meaningfully affect revenue
estimates; while the authors report similar results without
winsorization, the paper relies on the winsorized specification for
precision. Because most outcome analyses aggregate to period totals and
normalize per million inhabitants per day, a small number of large
donations could still influence estimates; supplemental medians,
quantile, or count models would help demonstrate robustness of
distributional claims (beyond the helpful histogram-based evidence on
donation-size categories). - Pretreatment differences and modeling
choices: Pretreatment imbalances are non-significant for Save the
Children outcomes but significant for the 23-charity alliance data. The
preferred specifications include lagged outcomes, fixed effects (where
applicable), and DiD. This is appropriate, but given multiple datasets
and outcomes, a consistent DiD framework (with pre-trends
visualizations) would strengthen comparability and reduce ambiguity
about model-dependent results. - Competitor effects: Evidence of
crowd-out relies on online-only donations for the 23-charity alliance
and project categories on a donation platform for the second dataset.
The former omits offline giving (unlike the focal charity data), the
latter uses project tags (children-related vs other) with potential
misclassification and multiple tagging. Several key coefficients are
only marginally significant (p\textless0.10), and there are many tests
without multiple-comparison adjustments. The paper states crowd-out
cautiously, but its magnitude and scope remain uncertain. Concurrent
campaigns by other organizations or auction-market interference on
Facebook could be part of the mechanism; these are not directly
disentangled from donor-budget substitution. - Profitability estimation:
The immediate revenue--to--ad-spend ratio is 1.45 based on point
estimates, but the 90\% CI is wide and includes substantially lower
values. The long-run ROI assumes a lifetime multiplier of 1.75 for new
donors and similar persistence for existing donors; this is plausible
but not estimated from the experiment. The cost side excludes potential
fixed costs (creative, management) and processing fees. The
profitability conclusion should be framed as ``probable with wide
uncertainty,'' rather than definitive, and sensitivity analyses for
alternative LTVs and cost structures would improve credibility. -
Pre-registration deviations: Blocking also used Facebook reach (not
pre-specified), and the impression-allocation dimension was added.
Excluding the top 1\% and bottom 5\% of Facebook reach also deviates
from a pure census approach. These changes are well motivated (budget
feasibility, power, and algorithmic allocation questions) and
transparent, but they reduce the strength of strict pre-commitment. -
External validity: The experiment ran during the peak German giving
season (mid-November through year-end); effects may differ outside this
period or in other countries/platforms. The campaign was intentionally
largely untargeted. While this increases generalizability to the
population of Facebook users in Germany, it may understate effects
achievable with standard fundraising targeting. The heterogeneity
analysis shows substantial scope for targeting, supporting this point. -
Intermediate metrics vs outcomes: The paper's conclusion that clicks and
short-run attention are poor proxies for long-run donations is
consistent with the presented comparisons, but the inability to trace
individual click-to-donation conversions (due to tracking limitations)
weakens causal linkage claims between specific intermediate metrics and
outcomes. The empirical message remains strong but should be
characterized as ``indicative'' rather than ``conclusive'' regarding
mechanism. - Statistical inference and presentation: Most core findings
are robust and supported by randomization inference. However, many
secondary analyses (competitors, heterogeneity splits, factorial arms)
involve multiple hypothesis tests; the paper could report adjusted
p-values or emphasize effect sizes with confidence intervals to guard
against false positives. - Reproducibility and open science: The study
uses proprietary donation and platform data aggregated at PLZ-day and
semi-aggregated ad metrics. The appendix is helpful, and the design was
preregistered. However, the underlying data and code do not appear to be
publicly available, limiting computational reproducibility and
reanalysis. Clearer replication materials (synthetic data, code,
detailed variable construction) would substantially raise the paper's
value as a building block.

Overall, this is a credible, policy-relevant, and unusually well-powered
field experiment addressing an important question. The strongest
contributions are: (i) measuring total donations across all channels,
mitigating common substitution and measurement biases; (ii) documenting
persistent effects beyond the campaign; (iii) providing evidence of
inter-charity crowd-out; and (iv) demonstrating that click/attention
metrics can mislead campaign evaluation. The main reservations relate to
the uncertainty around revenue magnitudes and ROI, the reliance on
assumptions for lifetime value, the limits of competitor data and
multiple testing, and the inability to estimate dose--response. These
caveats are acknowledged and do not undermine the central conclusion
that online video fundraising, even when broadly targeted, increases
charitable giving to the focal charity. The paper is valuable for both
academics and practitioners and is near the top of recent work in this
space.

\end{tcolorbox}

\begin{tcolorbox}[enhanced jigsaw, colback=white, colbacktitle=quarto-callout-tip-color!10!white, bottomtitle=1mm, bottomrule=.15mm, opacitybacktitle=0.6, rightrule=.15mm, title=\textcolor{quarto-callout-tip-color}{\faLightbulb}\hspace{0.5em}{Model reasoning trace}, left=2mm, opacityback=0, colframe=quarto-callout-tip-color-frame, toptitle=1mm, arc=.35mm, titlerule=0mm, breakable, coltitle=black, leftrule=.75mm, toprule=.15mm]

\textbf{Assessing paper content and structure}

I need to create a JSON assessment following the schema for
``paper\_assessment\_with\_tiers\_v2.'' The document in question is a
WZB discussion paper titled ``Does online fundraising increase
charitable giving? A nationwide field experiment on Facebook'' by Adena
\& Hager. The assessment should cover around 1000 words based only on
the PDF. I'll identify key issues like identification threats, data
limitations, and inconsistencies. There are conflicting instructions on
percentiles and credibility intervals, which is puzzling! I should
clarify that.

\textbf{Preparing structured assessment}

This is indeed an odd constraint! I want to ensure I stick to the
end-of-instructions requirement to avoid any confusion with earlier
guidance. I'll provide a single number for both percentiles and tiers
while still selecting plausible values. For the applied area, I think it
should focus on ``applied and policy stream'' since this is an economics
field experiment on online fundraising. The reference group will include
serious research in marketing and charitable giving. I'll ensure the
JSON is precisely structured without including citations or external
info, just focusing on internal details. It sounds like I need to be
thorough in my diagnostic summary, covering around 1000 words!

\textbf{Summarizing main contributions}

The study employs a nationwide geo-randomized field experiment across
7,686 postal codes in Germany, featuring a 14-day Facebook video ad
campaign for Save the Children. There's a control group of postal codes
that didn't receive ads, utilizing randomization based on donation
potential predictions. It employs a 2x2 factorial design on video
content (empathy vs.~effectiveness) and impression allocation without
individual targeting. The outcomes measured the donation revenue and
frequency per postal code daily, gathered from Save the Children's data
over several months. This analysis provides compelling insights!

\textbf{Analyzing robustness and results}

January 10 data is included for robustness checks. The study uses
complementary data from competitor charities, aggregating information
from a group of 23 charities focused on online giving, along with
donations from the betterplace.org platform for children's projects
versus others.

The main findings indicate that Save the Children sees a significant
revenue increase of €15-18 per million inhabitants per day long-term and
an increase of 0.22 donations per million daily. There are potential
negative effects on competitor charity revenue, suggesting crowding out.
Click metrics alone may not accurately reflect long-term donation
performance.

\textbf{Examining negative interactions and effects}

On the betterplace platform, there's a negative short-term impact on
revenues for children-related causes, which suggests a substitution
effect towards similar causes. The effects appear to be stronger in
areas with higher potential, urban settings, more employment, and a
larger Catholic population, among other factors.

Facebook's free allocation seems slightly better compared to fixed
budgets, and there's a noted short-term difference between empathy and
effectiveness content in donation frequency. Robustness checks are
thorough, but there are concerns about data privacy and the
accessibility of the provided information, likely due to confidentiality
issues.

\textbf{Analyzing units and treatment assignments}

The analysis units are aggregated at the PLZ-day level, which could be
too coarse. Winsorization at €1000 might impact long-tail donations, but
they claim it yields similar estimates. Treatment is assigned at the PLZ
level, but exposure to ads only occurs among Facebook users in those
treated areas, which leads to partial exposure. There are challenges
measuring treatment intensity across PLZs, potentially affecting
dose-response identification. The randomization blocking method
considers estimated potential and reach, but there seems to be a lack of
preregistration for these decisions.

\textbf{Discussing pre-registration and validity issues}

The study mentions changes from pre-registration, but I wonder if these
materially compromise the findings---it seems unlikely. They do address
non-significant pre-treatment differences using lagged dependent
variables and DiD, although they find significant differences in
competitor data. While randomization minimizes selection bias, some
pre-treatment variance might just be noise. The potential spillovers
from treated neighbors to controls raise concerns about selection
contamination and challenge the SUTVA assumption, complicating the
analysis of competitor donations. Lastly, the results are largely
untargeted, which limits generalizability across different contexts and
regions.

\textbf{Evaluating ROI estimates and identification threats}

The ROI estimates are based on a lifetime value assumption of 1.75,
which appears to come from retention rates, possibly from Save the
Children donors or general trends. They assume that the impact of repeat
donors equals that of new donors, but the retention numbers used seem
coarse. There's no clear inclusion of overhead or credit card fees, and
they only account for ad spend costs, not creative content or staff
expenses. They state direct campaign costs at €33,700 but may overlook
additional fixed costs. I'm curious about potential confounding factors
affecting identification.

\textbf{Considering measurement challenges and competitor effects}

The assignment is done at the PLZ level, but competitor charities could
be running concurrent campaigns that may target different audiences.
Randomization should average these effects, but if competitor charities
adjust in response to the Save the Children campaign---like interference
in ad auctions or algorithmic shifts---this could impact treated versus
control PLZs and lead to issues beyond simple donor substitution.

Regarding measurements, they aggregate Save the Children's donation data
by PLZ-day, yet there may be unaccounted donations without proper PLZs.
While they claim to capture data across all channels, the inability to
link immediate donations to clicks due to tracking failures might limit
the analysis. They might be relying on overall totals, which could be
reasonable.

\textbf{Analyzing data and measurement issues}

The data on competitor charities comes only from online giving, while
Save the Children includes both online and offline contributions. This
could mean that some observed substitution effects might reflect shifts
in giving channels. While they measure substitution between online
charities using platforms like betterplace, measurement errors could
impact the estimated effects.

They're testing multiple hypotheses with various layers of heterogeneity
but don't adjust for multiple comparisons, risking Type I errors at the
0.1 p-value level. The regression specifications seem somewhat limited
since they use OLS with robust standard errors, but I wonder if the
aggregated data presents issues like zero-inflated counts or serial
correlation over time. Their winsorization at €1000 per PLZ-day might
lower variance but could also downwardly bias the mean if there's a
heavy tail, although they claim the results remain similar without
winsorization.

\textbf{Evaluating methodology and controls}

They included pre-treatment donations as a control, but I wonder if that
could lead to a ``bad control''? Luckily, it's not influenced by the
treatment, which is good. They decided to exclude extremes of the
Facebook reach variable, which might compromise their ``nationwide''
claim, although they still cover 94\% of postal codes.

When interpreting their findings, a positive and significant spillover
coefficient implies that treated neighbors contribute more, which could
be correlated with unobserved factors like urbanization. However, this
could also be spurious.

There's noncompliance too, as only 10\% of Facebook users in treated
areas saw the ads, which likely dilutes the ITT effect. Balance tests
appear limited, as they only show Table A1. They changed some design
aspects post-registration, noting adjustments like impression allocation
and block reaching, which they should ideally discuss further. Lastly, I
find it interesting they used Stata and should provide code for
reproducibility.

\textbf{Analyzing data transparency and competitor effects}

It's not entirely clear how they handled data, as they used third-party
data and couldn't share it. However, they do provide OSF
pre-registration and an appendix, which suggests data confidentiality
might pose a moderate barrier to reproduction.

When looking at competitor effects, they found a negative impact of -€59
long-term per million/day based on differences, which relies on
controlling for pre-treatment outcomes. Their use of six-block
randomization, with only a 33\% control share, is balanced, and they
even provide maps of treatments, which is great.

They compare ``treatment'' versus ``control'' across the same periods to
ensure balance, and their dataset includes 13,269 donations from the EU
aggregator---so now, I'll need to produce the percentiles.

\textbf{Evaluating study quality and impact}

The design of the study is strong, resembling a near-nationwide field
experiment with robust findings. The results are clear, demonstrating a
positive effect and some substitution, which could be relevant for
practitioners.

I'm considering an overall rating---maybe around 88? That sounds good,
but is it top-of-field? The ROI feels partly uncertain since effect
sizes are moderate. A p-value less than 0.1 for revenue looks promising,
although the competitor substitution p-values are often around 0.1,
which is just okay.

\textbf{Rating the study's quality}

I think I'd give an overall rating of 85.

For claims and evidence, they carefully characterize things and avoid
overclaiming, but there are some leaps in ROI. I'd rate that an 82.

Their methods are solid, with good randomization and blocking, although
they chose to winsorize, which raises potential SUTVA issues. I'd give
that an 84.

The advancement of knowledge is good as well, landing at 80.
Communication is clear, so I'd rate that an 88.

However, for open science, the data and code aren't available, so that
gets a 55.

Finally, in terms of global relevance, I'd rate it 75. I think it has
the potential for a top-tier journal like the Journal of Public
Economics, so I'd aim for around 4.2, which indicates ``top field.''

\textbf{Evaluating the study's journal potential}

I see the study is significant but maybe not quite A-journal quality. It
could fit into journals like JPubE or Management Science, possibly even
ranking in the top field. I'd rate it a 4.0 for potential.

I think ``will'' might be around 3.6. However, the JSON requirements
state that scores need equality across tiers. So, I'll align the
``should'' to 4.0 and ``will'' to 3.7, both requiring equal values.

For my overall ratings, I'm considering:

\begin{itemize}
\tightlist
\item
  Overall: 86
\item
  Claims evidence: 84
\item
  Methods: 85
\item
  Advancing knowledge: 78
\item
  Logic communication: 90
\item
  Open science: 60
\item
  Global relevance: 78
\end{itemize}

I need to keep in mind that we promised quantitative metrics based on my
insights, considering calibrations over the last three years.

\textbf{Summarizing assessment issues}

I want to address several key issues in the assessment summary. First,
there's a SUTVA violation and potential spillovers, which could
underestimate the treatment-control difference.

I see some pre-treatment differences in the competitor dataset, which
they adjust with DiD methods, but it still raises concerns. There are
uncertainties from aggregator-level measurements and fluctuations in
donation levels that lead to borderline significance for revenue,
despite their use of a large sample.

They use a ``lagged dependent variable'' effectively alongside
randomization. However, the ROI calculation using a lifetime value of
1.75 seems overly optimistic, ignoring creative and staffing costs.

The intermediate metrics do not effectively link to outcomes due to
failed tracking. They also missed opportunities to leverage impression
counts at the PLZ level for estimating treatment, which hinders
intensity estimation.

Excluding 6\% of PLZ may introduce bias, although they offer
justifications. Some Save the Children donation data could be
misassigned, affecting the results, yet they mention the potential for
randomness in misassignment.

For competitor measurements on betterplace, potential classification
errors could impact results, and significant negative interactions
appear to be short-term. The analysis includes postal-code fixed effects
for improved accuracy, which I think is a positive step.

The ``share of treated neighbors within 30 km'' measure might correlate
with unobserved factors, but they do use block fixed effects---a good
choice. Their frequency and revenue normalization omit baseline
differences, yet they make necessary adjustments.

While the 2x2 design isn't strongly powered, they acknowledge possible
underpowering, which is thoughtful. They've omitted recipient surveys,
but perhaps that's acceptable.

They used an ``untargeted campaign'' approach, limiting generalization
to typical charity practices, though they do test heterogeneity.

The sample period is narrow, preceding the end-of-year giving surge,
suggesting the effects may vary by month. Their reported profitability
ratio of 1.45 with a wide confidence interval might indicate possible
unprofitability, which they acknowledge.

They discuss how the Facebook algorithm favors cheaper impressions,
yielding improved metrics without significant differences noted, and
recommend caution there. Investigating channel substitution by donation
source shows no negative impacts---a positive finding.

Lastly, they use Stata for randomization inference and presumably share
code, which could help with reproducibility.

\textbf{Assessing the research design}

In this chain of thoughts, I'm highlighting some key points about the
study's design and strengths. It's clear that pre-registration is
beneficial, even though some data from the Alliance and betterplace
might be incomplete. They use this data for measuring substitution,
suggesting there's some evidence backing their approach.

It's worth noting that the winsorization threshold of €1000 per PLZ-day
seems arbitrary, but considering the aggregated donations, it may
provide relevant insights. Standard errors are robust, and while they
don't cluster across neighbor interactions, this isn't a significant
concern.

I see an interesting synergy here: the idea of spillovers to neighbors
implies that the treatment effect (TOT) in treated areas includes
contributions from unaffected donors in control areas, making the TOT
appear larger.

They suggest that further research could investigate the
``impression-of-ask'' for future insights.

Now, I need to craft a diagnostic summary that pinpoint specific issues
while being neutral and clear. I'll start by outlining the design and
contributions, followed by the study's key strengths, which include
geo-randomization, extensive coverage, and robust competitor data. Then,
I'll identify potential threats to validity.

\textbf{Analyzing revenue effects and data limitations}

I'm noting that the study shows a weak total effect on revenue, as the
p-value is only less than 0.10. The ROI estimate seems uncertain with a
wide confidence interval, and the results only reflect immediate income
while being heavily winsorized, which could lead to frequent zeros. This
makes it hard to make strong claims about profitability. There are also
SUTVA/spillover issues and potential algorithmic mis-targeting.

The presence of a positive neighbor treatment effect suggests the main
intention-to-treat (ITT) is a lower bound, complicating inferences about
competitor substitutions and direct effects.

Additionally, only about 10\% of Facebook users received an ad
impression, indicating small ad intensity while treating this as ITT.
The absence of a treatment-on-the-treated (TOT) effect or dose-response
metric means that patterns observed between ``free vs.~fixed'' and
``empathy vs.~effectiveness'' rely on limited power and aggregated
metrics. Also, the Save the Children dataset has limitations, with
potential misassignments concerning post codes and measurement errors
due to its aggregated nature.

\textbf{Exploring competitor analysis and profitability calculations}

I'm focusing on the competitor analysis, where issues arise with the
alliance-only online classification and pre-treatment imbalances related
to betterplace's `children projects.' The negative results might only be
significant at the 10\% level, and there's a lack of multiple testing
corrections. Additionally, unobserved competitor campaigns may distort
substitution effects, making them reflect ad auction displacement rather
than real donor budgets. The instrument employed doesn't seem strong.

Moving to profitability calculations, the analysis relies on the assumed
lifetime value of 1.75, but replicating this would necessitate detailed
code and a full breakdown of costs, which isn't robust against various
alternatives.

Notably, there was a deviation from pre-registration concerning
impression allocation treatment, as well as additional blocking to
maximize external validity. However, this deviation reduces their
ability to make strong prior claims about treatment effects. Finally,
reproducibility is limited due to non-public data and proprietary
metrics, which makes it difficult for third parties to replicate the
results.

\textbf{Assessing external validity and overall metrics}

I'm noting that the study's external validity is limited to Germany and
the end-of-year timeframe. They mention that more donors might be active
during this period, which could affect generalization to other times.

Regarding the explanation of using click data as an inappropriate proxy,
it seems plausible due to the failed tracking of clicks on the Save the
Children site. While it's a reasonable claim, it remains inconclusive.

Now, about the overall midpoint metric, I'm thinking maybe 84? We
previously considered 85, but after recalibrating against the past three
years of significant research, I think I'll actually settle on a higher
score of 88 instead of the earlier 86.

\textbf{Finalizing numerical scores and tier classifications}

I've settled on some final scores: for overall, I'm picking 87. Looking
at the details, I'll assign claims and evidence a score of 83, methods
85, advancing knowledge 79, logic communication 90, open science 55, and
global relevance 78.

For the tier classifications, I'll choose ``should'' as 4.1, marking it
between a top field journal and a marginal A-journal. However, I'll
adjust the ``will'' score to 3.8.

To keep things balanced, I'll ensure the confidence intervals are equal,
setting both ci\_lower and ci\_upper to 4.1 for the ``should'' score.
Now, we'll get started on crafting that 900-1100 word assessment
summary!

\textbf{Crafting the assessment summary}

I need to summarize the study clearly without referencing external
citations or authors, just focusing on ``the paper.'' So, I'll start
with a sentence highlighting that it's a georandomized study with 7,686
PLZ, a 14-day campaign, and a 2x2 factorial design. The dataset covers
Save the Children's full donations across channels, plus competitor data
from alliance and betterplace, aggregated per PLZ.

Next, I'll systematically address the issues: looking at primary effect
sizes, significance, pre-treatment differences, and resolution. I'll
also cover SUTVA and spillovers, the implications for treatment effects,
and the limited exposure intensity where only every tenth Facebook user
saw an impression.

\textbf{Evaluating data and methodology}

I've noticed that the study didn't measure the treatment on the treated
(TOT) across PLZ, and it didn't utilize regression for TOT. The data is
aggregated with winsorization, showing a heavy-tailed distribution,
making it hard to attribute results to individual donors. The competitor
data only includes online interactions related to children's projects,
which appears weak. There's also a risk of Type I error due to multiple
testing.

Regarding profitability, the lifetime multiplier relies on retention
rates but excludes overhead costs and creative expenses, leading to a
wide confidence interval. There are intangible benefits that aren't
fully captured.

The study deviated from its pre-registration by mixing free and fixed
strategies and using reach for blocking, which raises concerns about
analysis-plan drift. External validity is limited to the time of year
studied; effects might not apply in other months or during disasters.
Their robustness includes randomization inference and winsorization, yet
their analytical choices, like using OLS weighting and normalizing per
million inhabitants, could obscure heterogeneity.

Replicability is somewhat compromised since the dataset isn't shared and
the code is unavailable, although it offers semi-aggregated click data.
Lastly, there are limitations in relying on Facebook's estimated reach
and advertising manager exposures, and they could conduct additional
tests for TOT-of-intensity and cross-lags.

\textbf{Creating JSON structure}

So, the instructions tell me that the summary should come before any
scoring and guide uncertainty. I need to create a JSON object that
strictly adheres to the given schema without any extra keys involved.
It's a bit odd that the rules state the percentiles should be in {[}0,
100{]} with the lower\_bound, midpoint, and upper\_bound all equal but
okay, I'll follow that.

For the metrics, I'll set the overall midpoint at 87, with both lower
and upper bounds also at 87.

Then, I'll do the same for the other percentiles while keeping in mind
the tiers from 0 to 5 and ensuring they're all numbers, not strings.
I'll set ``should'' at 4.1 and ``will'' at 3.8.

Next, I'm crafting the summary text, which should be about 1000 words,
aiming for 900-1100. I'll include suggestions like multi-level modeling,
cost-effectiveness, and unifying data. Finally, I need to double-check
that the JSON schema matches correctly and finalize everything.

\end{tcolorbox}

ewpage

\section{Green et al.~2025}\label{green-et-al.-2025}

\begin{tcolorbox}[enhanced jigsaw, colback=white, colbacktitle=quarto-callout-note-color!10!white, bottomtitle=1mm, bottomrule=.15mm, opacitybacktitle=0.6, rightrule=.15mm, title=\textcolor{quarto-callout-note-color}{\faInfo}\hspace{0.5em}{Model assessment summary}, left=2mm, opacityback=0, colframe=quarto-callout-note-color-frame, toptitle=1mm, arc=.35mm, titlerule=0mm, breakable, coltitle=black, leftrule=.75mm, toprule=.15mm]

This manuscript presents a focused theory-informed systematic review and
meta-analysis of randomized controlled trials (RCTs) that attempt to
reduce consumption of meat and animal products (MAP), with the key
contribution being a strict inclusion of behaviorally measured outcomes
assessed at least one day post-intervention and minimum sample
size/cluster criteria. The authors code 112 interventions from 41
studies (\textasciitilde87,000 participants) and use robust variance
estimation (RVE) to account for within-study dependence. The main
quantitative result---a small pooled standardized mean difference (SMD ˜
0.07, 95\% CI {[}0.02, 0.12{]}) for net MAP consumption---with
publication-bias-adjusted estimates near zero, contrasts with more
optimistic prior syntheses that mixed intentions/attitudes and
non-randomized designs. They also distinguish interventions targeting
red and processed meat (RPM), which show larger effects (SMD ˜ 0.25),
and argue this likely overstates net MAP reduction because of plausible
substitution to poultry/fish. The paper is timely, policy-relevant, and
generally careful in interpretation. That said, several methodological
and reporting choices merit clarification or revision.

Study identification and inclusion: The search is pragmatic and
multi-pronged (reviews, citation chasing, crowdsourcing, targeted search
terms, an AI tool, and nonprofit databases). However, the restriction to
English may introduce language bias. It would help to quantify the
contribution of gray literature among included RCTs and to document
screening inter-rater reliability; as written, data extraction and
inclusion decisions were primarily performed by one author, which raises
error risk and subjective bias. The PRISMA diagram is provided, but a
fuller accounting of reasons for exclusion at full-text stage would
improve transparency. The requirement of a pure control is defensible
but excludes some common cafeteria-level interventions where ``usual
practice'' or partial exposure forms the counterfactual; the authors
partly address this via a ``marginal studies'' sensitivity, but the
implications for external validity could be emphasized.

Outcome selection and measurement validity: Restricting to behavioral
outcomes measured =1 day post-treatment is a major strength. Yet, many
included outcomes appear coarse (e.g., binary self-reports of reduction
vs.~no reduction) and self-reported, which are vulnerable to social
desirability and recall bias, especially when messaging primes ethical
or identity frames. The authors code self-report vs.~objective
assessment but do not stratify or adjust by this risk-of-bias dimension;
reporting subgroup results (and whether effects are attenuated with
objective measures) would be useful. The decision to use the longest
follow-up per intervention is consistent with the paper's emphasis on
sustained change, but it may mix widely varying exposure ``doses'' and
contexts. A sensitivity using the earliest non-immediate behavioral
outcome could show whether effects decay.

Effect-size computation and harmonization: The paper uses Glass's
\textless U+0394\textgreater{} (treatment-control mean divided by
control SD), reportedly the ``pre-treatment control group'' SD when
available; this needs clarification. In most included RCTs there is no
natural pre-treatment SD on exactly the same outcome measure used at
follow-up, and using a baseline SD (if measured) may be inappropriate if
outcomes change scale distributions or if the analysis uses post-only
means. If the intended denominator is the post-treatment control SD (the
more typical Glass's \textless U+0394\textgreater), the text should say
so unambiguously. Where means and SDs are missing, conversions from
coefficients/\textless U+03B7\textgreater²/z are standard; however, the
imputation rule for ``unspecified nulls'' (setting the effect size to
0.01) is problematic. This assigns a positive effect where direction is
unknown and could upwardly bias the pooled mean precisely in the range
of small effects under study. A more neutral imputation (e.g., 0 with
imputed variance based on study size) or exclusion with sensitivity
analysis is preferable. The difference-in-proportions estimator for
binary outcomes is reasonable, but details on aligning signs (so that
positive values uniformly indicate reductions in MAP) should be
explicit.

Dependence, clustering, and weighting: The main meta-analytic model uses
RVE via robumeta with small-sample correction, appropriate for multiple
interventions per study sharing a control. However, the paper does not
report the assumed within-study correlation
(\textless U+03C1\textgreater) or show sensitivity to alternative
\textless U+03C1\textgreater{} values; robumeta's weights can change
markedly with \textless U+03C1\textgreater. More importantly, several
studies are cluster-randomized. It is not explicit whether the
effect-size variances were adjusted for clustering (design effects/ICC
or cluster-level summaries). If not, those studies' precision would be
overstated and they would be overweighted. The minimum of 10 clusters
helps but does not resolve variance mis-specification. The authors
should confirm and, if needed, re-estimate vi using cluster-corrected
SEs or design effects and demonstrate robustness.

Publication bias and sensitivity: The use of selection models (favoring
significant positive results) and the ``worst-case''
non-affirmative-only sensitivity are commendable, and the bias-adjusted
pooled effects near zero strengthen the headline conclusion. Yet
selection models and significance-funnel approaches in metafor assume
independence of effect sizes; since interventions within a study share a
control, dependency remains unless the analysis is aggregated to one
effect per study or a multilevel selection approach is used. The authors
should clarify whether selection analyses were performed on
study-aggregated effects and, if not, provide an aggregated sensitivity.
The significance-funnel plot is informative; adding a p-curve or caliper
test would further triangulate.

Moderator and subset analyses: The categorization into four theories and
the RPM vs.~MAP split are conceptually valuable. Because interventions
and messages often overlap across categories and a single study can
contribute to multiple classifications, the authors prudently avoid
causal meta-regressions on theory. Still, they report subgroup pooled
estimates with very small samples in some cells (e.g., choice
architecture: 3 estimates, CI extremely wide). This should be presented
more cautiously (e.g., descriptive only) and, where possible, collapsed
across similar mechanisms. The temporal decline in average effects is
interesting; adding cumulative meta-analysis or meta-regressing effect
size on publication year (study-level, one effect per study) would
better visualize potential decline effects.

Scope and inference: The paper's main claim---meaningful, sustained
reductions in net MAP consumption via current behavioral interventions
remain largely elusive---is supported by the evidence assembled under
strict criteria, and the discussion acknowledges limitations,
substitution, and external validity. The contrast with much larger
estimates when relaxing inclusion criteria is persuasive. However, the
normative framing ``unsolved problem'' should be bounded by the sample:
RCTs with behavior assessed =1 day, minimum size, pure control, in
mostly high-income settings. The scarcity of rigorous choice
architecture RCTs with delayed outcomes is itself a finding but also a
constraint on generalization. The paper is careful not to overclaim
superiority of any theory, and it highlights promising under-tested
interventions and measurement approaches.

Transparency and reproducibility: The authors emphasize computational
reproducibility (Rmarkdown, containerization) and provide a DOI to the
code/data repository. That is exemplary. The main weaknesses are
single-coder data extraction, limited reporting on coding reliability
and adjudication procedures, and lack of a preregistered protocol.
Adding an archived protocol (even post hoc), inter-rater checks on a
subsample, and a full data/codebook mapping each effect-size computation
would further strengthen replicability.

Overall, this is a rigorous, policy-relevant synthesis with conservative
conclusions, appropriate caution about moderators, and strong
open-science practices. Addressing the flagged issues---especially the
``unspecified null = 0.01'' imputation, clarity on Glass's
\textless U+0394\textgreater{} denominator, cluster adjustment in
variances/weights, independence in publication-bias analyses, and
additional risk-of-bias stratifications---would substantially increase
confidence. Even with these caveats, the paper adds valuable clarity to
a crowded literature and provides a realistic baseline for future trials
and policy design.

\end{tcolorbox}

\begin{tcolorbox}[enhanced jigsaw, colback=white, colbacktitle=quarto-callout-tip-color!10!white, bottomtitle=1mm, bottomrule=.15mm, opacitybacktitle=0.6, rightrule=.15mm, title=\textcolor{quarto-callout-tip-color}{\faLightbulb}\hspace{0.5em}{Model reasoning trace}, left=2mm, opacityback=0, colframe=quarto-callout-tip-color-frame, toptitle=1mm, arc=.35mm, titlerule=0mm, breakable, coltitle=black, leftrule=.75mm, toprule=.15mm]

\textbf{Evaluating JSON structure for metrics}

I need to create a JSON object with two fields: an assessment summary
and a metrics object. The metrics object must include fields for each
percentile metric, detailing the midpoint, lower bound, and upper bound.
However, the instructions say that these values must be equal. This is
odd since it doesn't provide a proper range. I have to follow the
instructions closely, even if it feels unusual. I also need to fill
specific categories for percentile metrics and include two tier metrics.

\textbf{Crafting assessment from PDF content}

I need to base the assessment strictly on the content from the PDF the
user provided, which seems to include the entire paper and its
references. Since the authors' identities shouldn't influence our
judgment, I'll avoid mentioning them in the summary. My goal is to write
about 1000 words of a diagnostic summary that encapsulates the key
points from the text. I want to make sure I'm focusing on the content
itself without biases related to the authors.

\textbf{Analyzing assessment aspects}

I need to evaluate various components such as identification threats,
data extraction methods, and SMD calculation approaches, including the
use of Glass's delta and RVE. I also have to consider publication biases
and restrictions on RCTs, like group sizes and measurement timeframes.
Notably, they've set the ``unspecified null'' to 0.01 instead of 0,
which could introduce bias by implying a small positive effect. This
seems problematic, and I should definitely flag it for further
attention.

\textbf{Evaluating measurement concerns}

I need to assess the use of Glass's \textless U+0394\textgreater, which
considers pre-treatment control group standard deviation. In many RCTs,
there aren't pre-treatment measures available, and they seem to refer to
the baseline. This could lead to ambiguity since they may have relied on
pooled SD. There's also the consideration of positive sign alignments
with decreases in MAP that I need to verify. Additionally, the use of
RVE with small-sample corrections raises concerns, especially with 41
studies and 112 estimates. They might have missed immediate effects by
focusing on longer follow-ups, which could introduce biases.

\textbf{Analyzing assumptions and biases}

I'm considering the assumption that they've restricted to ``net MAP or
RPM consumption.'' They mentioned some outcomes are quite coarse, like
binary reduction versus no reduction, and they included self-reports
without adjusting for risk-of-bias variables. They didn't present the
Egger test either, which could be a concern. Their use of selection
modeling assumes a p\textless0.05 threshold, and there are potential
issues with correlated sampling errors. Although they've applied RVE to
address dependence, there might be critiques regarding how they handle
these correlations in their selection modeling.

\textbf{Addressing multi-analytic classification}

I see a potential issue with their multi-analytic classification of
categories. They haven't performed meta-regression but only provided
pooled SMD within categories. This approach might lead to dependencies,
especially since they have multiple effect sizes per study. Although
they used RVE for all 112 effect sizes, their subset analyses lack
meta-regression control. Their reasoning for not conducting
meta-regression---due to studies with multiple interventions crossing
categories---makes sense to avoid double counting, but calculating
pooled effects per category still raises concerns about how multiple
contributions are accounted for across different subsets.

\textbf{Questioning unspecified null and clustering issues}

I'm noticing that coding the ``unspecified null'' as 0.01 might be
problematic. With 112 effect sizes, they set the unspecified null to
0.01 instead of 0 when they lack enough information to compute SMD,
which could introduce a slight upward bias. They might have chosen 0.01
to avoid exactly zero for log transformations, but that doesn't seem
necessary for SMD. Additionally, their use of the Bernoulli difference
in proportion estimator might not properly account for cluster-level
correlations. I need to check if they adjusted for clustering in their
analyses since they mentioned requiring at least 10 clusters.

\textbf{Evaluating adjustment methods and standard deviations}

I noticed they mentioned ``first author conversions using standard
techniques,'' but they didn't seem to adjust for cluster design
adequately, possibly due to limited information. The use of
``pre-treatment control group SD'' is questionable, especially since
many RCTs measure consumption at follow-up. Referring to a baseline SD
for SMD at follow-up feels inconsistent---I think it should really be a
pretest measure. Using baseline variation might reduce bias, but if the
scale changed over time, it could complicate things. Hence, I'll flag
this as a possible misprint or a methodological concern.

\textbf{Reviewing SMD and data accessibility}

I'm noticing that they used sign-coded SMD to indicate a reduction in
MAP consumption, but they didn't clarify the sign of the effect. They
referenced ``108 RCTs and 112 interventions,'' but their forest plot
includes both positive and negative effects. They mentioned ``87k
subjects,'' which might include cluster-level counts, but I see they
didn't adjust for compliance or attrition, and it's unclear how missing
data was handled. While they provide their code and dataset along with a
DOI for full documentation---which is great for open science---I need to
check if the code and data are truly accessible for evaluation and
replicability.

\textbf{Evaluating data handling and biases}

They mention that the data was extracted by the first author, which
could lead to single-coder errors. While they consulted authors and used
publicly available datasets, they didn't double-code or verify effect
size calculations independently. This raises concern about potential
biases. They included some dissertations and grey literature but
restricted their analysis to English publications, which could introduce
bias as well. Notably, they included marginal studies in their
sensitivity analysis, showing that these can yield larger effect sizes,
although they advise caution in interpreting these findings.

\textbf{Assessing RVE and bias considerations}

They employed ``random intercept RVE robumeta,'' which includes a
small-sample correction, but I'm noticing that they used a default `rho'
of 0.8 without specifying what weighting they applied. This is
important, as variations in rho could change results, and they didn't
vary their assumptions between 0 and 1. They also didn't conduct a
robust meta-reg with cluster-level correlations. On the positive side,
they correctly identified ``worst-case publication bias'' yielding a
small effect, which is useful information for understanding their
findings.

\textbf{Reviewing inclusion criteria and considerations}

They included ``two choice architecture studies with only 3 estimates,''
which raises some questions. They speculated on reasons for this,
mentioning a lack of long-term or hypothetical outcomes, but their
threshold for inclusion seems to exclude interesting cases like
``defaults at buffets,'' even if the control had a baseline. On a
positive note, they did utilize the longest follow-up measure for each
intervention, which is good. However, it's curious that they didn't
discuss ``nudging philanthropic disclaimers.''

\textbf{Analyzing bias and intervention effects}

One nuance here is that they used a ``PublicationBias'' function that
only considers non-affirmative results for worst-case estimation.
However, with Random Variance Estimation (RVE) not being independent,
there might be some issues. They employed metafor's generic function
rather than RVE, which could disrupt the correlation. It seems they
didn't adjust for dependent effect sizes, but it's acceptable since each
of the 112 effect sizes corresponds to distinct interventions. Multiple
interventions sharing control within the same study are treated as
dependent, and for publication bias, they didn't make any adjustments.
Still, there isn't a huge difference overall.

\textbf{Evaluating effect size computation}

They mention using a ``difference in proportion estimator,'' which
treats discrete events as Bernoulli distribution draws---this seems
acceptable. However, they didn't specify the ``units'' for Standardized
Mean Difference (SMD) across different measurement scales, though they
addressed it with standardization. Another concern is that they included
cluster RCT experiments, but I'm wondering if they computed effect sizes
with cluster-level standard errors. If they didn't, that could affect
independence and variance. While they employed robust RVE, they didn't
clarify adjustments for the design effect, which is crucial. I'll flag
this issue.

\textbf{Assessing sample size and weighting}

They included approximately 87,000 participants, which is significant,
but the `units-of-randomization' might be less. It's great they set a
minimum of ten clusters, but they didn't weight by the number of
clusters. Instead, they might have weighted by `within-study variance'
assuming cluster-level data. If sample-level isn't adjusted, the cluster
studies could appear more precise than they really are. Also, they
included diaries or receipts for outcomes, which is good. However, they
noted the lack of a pre-analysis plan without incorporating that into
their risk-of-bias weighting. The SMD of 0.21 to 1.42 seems off because
it suggests only three estimates, indicating significant imprecision,
which is worth mentioning.

\textbf{Evaluating research score}

They discuss four categories but didn't specify doping, which is worth
noting. It's good they acknowledge the `lack of causality.' Now, I need
to produce numerical percentile scores for each metric. Overall, I'd
probably place it around 75\%, maybe 70\%. The paper demonstrates rigor
in its meta-methods and addresses relevant global priorities, but there
are methodological concerns, such as single coding and unclear cluster
adjustments.

Based on the strength of the open science aspects, like sharing code and
datasets, I would assign an overall score of 80\% and consider 90\% for
the open science component.

\textbf{Assessing claims and evidence}

I think the single-coder issue affects replicability, so maybe I'd score
it around 85. They make claims about unsolved problems and provide
balanced evidence, which is good. It's also a positive point that they
approach substitution issues carefully and restrict their analysis to
RCTs with long-run outcomes. However, the mention of `worst-case bias'
raises a flag. Overall, the claims seem measured, so I'd settle on
giving this section around 80\%.

\textbf{Evaluating various aspects}

There are definitely some issues, like the unspecified null of 0.01
being a bit unnatural and concerns about independence in selection
models, leading me to reduce the score to 76. For the methods, they're
good overall, but I'd rate them around 72 given the issues.

When it comes to advancing knowledge, they effectively challenge earlier
positive claims, so I would give it a solid 78. The logic and
communication are straightforward, with good visuals, so I'd score that
an 85. Their commitment to open science deserves a 90, and the topic's
global relevance is strong, earning an 88. Considering the cost/benefit
discussion, I'd settle on around 80.

\textbf{Determining journal rankings}

I need to ensure the credible intervals are adjusted for clarity, where
the lower bound equals the midpoint. Initially, the instructions called
for including credible intervals, but now they indicate to set them
equal, so I'll follow that.

Now, regarding the ``should/will'' scale from 0-5: for the ``should''
classification, I'd place it around 4.3 because it might fit into a top
field journal like ``PNAS'' or ``Environmental Behavior.'' For the
``will'' classification, I'll estimate it as 3.9 since many
meta-analyses are featured in solid B-journals like ``Food Policy.'' So,
that feels balanced!

\textbf{Preparing the analysis summary}

The final instructions direct me to set the score equal to both the
lower and upper credible intervals, so I'll clarify that point. Now, I'm
ready to create the 1,000-word summary while maintaining a neutral tone
and focusing on specific details.

I'll begin with the clarity of the question, noting that it's a
meta-analysis of RCTs aimed at reducing MAP with long-term measures,
highlighting the 41 studies and 112 interventions involved.

Key concerns include:

\begin{enumerate}
\def\labelenumi{\arabic{enumi})}
\tightlist
\item
  Study selection: It's limited to English only as of December 2023,
  considering published and grey literature but with a customized search
  strategy.
\item
  Outcome measurement: They shifted to net MAP vs.~RPM and included
  self-reported measures, which may be coarse.
\item
  Data extraction: A single coder was used, raising the risk of
  measurement errors and unspecified null imputation at 0.01.
\item
  Effect-size conversions: They utilized Glass's
  \textless U+0394\textgreater{} with the pre-treatment control group
  SD.
\item
  RVE modeling: The robumeta method was used for analyzing 112 effect
  sizes.
\end{enumerate}

This outline will guide my summary!

\textbf{Structuring the analysis summary}

I need to continue refining the summary analysis, focusing on specific
metrics. I have points like publication bias, where they used a
selection model with worst-case ``non-affirmative only'' data. Also, I
should consider heterogeneity and generalization, and recognize any
dangers in category-level pooling. For follow-up times, there's a
mention of the ``longest follow-up'' and the need to correct for cluster
assignment.

Now, I'm preparing to write around 800 to 1,100 words while crafting the
JSON for the assessment metrics. The
``Paper\_assessment\_with\_tiers\_v2'' schema should include nested
objects for ``tier\_should'' and ``tier\_will'', both containing scores,
lower bounds, and upper bounds.

Since the spec says to keep bounds equal, I'll make sure to set all
percentiles properly, ensuring it's valid JSON without any trailing
commas, and the property order isn't crucial. That will keep everything
neat and accurate!

\textbf{Selecting values for assessment}

Alright, I'm choosing values to present for the analysis. I'm thinking
an overall score might be around 82, which feels consistent with the
summary. For the individual categories, I could go with claims and
evidence at 78, methods at 74, advancing knowledge also at 80, and logic
and communication at a solid 86. For open science, I think 92 is
appropriate, and global relevance could sit at 85.

For tiers, I'll set `tier\_should' at 4.3 and `tier\_will' at 4.0. Now,
I just need to keep focused while crafting the 1,000-word summary,
maintaining clarity and robustness throughout!

\end{tcolorbox}

ewpage

\section{Kudymowa et al.~2023}\label{kudymowa-et-al.-2023}

\begin{tcolorbox}[enhanced jigsaw, colback=white, colbacktitle=quarto-callout-note-color!10!white, bottomtitle=1mm, bottomrule=.15mm, opacitybacktitle=0.6, rightrule=.15mm, title=\textcolor{quarto-callout-note-color}{\faInfo}\hspace{0.5em}{Model assessment summary}, left=2mm, opacityback=0, colframe=quarto-callout-note-color-frame, toptitle=1mm, arc=.35mm, titlerule=0mm, breakable, coltitle=black, leftrule=.75mm, toprule=.15mm]

This commissioned review provides a clear, well-scoped examination of
discounting choices used in philanthropic global health
cost-effectiveness, with the immediate goal of advising a specific
decision-maker. The manuscript's core contributions are: (i) retaining
an SRTP framing while articulating why SOC and harmonizing approaches
are less suitable in this context; (ii) clarifying the unit being
discounted in GiveWell-style modeling (consumption growth rather than
levels) and deriving an explicit, adapted wealth-effect formula r = d +
(\textless U+03B7\textgreater{} - 1)g; (iii) identifying and correcting
a minor computational error in the prior spreadsheet implementation;
(iv) proposing a defensible, transparent procedure to estimate g using a
population-weighted SSA+SA composite and a time-window anchored to
program effect duration, with an update cadence; and (v) highlighting an
important internal inconsistency: using \textless U+03B7\textgreater{} =
1 in the utility implicit in CEAs (log utility) while using
\textless U+03B7\textgreater{} = 1.59 for the wealth-effect component,
which can materially alter the overall rate. The report also reviews
discounting for health outcomes, pure time preference, temporal
uncertainty, and a ``compounding non-monetary benefits'' component. The
overall structure is logical and the reasoning is mostly transparent,
with useful appendices on derivations and institutional practice.

Key issues and limitations arise primarily from evidence depth,
parameterization rigor, and potential internal inconsistencies that are
flagged but not resolved. First, the choice to keep SRTP over SOC is
broadly reasonable for cross-intervention welfare comparisons, but the
paper does not quantify the decision stakes of alternative approaches
(e.g., implementing an SOC-based sensitivity case or the
SPC/weighted-average harmonization to show how rankings or absolute
valuations would change for representative interventions). Given the
centrality of this decision, readers would benefit from explicit
comparative numerics under plausible parameterizations and program
archetypes (consumption vs mortality vs morbidity outcomes) and explicit
discussion of potential crowding-out magnitudes for philanthropic funds
across feasible counterfactuals.

Second, the adaptation r = d + (\textless U+03B7\textgreater{} - 1)g is
a valuable clarification of what is implicitly being discounted in these
CEAs; however, the implications are only partially explored. The report
rightly notes that adopting a consistent \textless U+03B7\textgreater{}
across the entire modeling stack is crucial to avoid overdiscounting,
and that if \textless U+03B7\textgreater{} = 1 were used consistently,
the wealth effect falls to zero. Because this change alone can shift the
overall rate by \textasciitilde1.7--2.0 percentage points and
potentially alter program rankings, the absence of a worked
re-estimation of a few headline CEAs (or at least a stylized
demonstration of ranking sensitivity) leaves a major decision lever
underexplored. Relatedly, the suggestion to include either donor or
beneficiary pure time preference, but not both, is correct; yet, no
quantitative illustration is provided for plausible donor-weighting
choices, which would illuminate how sensitive choices are to stakeholder
perspective.

Third, the proposed method for estimating g (3.3\%) is transparent but
rests on several judgement calls that deserve more robustness checks.
The SSA+SA population-weighted composite and 40-year window tied to
deworming are reasonable, but the paper itself flags a puzzling result:
the composite CAGR lying below that of both components over part of the
horizon and tracking SA closely; this could arise from weighting,
smoothing, or interpolation choices. The authors acknowledge limited
time to resolve the anomaly, which weakens confidence. Additional
triangulation would strengthen the recommendation: e.g., comparing
alternative global forecasts (beyond IMF and Goldman Sachs), decomposing
growth into productivity and demographics, checking sensitivity to
funding-allocation weights versus population weights, and testing
alternative horizon widths (20, 30, 50 years) and start/end-year
anchoring. The proposed 5.2-year update cadence is pragmatic, but the
threshold (0.1 pp change warranting an update) is ad hoc; a
justification aligned with decision-value or threshold
cost-effectiveness changes would be preferable.

Fourth, recommendations on non-consumption components are mixed in
strength. For d, the report synthesizes philosophical and empirical
arguments well and makes a defensible case for d = 0 in this
application. For temporal uncertainty (1.4\%), and especially for the
``compounding non-monetary benefits'' (0.9\%), the evidential base is
thin. The temporal uncertainty component is broadly aligned with other
estimates but mixes existential/catastrophic risks with other pathways
to benefit non-realization in a way that risks double counting with
other parts of the CEA (e.g., baseline mortality, fade-out) unless
carefully separated. The 0.9\% non-monetary compounding term is
acknowledged to be weakly justified and partly a historical compromise;
no alternative modeling frameworks (e.g., explicit dynamic household
investment pathways, or SPC-style reinvestment for the relevant subset
of benefits) are trialed. Given these components together account for
more than half of the overall rate in some parameterizations (especially
if \textless U+03B7\textgreater{} were set to 1), their uncertainty
deserves more structured analysis (e.g., priors, ranges, and
decision-relevant sensitivity).

Fifth, the health-vs-consumption discounting section captures the state
of the literature accurately: equal discounting remains common, but
arguments for discounting health at a lower rate are stronger
conceptually; nevertheless, the report does not specify a practicable
health rate beyond continuing to apply temporal uncertainty only. This
conservative, ``status-quo-with-justification'' recommendation is
understandable, but it would be more decision-useful to propose at least
a small set of candidate health rates with transparent assumptions
(e.g., under additively separable utility versus specific non-separable
forms), plus implications for representative interventions.

Sixth, while the manuscript is technically careful---the correction of a
spreadsheet formula and explicit derivations are strong
points---reproducibility for applied users depends on linked
spreadsheets and external projections. The report references online
resources but does not embed full data snapshots or code within the
document; version control, a frozen dataset appendix, and a worked
example reproducing the 4.3\% rate calculation end-to-end would enhance
replicability. Some internal consistency concerns remain: clear
separation of units (utils vs consumption levels vs log consumption) is
emphasized, but readers might still find it challenging to trace how
each component maps to the unit actually discounted in different CEA
modules. Finally, the institutional scan is informative but shallow by
design (\textasciitilde10 hours), relying on convenience samples and
email inquiries; conclusions drawn from that scan should be treated as
suggestive rather than definitive.

Overall, the report advances clarity on a few pivotal modeling choices
(unit of discounting, \textless U+03B7\textgreater{} consistency,
explicit wealth-effect formula, and a more principled g estimate and
update process), provides a balanced and transparent discussion of
contested components, and responsibly flags uncertainties. The main
weaknesses are the limited depth of quantitative robustness, reliance on
a few external forecasts without triangulation, unresolved anomalies in
growth aggregation, and the absence of sensitivity demonstrations on
headline program rankings for high-stakes parameter changes (notably
\textless U+03B7\textgreater{} and the inclusion/exclusion/magnitude of
the 0.9\% component). These limitations are typical for an applied
advisory report on a tight timeline, but they constrain
generalizability. With additional analyses that quantify
decision-sensitivity and resolve the growth-aggregation puzzle, the work
could serve as a stronger reference for the broader global health CEA
community.

\end{tcolorbox}

\begin{tcolorbox}[enhanced jigsaw, colback=white, colbacktitle=quarto-callout-tip-color!10!white, bottomtitle=1mm, bottomrule=.15mm, opacitybacktitle=0.6, rightrule=.15mm, title=\textcolor{quarto-callout-tip-color}{\faLightbulb}\hspace{0.5em}{Model reasoning trace}, left=2mm, opacityback=0, colframe=quarto-callout-tip-color-frame, toptitle=1mm, arc=.35mm, titlerule=0mm, breakable, coltitle=black, leftrule=.75mm, toprule=.15mm]

\textbf{Evaluating estimation methods}

I'm exploring the ``time window'' method for estimating growth using IMF
WEO and Goldman Sachs data, updated every five years. They anchor this
with a 40-year window linked to deworming, resulting in an estimate of
3.3\% growth. However, there are issues with composite CAGR and heavy
reliance on Goldman Sachs's techniques. Their open science aspects
include shared calculations but lack replicability due to missing links.
Though their review process involved experts, it's not exhaustive, and
the report serves more as a practical tool than a novel academic
contribution.

\textbf{Clarifying norms and descriptions}

I need to ensure there's a clear distinction between normative and
descriptive statements while providing credible disclaimers. This
separation will help prevent any confusion about what's being presented
as a standard versus what's merely an observation. It's essential to
present information transparently, making it easier for users to
understand the context and implications of the data and insights shared.
Overall, I want to maintain clarity and credibility in my communication.

\textbf{Identifying potential issues}

I see several potential issues in the analysis. First, the argument for
adopting SRTP over SOC is weakly formalized. There's uncertainty in the
``health vs consumption discounting'' claims, which lack strong
guidance. The temporal uncertainty is broadly defined, and without first
principles calculations, it's hard to evaluate. Additionally, the
non-monetary benefits appear poorly grounded in theory. The consumption
growth reasoning shows puzzling results that they can't explain. There's
inconsistencies in utility curvature, but they do provide some positive
aspects like referencing derivatives. Overall, the methodology has its
merits but also significant gaps.

\textbf{Assigning percentiles for evaluation}

I need to assign percentiles for the evaluation. I'm thinking of giving
an overall score of 70\%. For claims and evidence, maybe 72\%, and
methods could be around 65\% since they didn't conduct rigorous
empirical analysis, but it's reasonable for a policy review. I think
methods deserve 68\%. For knowledge advancement, I'll lean towards 60 or
65\%. The logic and communication are good---maybe a solid 78\%. Open
science transparency could get 65\%, and I see high global relevance, so
I'm giving that an 80\%. Regarding publishing, this commissioned report
might fit well in a top B-journal instead of a standard journal.

\textbf{Evaluating journal suitability}

Many economics and health economics journals might demand more formal
modeling, but this report could be accepted in fields like development
practice or health policy analysis. I'm leaning towards a score of 3.2
for the ``should'' rating, aiming for a top B-journal or strong field.
For ``will,'' I think a 2.8 is appropriate since it's a report rather
than a journal article. Given the thorough nature of the document, I'll
finalize ``should'' at 3.1 and ``will'' at 2.7. However, the guidelines
call for credible intervals that match the scores. Got it!

\textbf{Creating a summary}

I'm preparing to craft a summary around 1000 words, but it feels like a
lot! I'll aim for approximately 900 to 1100, focusing on issues that
need specifics: the approach used, key contributions, and conceptual
mismatches. I'll cover the wealth effect formula and inconsistencies.
Topics include the reliance on Goldman Sachs, demographic oversights,
measurement of time, and errors in double counting. I'll also mention
identification threats and the lack of a systematic review, which may
further limit replicability. Finally, I'll fill out the JSON structure
as needed.

\textbf{Crafting the summary}

I'm ready to craft the assessment summary while keeping my percentiles
in mind. Overall, I'm setting it at 72. For claims and evidence, I think
74 is fair but might stick with the earlier 72. The methods score will
be 66, and advancing knowledge gets a 64. For logic and communication, I
considered 80 but will settle on 78. Open science can be 64, and global
relevance is set at 82. I will keep the tier ``should'' at 3.0 and
``will'' at 2.7, producing a thorough summary while avoiding individual
names and citations as required.

\textbf{Identifying key criticisms}

I'm noticing several key criticisms in the analysis. First, while the
conclusion favors keeping SRTP over SOC, it lacks quantified trade-offs
for SOC and could benefit from a demonstration on a sample program. The
derived formula they present seems helpful, but they didn't fully
re-estimate the entire cost-effectiveness analysis. Raising the growth
rate to 3.3\% relies heavily on external sources, which introduces
uncertainty without addressing the composite CAGR issue. The
recommendations on temporal uncertainty and compounding non-monetary
benefits also seem weak, alongside concerns around double counting and
the lack of replication options.

\end{tcolorbox}

ewpage

\section{Peterman et al.~2025}\label{peterman-et-al.-2025}

\begin{tcolorbox}[enhanced jigsaw, colback=white, colbacktitle=quarto-callout-note-color!10!white, bottomtitle=1mm, bottomrule=.15mm, opacitybacktitle=0.6, rightrule=.15mm, title=\textcolor{quarto-callout-note-color}{\faInfo}\hspace{0.5em}{Model assessment summary}, left=2mm, opacityback=0, colframe=quarto-callout-note-color-frame, toptitle=1mm, arc=.35mm, titlerule=0mm, breakable, coltitle=black, leftrule=.75mm, toprule=.15mm]

This registered systematic review and meta-analysis assembles an
unusually large experimental evidence base (93 studies, 1,307 effects
across 45 LMICs) on social safety nets (SSNs) and women's economic
achievements and agency, with clear inclusion criteria, broad SSN
coverage, and careful outcome coding. The analytic strategy
(standardized mean differences converted to Hedges' g; robust variance
estimation {[}RVE{]} to handle within-study dependence; multilevel RE
and influential-point checks as robustness) is appropriate and largely
well executed. That said, several features limit interpretability and
design guidance. First, heterogeneity is extremely high throughout (I2
commonly 80--95\%+; t2 \textgreater{} 0.02), yet the paper emphasizes
pooled mean effects without reporting prediction intervals or
distributional summaries of true effects. Given the breadth of contexts,
modalities, and outcomes, a single pooled g˜0.11 risks overgeneralizing
and is of limited policy value without a sense of between-study
dispersion and likely range in new settings. Meta-regressions recover
few moderators and retain high residual heterogeneity, suggesting either
limited power to detect realistic moderators, substantial measurement
noise in moderator coding, or genuine context dependence. Second,
outcome harmonization is ambitious but stretches conceptual
comparability. ``Economic achievements'' and ``agency'' are
multi-dimensional constructs; the paper pools across diverse indicators
(including binary and continuous) via SMDs. Although standard, SMDs
reduce interpretability for practitioners (e.g., how large is g=0.11 in
hours worked or decision-making probability terms?) and may mix
substantively different measures under a common metric. Additional
stratification by measurement type (binary vs continuous; index vs
single indicator) and presentation of effects in natural units where
feasible would improve external usefulness. Relatedly, it is unclear
whether aggregate indices and their constituent components were ever
both included for the same sample/time; explicit safeguards against
double-counting would be helpful. Third, the dependence structure is
handled with RVE, but weighting details are under-specified (e.g., exact
weight formula, assumed within-study correlation
\textless U+03C1\textgreater, and sensitivity to alternative
\textless U+03C1\textgreater{} values). With 1,307 effects from 93
studies, a few studies likely contribute disproportionate numbers of
outcomes and time points; sensitivity checks that collapse to one effect
per study per domain (or use study-level aggregates) would test
robustness to outcome multiplicity. Fourth, moderator coding seems too
coarse or noisy for strong design conclusions. For example, ``value of
benefits'' is in nominal USD rather than PPP-adjusted or scaled by local
consumption/baseline income; ``scale of implementation'' and ``pilot vs
mid-level vs at-scale'' appear categorical with potential
misclassification; ``cash-plus'' is highly heterogeneous; and ``gender
targeting'' shows little variation. These choices attenuate power to
detect gradients (e.g., dose--response with transfer size) and likely
contribute to null moderator findings. Fifth, time dynamics are
insufficiently interrogated: most interventions last ˜12 months with
follow-up ˜14 months post-intervention; but SSN effects can accumulate
or fade. While a simple \textless12 vs =12 months split is reported,
more granular modeling (event time, longest vs earliest follow-up, or
pre-specified primary horizon) would clarify durability. Sixth, although
the authors state ``little evidence of publication bias,'' methods are
not detailed in the main text, and standard small-study or funnel
diagnostics are complicated by dependent effects; reporting study-level
funnel plots, RVE-compatible Egger tests, or p-curve/p-uniform would
strengthen this claim. Seventh, risk of bias/quality assessment is
referenced (Table S4) and used as a control, but it does not appear to
inform weights or stratified analyses; explicit RoB domains
(randomization, attrition, selective reporting, measurement) and
RoB-based sensitivity would increase credibility. Eighth, the
cost-benefit synthesis is informative but limited: only
\textasciitilde25\% of studies report BCR/IRR; methods and benefit
definitions vary; most do not include women's own economic gains or
agency measures, making comparisons difficult and likely underestimating
gender-specific value. A standardized re-analysis of costs/benefits
where microdata are available (or harmonized assumptions) would be more
decisive. Ninth, selection and generalizability: restricting to RCTs
improves internal validity but may bias towards pilots and specific
geographies (heavy in sub-Saharan Africa) and populations (few older
women; fragile/urban underrepresented). The meta-regression's ``pilot
lower than mid-level'' result is suggestive but could reflect targeting
differences; external validity remains a concern. Tenth, minor internal
inconsistencies in reported point estimates/precision should be
corrected (e.g., public works effect size appears as 0.127 elsewhere but
0.273 once; a decision-making CI is printed as 0.036--0.37). Finally,
while the study states replication files will be public and provides
supplementary materials, the data extraction instrument is not public,
limiting auditability of coding and decisions (e.g., outcome
inclusion/exclusion, direction coding, plus-component classification).
Despite these issues, the conclusions are measured, the main claims
(small positive effects on women's outcomes overall, with
UCTs/asset/social care robust; CCT and in-kind smaller) are consistent
with the assembled evidence, and the paper adds substantial value by
bringing together a broad experimental literature and by transparently
acknowledging limits where moderators cannot be pinned down. Priority
improvements: add prediction intervals; provide per-study-per-domain
sensitivity; articulate RVE weighting/\textless U+03C1\textgreater{} and
sensitivity; expand moderator measurement
(PPP-normalized/consumption-scaled transfer size; clearer ``plus''
typology); deepen time-horizon analysis; present publication-bias
diagnostics adapted to dependent effects; integrate RoB into
weighting/stratification; and release the extraction protocol/instrument
and a reproducible codebook linking each effect to its raw source and
coding decisions.

\end{tcolorbox}

\begin{tcolorbox}[enhanced jigsaw, colback=white, colbacktitle=quarto-callout-tip-color!10!white, bottomtitle=1mm, bottomrule=.15mm, opacitybacktitle=0.6, rightrule=.15mm, title=\textcolor{quarto-callout-tip-color}{\faLightbulb}\hspace{0.5em}{Model reasoning trace}, left=2mm, opacityback=0, colframe=quarto-callout-tip-color-frame, toptitle=1mm, arc=.35mm, titlerule=0mm, breakable, coltitle=black, leftrule=.75mm, toprule=.15mm]

\textbf{Formatting JSON assessment}

I need to create strict JSON with specific fields: assessment\_summary
and a metrics section that includes percentile scores for various
categories. There's a twist with midpoints, meaning I must produce point
estimates instead of ranges, which is a bit unusual. I should read the
content provided by the user and evaluate it based only on the PDF text.
I'll aim for a diagnostic summary of around 1000 words, focusing on
neutral and specific issues like identification threats and data
limitations.

\textbf{Evaluating provided content}

I need to focus solely on the information given in the text, which is an
extensive meta-analysis about the effects of SSN on women's economic
achievements and agency. The methods used include RVE meta-analysis and
conversions to Hedges' g, with rigorous assessments like multi-level RE
and checks for influential effects. They aimed to include studies in
multiple languages but found none in French or Spanish. Overall, they
examined 1,307 effect sizes from 93 studies across 45 LMICs, with a
pooled effect size of g = 0.107.

\textbf{Analyzing effects and findings}

The meta-analysis showed strong effects on savings, assets,
expenditures, labor force participation, work intensity, voice,
autonomy, and decision-making. However, they found null results for debt
and care work intensity, and insufficient power for care work
participation and leadership. Meta-regression indicated higher effect
sizes in the ECA region but was largely driven by one study. Lower
results were seen in LAC for basic specifications, especially with CCT
compared to UCT. High heterogeneity persists, and only 25\% of studies
reported cost-benefit analyses, primarily related to UCT.

\textbf{Identifying limitations and critiques}

The analysis highlights several limitations, such as the inability to
detect moderators, high heterogeneity, short follow-up periods, and
restrictions to only RCTs. They also mention issues with measurement
limitations regarding agency and potential publication bias. Moving from
aggregated data to individual women's benefits presents challenges due
to heterogeneity in components and limited gender targeting.

Now, I need to critique their methods, particularly around outcome
selection and coding. They transform outcomes to standardized mean
differences, but combining different measures may lead to
interpretational issues. A small scale of g 0.1 coupled with high
heterogeneity makes it hard to connect to policy significance, despite a
few cost-benefit analyses provided.

\textbf{Reviewing method suggestions}

I'm considering how to improve interpretability by converting to
relative risk or proportions, particularly for binary outcomes. They
mentioned using standardized mean differences (SMD) and Hedges' g, which
is typical, but it raises questions about mixing effect types. I noted
the use of correlated-effects RVE that requires specifying the
correlation parameter rho, which wasn't stated. They might have
defaulted to values like 0.8 or 0.7, affecting the standard error but
not the effect magnitude.

Regarding small-sample corrections, they followed Tipton's guidance and
used degrees-of-freedom greater than 4, which is positive. However, the
1,307 effects across 93 studies create a high ratio. The RVE is robust
to dependencies, but I'm curious if they weighted by precision or used
equal weights when clustering at the study level.

\textbf{Examining weight usage and methodology}

They mention using weights in their analysis but don't provide specific
details. Typically, equal weights are used in RVE, so it's unclear if
they applied the weighting method suggested by Hedges, Tipton, and
Johnson, which involves specific calculations. I should point out this
lack of clarity.

Additionally, they decided to mix stand-alone SSN with bundled
components to capture moderated effects. However, this approach
introduces definitional heterogeneity and potential misclassification
due to reliance on textual coding, which might limit their ability to
identify effect modifiers effectively.

\textbf{Providing suggestions for analysis}

I need to suggest better handling of studies with multiple arms and
follow-ups. They note the time since intervention and treat the `time
post-intervention at survey' at the effect level, but I wonder how they
managed multiple follow-up times from the same study. Clustering helps
with dependence, but mixing repeated measures could favor the largest
effect or create moral hazard. They might consider analyzing at a single
time point or using the longest-run measurement to minimize
multiplicity.

Regarding the unit of analysis, they include both individual and cluster
RCTs, but I'm unsure if they adjusted standardized means for clustering.
There could be a risk of ignoring intracluster correlation, affecting
replicability. I noticed they mentioned quality assurance but didn't
provide specifics on risk-of-bias assessment for randomization,
attrition, or measurement across included studies.

\textbf{Reviewing assessment details}

They mention a ``quality assessment score,'' which is a start. However,
when discussing publication bias, they acknowledge little evidence but
don't provide details about Egger's test. For RVE with dependence,
analyzing funnel plots and using Egger's test may be complicated, and
they haven't described any robust methods.

Regarding selection, only including RCTs may bias their sample towards
small pilot interventions, which often show smaller effects, raising
concerns about generalizability.

They conducted searches in other languages but found none, possibly
indicating selection or regional bias. I suggest they consider adding
forward and backward citations or translations.

Lastly, the data extraction form isn't publicly available, creating
barriers to replication. They mention plans to deposit replication
files, which is reassuring, but the lack of access to the data
extraction form could hinder understanding of their coding decisions.

\textbf{Highlighting replicability limitations}

There's a replicability limitation here. They perform subgroup analyses,
but with small sample sizes, such as public works with n=106 effects
across 9 publications, which yields significance at p\textless0.05.
However, the effect size of g = 0.127 may be overshadowed by aggregated
results showing variation in measurement.

The large effect size for the ECA region seems driven by one study with
a high benefit level, and they rightly caution against it.

Additionally, the definition of ``agency'' suffers from conceptual
heterogeneity, as they aggregated across measures. I wonder if they
ensured proper comparability in their analysis.

\textbf{Examining methodological concerns}

They mention ``woman specific'' indicators but don't clarify which ones,
potentially mixing household-level asset ownership with women's
ownership. It's good they exclude non-woman-specific indicators.

Regarding binary outcomes, using standardized mean differences (SMD)
might not be the best choice without clear details, which raises
replicability concerns.

Also, the high heterogeneity of I² at 90\%+ alongside a pooled effect of
g = 0.107 seems problematic. The effects are likely too broadly
distributed, and they didn't include prediction intervals, which could
be useful.

Multiplicity and analytic flexibility also raise concerns, as correlated
measurement may introduce selection biases. They coded pilot versus
mid-level versus large-scale implementations, but variable definitions
lack clarity. Presenting the pooled effect may misrepresent data, and
quantifying treatment effects would enhance understanding.

\textbf{Analyzing the interpretation of outcomes}

It's important to note that mixing multiple correlated outcomes per
study can introduce weight biases. While RVE attempts to manage this,
aggregating all outcomes together, especially across different domains,
makes it hard to interpret in terms of women's empowerment.

When they conclude that ``SSNs can economically and socially empower
women,'' it feels a bit over-interpreted as empowerment is a normative
term. They don't measure intangible aspects or address high
heterogeneity, which should moderate their conclusions.

Regarding ``cash plus'' components, they suggest these are rarely more
effective than cash alone, but they didn't observe this in their sample
and should be cautious about heterogeneity.

For cost-benefit analysis, they summarize normative findings without
standardized discount rates, which could be misleading. They also
mention risk-of-bias indicators but don't clarify what qualifies as high
quality.

\textbf{Considering quality and data management issues}

It's a bit puzzling that they mention ``if medium quality''; it sounds
like they may only have low vs.~medium categories, but I think this
needs more detail. Not weighting the meta-analysis by risk-of-bias could
be an issue, though it might not cause major shifts.

Regarding potential double-counting, many publications have multiple
arms, and I'm wondering if they've adequately avoided counting the same
control group multiple times. RVE partially addresses this, but mixing
effect sizes with shared controls could lead to double counting, even
though they cluster by study, which seems acceptable.

For data management, they aim to provide robustness against `Cook's D'
influences, but the threshold remains unspecified, and visual
inspections might introduce subjectivity. They mention that replicates
are unaffected, which is positive.

Lastly, when it comes to domain classification, their definition of
``majority'' in aggregated indicators needs clarification, and I'm
curious if they've mistakenly double counted aggregated indicators and
their components.

\textbf{Addressing data quality and outcome concerns}

There's a risk of double counting here because it seems they might treat
separate effect categories as unique. However, including both aggregated
indices and subcomponents from the same sample could overweight specific
studies. I wonder if they avoided this mixing, but they didn't say
explicitly.

Also, they're not analyzing potential harm, such as domestic violence or
intra-household tensions, which feels like a significant oversight since
they're focusing on positive outcomes like economic achievements and
agency.

When it comes to nonlinearity with benefit values, they used terciles of
`total value of economic benefits' without adjusting for purchasing
power parity (PPP) or local prices, which might mismeasure `value' due
to variations across countries.

For assets and savings, mixing `any savings' with total amounts seems
off, as SMD calculations may not be standardized across types. Finally,
regarding expenditures, did they consider whether these were
specifically women's own spending versus household expenditures?

\textbf{Clarifying outcome measurement concerns}

They say outcomes are ``woman-specific only,'' but I'm not entirely sure
about how they measure expenditures. It seems like expenditures might be
assessed at a household level, yet they state that they only include
``woman-specific'' outcomes. I wonder if they truly focused on women's
own expenditures or if they incorporated aspects like ``consumption per
adult female.'' That approach seems a bit uncertain because it raises
questions about whether they're accurately capturing women's unique
impacts.

\textbf{Questioning measurement and validity}

I'm considering the risk that they might have included household-level
data, even though they claimed to focus on ``only woman-specific''
outcomes, so it's likely they limited their analysis to women's
consumption. I need to check for replicability.

Then there are agency measurement issues; they've noted limitations in
reliability which is worth considering. Also, they used a 12-month
threshold to categorize duration and follow-up, but I'm wondering if
testing other thresholds might yield different insights since they
didn't show those details.

Regarding external validity, most studies are based in sub-Saharan
Africa, which introduces regional variations. They often look at youth
versus adult demographics, noting that there aren't many older women
represented nor a balance between urban and rural settings---this could
skew their findings.

Finally, the classification of sample age groups seems approximate since
they use share-of-sample averages, which might lead to
misclassification. Their comparisons between cash conditional transfers
(CCT) and unconditional transfers (UCT) could also be confounded by
regional differences. They tried to control for this, but interplay is
still an issue. Additionally, what do they mean by ``social care
services''? Does this include childcare, counseling, and case
management?

\textbf{Identifying discrepancies in effect sizes}

They claim the effects are robust, but the numbers at the effect level
seem small, and there's a lot of heterogeneity.

For ``public works programs,'' there's a confusing discrepancy: they
state an effect size of g = 0.273, but the figure shows 0.127. In the
text, they initially wrote 0.273 beside a significance of p \textless{}
0.05. However, the figure indicates 0.127 with significance. So, it
looks like I found a typographical error where they initially mentioned
0.127 but later typed 0.273 instead. It's important to clarify this!

\textbf{Noting internal inconsistencies and errors}

I'm spotting a few internal inconsistencies that should be highlighted,
perhaps as minor misprints. For instance, the ``decision-making effect''
is reported as 0.087 with a confidence interval (CI) of 0.036-0.37, but
that CI looks quite wide and maybe incorrectly transcribed.

There are also some formatting issues with the confidence intervals.
Additionally, many of their meta-regressions might exceed the threshold
of 4, suggesting unstable results, so they should mention small sample
corrections. Finally, they reference their quality assessment without
typical details, which is worth calling out.

\textbf{Addressing quality and bias concerns}

I'm looking at the Cochrane risk-of-bias tool, and I think it's worth
mentioning that the quality variable seems questionable. It feels odd to
limit categories to low and medium without including high quality.

They've coded various settings and components well, focusing on RCTs and
addressing grey literature. However, they claim a lack of evidence for
publication bias without a thorough analysis. There are also challenges,
like the potential overweighting of results from large studies, and the
use of I² with RVE, which isn't standard. Testing sensitivity for
effects could improve the reliability of findings.

\textbf{Evaluating quality and claims}

I'm considering the quality of the paper, which seems strong due to its
thoroughness and large sample size. It's a rigorous meta-analysis that
uses RVE methods and is pre-registered, which is great for
replicability. However, while they state that data and analysis
replication files will be publicly available if accepted, the lack of
publicly available data extraction forms detracts from their open
science commitment. I'd rate their open science as high, around 78 or
80.

Their claims about SSNs empowering women are made carefully, which is
positive, but I wonder if there's a risk of over-claiming without
sufficient evidence of moderate effects.

\textbf{Assessing claims and methods}

They propose eliminating conditionalities, but they also acknowledge
that it's difficult to fully separate conditions from design features,
which is a responsible approach. Their claims about social safety nets
(SSNs) empowering women seem credible, so I'd rate claims evidence
around 84.

Their use of RVE methods is robust but does have some issues, like
mixing standardized mean differences (SMD) across outcomes; I'd give
methods an 82.

For advancing knowledge, this paper pools data across modalities, which
is significant, especially in 2024 with this large dataset, so I'd rate
that 85.

The writing is clear, but there are minor inconsistencies, like a typo
related to numeric values. For open science, I'd rate it a 77, while the
global relevance is high at 92 due to the widespread use of social
safety nets. Overall, I'd provide a weighted rating of 86.

\textbf{Setting credible intervals}

I'm thinking about how to set the credible intervals, as the
instructions specify that the lower bound, midpoint, and upper bound
must all be equal. Traditionally, a 90\% credible interval is expected,
but here, the focus is on setting all bounds to equal the midpoints.
This approach aligns with the analysis of meta-analyses related to
gender and social safety nets (SSNs) in low-to-middle income countries
(LMICs). Considering the high quality of this analysis, I might rate it
around 89 overall but weigh disclaimers to set it closer to 86. As for
``tier\_should'' and ``tier\_will,'' I'll categorize them on a scale
from 0 to 5.

\textbf{Assessing article tier}

I'm considering the type of article, which incorporates new findings and
could potentially fit into a high-tier general interest journal like
``Science.'' However, I need to stick to the content without relying on
any external knowledge. Personally, I'd assign it a score of 4.2,
suggesting it's a marginal A-journal or a top field journal like ``World
Development'' or ``Journal of Development Economics.'' I'll assign
scores of 4.3 and 4.1 for ``should'' and ``will,'' respectively, while
ensuring that the lower and upper confidence interval scores match the
assigned score according to the instructions. Given the issues of
cost-benefit and heterogeneity, I believe a score of 4.2 is reasonable
for ``should'' and 4.0 for ``will.''

\textbf{Preparing summary outline}

I'm getting ready to produce a 1000-word summary that will highlight
major issues and benefits of the analysis. I want to include
method-level details, such as the scope and conceptualization, high
heterogeneity, mixing of outcomes, and minimal details on weighting and
correlation parameters in the RVE. I also need to mention how they
calculated SMD and converted binary data, alongside a note about the
lack of replicable aggregators.

I'll point out measurement issues related to transfer value not being
PPP, and the absence of predicted effect sizes like ``marinade.'' I'll
ensure the summary circumvents external knowledge and avoids referencing
authors or institutions. Paraphrasing is essential to avoid copying
text.

I'll also note any typos or inconsistencies I've noticed, like the
discrepancy between `g 0.273 vs 0.127' and the `decision-making
0.036-0.37.' Lastly, I want to include the replicability note about the
data extraction forms not being public.

\textbf{Starting summary crafting}

I need to mention the figure misalignment in the summary. Let's begin
with: ``The paper offers a registered systematic review and
meta-analysis of experimental studies on SSNs in LMICs, focusing on
women's economic outcomes and agency. The search identifies 93 studies
and 1307 effect sizes, calculating standardized mean differences using
Hedges' g. They apply robust variance estimation (RVE) to handle
within-study dependence, run multilevel random-effects for sensitivity,
and present orchard plots and meta-regressions with study, intervention,
and effect-level covariates. Results show a small positive pooled effect
of g 0.107, notably stronger for UCT, asset transfers, and social care,
while public works yield positive but less precise results and CCT shows
a smaller effect with in-kind not statistically significant.''

Next, I'll list the issues, starting with the extremely high
heterogeneity: I2 \textgreater80-95\% and t\^{}2 \textgreater0.02.

\textbf{Outlining key issues}

The paper presents an average effect across diverse interventions,
outcomes, and populations, but lacks prediction intervals and
distribution of true effects, which might make the mean less informative
for policy.

Next, outcome harmonization is a concern; terms like ``economic
achievements'' and ``agency'' are broad, and their subconstructs are
measured heterogeneously. The use of standardized mean differences
combines continuous and binary data, limiting clear definitions.

Multiplicity of effect sizes per study is another issue; while RVE
addresses correlation, it doesn't consider study-level quality. If some
studies contribute many outcomes, they could dominate the results. I'm
curious if the authors considered using one effect per study per domain
as a sensitivity check, but that's not reported.

Additionally, the moderator coding is problematic since ``transfer
value'' is measured in total USD instead of PPP, and terms like
``scale'' lack clarity. The categorization of ``plus components'' is too
coarse, and ``gender targeting'' shows limited variation, which hampers
design insights.

\textbf{Highlighting additional concerns}

There are several issues to consider with the study. First, selective
reporting might be a problem, as many included RCTs may not measure or
report gender-disaggregated outcomes. Without systematic extraction from
registry protocols, we can't confirm outcome reporting bias, which is
mentioned but not fully addressed.

Next, regarding publication bias, they state there's ``little
evidence,'' but they don't describe their method, which makes it hard to
trust their conclusions.

Cost-benefit integration is also lacking, as the methods aren't
standardized and aren't used to inform effect sizes. Many benefit-cost
ratios exclude considerations of women's incomes, which compromises
their analysis.

Furthermore, sample restrictions are apparent as the study only includes
RCTs weighted to Sub-Saharan Africa, lacking older women and showing
insufficient representation from fragile contexts.

Many effects are measured within a 12-14 month window, which doesn't
account for long-term results.

Finally, while they promise replication, the extraction form isn't
publicly available, raising doubts about replicability.

\textbf{Evaluating key metrics}

There are more factors to consider. The risk-of-bias assessment scoring
is unknown. It seems like they indicate low or medium quality without
sufficiently addressing high risk, only using it as a control.

Next, I noticed some inconsistencies, like the public works effect size
being reported as 0.127 versus 0.273, and decision-making confidence
intervals printed as 0.036--0.37.

Now, I need to produce the metrics, adhering to strict JSON formatting.
For global relevance, I'll set it high at 92, while open science will be
78.

For the overall metrics: midpoint at 86, claims and evidence at 84,
methods at 82, advancing knowledge maybe at 86, logic and communication
at 80 due to inconsistencies. This gives me a ``tier should'' of around
4.4 and ``tier will'' at 4.1.

\textbf{Clarifying metrics and summary structure}

I need to unify the scoring from 0-5. I think I'll go with 4.2 for
``should'' and 4.0 for ``will''. As for ``journal A,'' while it's a top
general field, the rating could be around 4.3, but I'm leaning towards
4.4 for ``should'' and 4.2 for ``will''.

I'll make sure each metric, like ``overall'' and ``claims\_evidence,''
is structured with midpoint, lower\_bound, and upper\_bound.

They want a summary of about 1000 words, aiming for 850-1100 words. I'll
ensure it's a compact paragraph, focusing on key issues like data
limitations and internal inconsistencies, and avoid heavy markup and
bullet points. I want to present a clear and concise analysis!

\textbf{Finalizing important issues summary}

I'm focusing on crafting a summary that highlights the most important
issues, making sure to do this elegantly and without mentioning any
authors. I need to avoid mentioning any lack of samples as well. It's
essential to finalize the document without oversight, such as ensuring
the images represent tables and figures correctly.

I'll aim for a word count of about 900-1100 words, targeting roughly
1000 words in total. Now, I'm ready to complete this step!

\end{tcolorbox}

ewpage

\section{Williams et al.~2024}\label{williams-et-al.-2024}

\begin{tcolorbox}[enhanced jigsaw, colback=white, colbacktitle=quarto-callout-note-color!10!white, bottomtitle=1mm, bottomrule=.15mm, opacitybacktitle=0.6, rightrule=.15mm, title=\textcolor{quarto-callout-note-color}{\faInfo}\hspace{0.5em}{Model assessment summary}, left=2mm, opacityback=0, colframe=quarto-callout-note-color-frame, toptitle=1mm, arc=.35mm, titlerule=0mm, breakable, coltitle=black, leftrule=.75mm, toprule=.15mm]

This paper delivers a high-resolution, pantropical map of where
deforested lands have biophysical potential to undergo natural forest
regeneration, and it aggregates this to country and biome summaries and
an overlay-based carbon removal potential. The approach is transparent
in intent---predicting potential, not feasibility---but several
methodological choices and data limitations affect inference strength
and precision. The dependent variable (presence/absence of natural
regeneration during 2000--2016) is derived from a prior global
remote-sensing classification that conservatively emphasized omission
errors in humid regions and required a minimum patch size; area-based
producer accuracy for natural regrowth in the humid biome was low while
user accuracy for detected patches was high. This means the positive
training class is a non-random subset of true regrowth, with systematic
under-detection likely concentrated in particular forest types,
geographies, and patch sizes. The predictive model then inherits these
detection and sampling biases, potentially distorting learned
relationships and shifting mapped potential away from contexts where
regrowth was common but poorly detected. Model fitting uses random
forests on millions of stratified points, with a sensible
variable-selection procedure; predictions are ultimately based on
biophysical predictors alone because a combined socioecological model
yielded very similar accuracy. Omitting socioeconomic predictors makes
the maps more temporally stable and widely available, but it also
reframes outputs as ``biophysical potential'' and risks overpredicting
potential in places where human pressure or governance constraints are
binding. Validation emphasizes overall accuracy around 88\% using a
balanced validation set; however, accuracy declines at moderate
distances from training locations, indicating residual spatial
autocorrelation and some optimism in headline accuracy. The use of
accuracy rather than threshold-free metrics, and the balancing of
classes (prevalence set to 50\%) during training and validation, further
complicates interpretation of both accuracy and the calibration of
predicted probabilities. The paper interprets the continuous model
output as a probability and converts this to ``expected area'' by
multiplying by pixel area and summing, yielding 215 Mha. Yet the
calibration of random-forest scores to true probabilities is not
established; without calibration (e.g., isotonic or Platt scaling) under
the real-world prevalence, the expected-area identity may be biased. The
extraordinarily narrow ``confidence intervals'' reported for area (±0.22
Mha at global scale) are not credible statistical uncertainty; they
reflect computational or rounding artifacts, not propagation of
uncertainty from training data errors, cross-validated prediction error,
input-layer uncertainty, or probability calibration. A threshold
(\textgreater0.5) sensitivity shows a substantially larger area,
highlighting dependence on interpretation choices. Many predictors are
coarse (250 m--1 km) while predictions are at 30 m; this scale mismatch
introduces blockiness and may overstate local precision. The domain of
``available for restoration'' is defined liberally by excluding only
forests, water, urban, bare, and forestry areas; while appropriate for
modeling, its presentation alongside country totals risks being misread
as areas that are socially or legally available for reforestation.
Temporal assumptions also matter: the model projects to 2015/2030 by
assuming that 2000--2016 relationships hold and by updating some
forest-context layers with 2018 data; it does not incorporate climate
change trajectories, shifting fire regimes, or evolving socioeconomics
that strongly affect regeneration and permanence. Carbon estimates are
calculated by downscaling a 1 km accumulation dataset to 30 m,
multiplying by modeled potential, and summing; the stated range lacks a
clearly defined uncertainty model and appears not to propagate
classification, calibration, or permanence risks. The paper is
commendably explicit that realized sequestration will be lower due to
leakage, reclearance, and policy constraints, yet some statements
comparing potential annual removals to current pantropical losses could
be interpreted too strongly given these omitted risks and uncertainties.
Spatial performance is uneven, with lower accuracy in parts of Southeast
Asia; country and biome summaries lack uncertainty decomposition, which
would be useful for decision makers. On replicability, large gridded
outputs are publicly archived and inputs are open, but code is available
only upon request; combined with the volume of preprocessing, this
raises barriers to full reproduction, code review, and extension.
Strengths include a clearly articulated objective focused on actionable
planning, a very large training sample, careful variable selection,
explicit treatment of autocorrelation effects (even if not using full
spatial block cross-validation), and extensive caveats about use and
governance. The resulting layers will be highly valuable as a first-pass
screen for assisted natural regeneration opportunities and for national
restoration dialogues. The main improvements needed relate to
uncertainty characterization (probability calibration, spatially
explicit error propagation, and country-level error budgets), spatial
cross-validation that reduces spatial leakage, clearer separation of
statistical precision from conceptual ranges, dynamic scenario analysis
including plausible climate and human-pressure changes, and full open
release of code. Overall, this is an influential, practically useful
mapping contribution with limitations that mainly concern probabilistic
interpretation, uncertainty, and temporal and socioecological external
validity.

\end{tcolorbox}

\begin{tcolorbox}[enhanced jigsaw, colback=white, colbacktitle=quarto-callout-tip-color!10!white, bottomtitle=1mm, bottomrule=.15mm, opacitybacktitle=0.6, rightrule=.15mm, title=\textcolor{quarto-callout-tip-color}{\faLightbulb}\hspace{0.5em}{Model reasoning trace}, left=2mm, opacityback=0, colframe=quarto-callout-tip-color-frame, toptitle=1mm, arc=.35mm, titlerule=0mm, breakable, coltitle=black, leftrule=.75mm, toprule=.15mm]

\textbf{Analyzing model limitations}

I see another limitation in the model's assumptions. They assumed
stationarity from 2000-2016 when predicting for 2015 to 2030. They
included present conditions like 2018 tree cover but didn't factor in
future climate or socio-economic changes. By excluding socio-economic
predictors, they reduced potential model improvements slightly, but this
could bias predictions in areas where socio-economic factors play a
significant role, potentially leading to unrealistic outcomes in
``biophysical potential.'' They clarify it's about modeling potential,
not feasibility.

\textbf{Examining dataset limitations}

I notice another issue with the primary regrowth dataset used, which
came from Global Forest Watch to map regrowth over 5 m. In humid
tropics, the producer's accuracy is low at just 18.7\% because of
omission errors, although user accuracy is high at 85.1\%. The detection
biases may skew the training sample and affect model calibration. Their
conservative approach to mapping regrowth likely favors more accessible
and larger areas, and many small regrowth patches are excluded. It's
unclear if the training sample is representative across different biomes
without proper stratification.

\textbf{Clarifying domain definitions}

I need to define the domain used in the study. They included all parts
of forest biomes within ±25° but excluded water, bare land, urban areas,
and so on, while purposely opting for a liberal domain. They used 2018
tree cover to exclude forests but may misclassify certain savannas or
drylands as part of forest biomes. While they mention not including
savannas, their definition could still capture mosaic-savanna zones.
Additionally, there's a scale mismatch since predictions at a finer 30 m
resolution are computed alongside coarser predictors, which could impact
accuracy.

\textbf{Assessing the weighted area approach}

I see another issue with the weighted area approach, where the
probability of regrowth times pixel area is calculated. However, the
random forest classification is trained on presence/absence from
2000-2016, and the output probabilities aren't properly calibrated.
Calibration techniques like Platt scaling or isotonic regression are
necessary to interpret results as probabilities. They also didn't
present AUC, only accuracy, which may mislead since balanced sampling
artificially sets prevalence to 0.5. Without proper probability
calibration, the estimate of expected area could be questionable and not
reflect actual conditions.

\textbf{Reviewing definitions and calibration}

They used specific definitions, claiming that ``the continuous
potential'' can be seen as the probability of natural regeneration per
pixel, with the weighted-area value representing the expected area for
regeneration. However, they didn't perform any calibration on this.
Their accuracy and out-of-bag accuracy are reported as 0.878, but they
failed to discuss probability calibration, which is important. Regarding
carbon sequestration potentials, they calculated totals and ranges from
previous datasets, but it's unclear where the range comes
from---possibly from Monte Carlo methods or carbon dataset quartile
variation, yet they didn't show how they computed uncertainty.

\textbf{Identifying dataset uncertainties}

The dataset presents uncertainties by grid cell, but their method
multiplies 30-year totals without mentioning how uncertainty is handled.
The ``range'' they provide likely comes from minimum and maximum values
across these uncertainties, which is unclear. They need to clarify how
uncertainty propagates and address independence and omitted below-ground
carbon, since they reference adding 22--28\% from literature.
Additionally, they didn't consider factors like degradation baseline,
leakage, or permanence. While they frame the ``biophysical potential,''
the abstract could mislead policy implications with its stated
above-ground carbon sequestration potential of 23.4 Gt C over 30 years.

\textbf{Checking carbon sequestration claims}

I noticed they mentioned that ``new carbon sinks could increase global
sequestration by 14.3\% per year'' but also stated ``mitigating current
pantropical forest carbon losses by 90.5\% per year,'' which might be
inconsistent. They claim assisted natural regeneration can sequester
significant carbon but didn't show how they calculated this. It seems to
derive from 23.4 Gt over 30 years, equating to 0.78 Gt C per year. With
a net carbon loss of about 0.86 Gt C per year, I need to verify Baccini
2017's values, which reported a net carbon source of 1.1 Gt C per year.
My memory on this is a bit fuzzy!

\textbf{Evaluating carbon loss mitigation claims}

They state that ``mitigating current pantropical forest carbon losses by
approximately 90.5\% per year,'' referencing Baccini, raises some
questions. If regeneration can offset about 0.8 Gt per year, that
suggests total losses of around 0.9 Gt per year. While this might be
plausible, there are logical leaps involved since their potential
benefits might not actually be realized. It's a bit tricky to interpret,
yet they present it confidently as potential. I need to consider these
assumptions carefully!

\textbf{Analyzing dataset issues}

They wisely exclude plantation regrowth from their analysis, but it
seems their training dataset might still contain misclassified
plantations. They reported a previous algorithm with 90.6\% accuracy;
however, an area-based producer accuracy of just 18.7\% in humid tropics
suggests significant under-detection. This non-random classification
error could undermine generalizability. While they argue for the
robustness of the detected patches, the validation accuracy of 87.9\%
relies on potentially biased training dataset presence points. Field
validation is unlikely, and poor accuracy in Southeastern Asia raises
another red flag.

\textbf{Evaluating modeling assumptions}

The modeling approach assumes that predicted `potential' is valid from
2015 to 2030 without considering changes, which is questionable since
the climate is constantly shifting. Additionally, the ``distance to
forest'' will likely change as regrowth occurs, yet they didn't model
this dynamic feedback. They mentioned underestimating impacts like
positive feedback and the exclusion of savanna/dryland. Climate change
could also hinder regrowth, leading to ambiguity. They present a figure
showing ``available for restoration'' area by country, but they don't
clarify how this was calculated, leaving some mystery around the
methodology used.

\textbf{Analyzing definitions and reproducibility}

They define the study's domain as all tropical forest biomes, excluding
areas like water, urban, and existing forests, while including factors
like `distance to water' as covariates. It seems that ``available for
restoration'' refers to the total area of this domain, but their
disclaimers indicate some harvest/regrowth classifications might be
included. Although they mention using a forestry dataset, they didn't
clearly define ``available for restoration.'' Additionally, requesting
code via email isn't ideal; it should be publicly accessible. While they
use accessible datasets, the lack of code hampers reproducibility, and
the confidence intervals aren't credible.

\textbf{Evaluating accuracy and methodology}

They mention the model overcame potential overfitting with an out-of-bag
(OOB) accuracy of 87.8\%. They used a balanced sample and omitted
socioeconomic features due to similar accuracy, but I wonder if the
limited role of socioeconomic data in predicting presence versus
absence, as indicated by remote sensing, could affect results. They also
disclaim that the maps are only one input, which is good. Since this is
published in Nature, it seems high quality, but I believe we should look
beyond that. We need to propose credible metrics.

\textbf{Ranking evaluation criteria}

Based on the content, I would rank high---maybe around the 85th
percentile overall.

For claims and evidence, they articulate strong claims and provide
robust mapping and a predictive model. However, there are issues like
reliance on one training dataset with major omissions and unrealistic
confidence intervals. I'd place this at about 75.

For methods, they've implemented random forest modeling and accounted
for autocorrelation, yet there's heavy reliance on coarse data. This
leads me to perhaps a 70.

Regarding advancing knowledge, their maps on natural regeneration
potential are valuable, globally relevant for restoration planning, so
I'd mark this as a 90.

On logic and communication, while the writing is clear and disclaimers
are present, there are leaps in concluding about carbon losses and some
misuses of ``CI.'' I'd score this around 82.

For open science, while the data is available on Zenodo and the input
data is open, the code is only available on request, which limits
replicability. I'd give a score of 65.

Finally, in terms of global relevance, this work ties in well with
global priorities and quantifies information by country, so I'd rate
this a solid 92.

\textbf{Assigning tier ratings}

I need to assign ``tier\_should'' and ``tier\_will'' ratings from 0 to
5. Based on quality, I think ``should'' deserves a high score, perhaps
around 4.5 or 4.7, due to some identified issues. For ``will,'' since
it's already published in Nature, I'd predict around 4.6 to 4.8.
However, they want continuous numbers, so I'll set ``should'' at 4.4 and
``will'' at 4.7.

If I consider possible underestimation due to method-related issues,
perhaps ``should'' could be 4.2 or 4.4, while ``will'' could be 4.8
because of strong interest. Still, the dataset not being open-coded is a
consideration, but it's published in Nature, so I want to remain fair.

Now, I need to craft a summary, aiming for around 1000 words. I'll start
by summarizing the approach and identifying major issues.

\textbf{Highlighting dataset issues}

I'm noticing that the training dataset has biases, especially regarding
natural regrowth mapping, which shows significant omissions in humid
biomes. This can lead to a non-random sample of presences, and the model
might learn from incomplete detections, distorting variable
relationships and potentially mispredicting areas.

Next, regarding probability calibration and the ``expected area''
method, they treat random forest outputs as probabilities, but without
proper calibration and baseline prevalence, the resulting estimates of
215 million hectares could be inaccurate. They present highly precise
confidence intervals that don't seem credible because they didn't
account for uncertainty across inputs.

Lastly, concerning spatial autocorrelation, while they tried to evaluate
it, accuracy tends to decline with distance, so their optimism might not
be justified. They didn't apply block cross-validation and reported
overall accuracy, avoiding metrics like AUC. The balanced classification
gives an unrealistic sense of accuracy, and there are no confusion
matrices or biome-specific metrics presented beyond overall accuracy.

\textbf{Addressing scale and domain issues}

I see that there's a scale mismatch with the 30 m predictions derived
from 300 m to 1 km predictors, which could cause smoothing and
artifacts. They've acknowledged this with visual evidence in Fig. 1,
suggesting it could impact local planning as they consider these results
to be preliminary.

The domain definition appears overly broad, including areas where
natural regeneration may be undesirable---like non-forest ecosystems,
productive farmland, or communities---which could potentially mislead
their ``available for restoration'' metrics.

Lastly, they assume conditions from 2000 to 2016 without modeling
climate change or socioeconomic factors, which could really undermine
their 2030 predictions. Also, the carbon accumulation overlay from the
Cook-Patton dataset has a scale mismatch issue, with 1 km data being
downscaled to 30 m without considering edges and heterogeneity.

\textbf{Evaluating carbon accounting and code accessibility}

They don't initially account for below-ground carbon, indicating an
additional 22--28\% should be included, and they don't consider the risk
of reclearance. Their discussions on leakage and additionality seem
inadequate since their yearly rates are used to support claims of
significant offsetting of losses, which likely exaggerates realized
mitigation.

While they exclude plantations, algorithmic misclassification might
still exist, and they haven't incorporated a classification probability
step.

Regarding code accessibility, it's not publicly available, making
replication difficult. Although they provide predicted maps on Zenodo,
the training and modeling code needing to be requested further hampers
replicability. Additionally, using 2018 tree cover for distance
predictions while training on 2000 cover could lead to variable shifts,
potentially degrading the predictions since they haven't cross-validated
changes.

\textbf{Observing assumptions and suggestions for improvement}

The weighted area assumption sums across all pixels to determine
``expected area,'' but it overlooks spatial dependence and positive
feedback, which are mentioned but not integrated. The country-level
summarization is good, although it uses administrative boundaries
without addressing ecoregion differences, and there's a lack of error
budget for these country-level sums.

They note that predictions may degrade in Southeastern Asia, but don't
confirm this for later years. For improvements, I should suggest block
cross-validation, probability calibration via isotonic regression, and
incorporating socioeconomic variables.

Overall, I thought the score should be around 85, but I'm reconsidering
an 86.

\textbf{Finalizing ranking and summary considerations}

I'm setting the overall score to 86, considering it's intended to
reflect percentiles within {[}0, 100{]}. For research in remote sensing
global mapping, I'd lean toward an 88, but due to certain issues, I
might adjust this to 82. However, advancing knowledge could be rated
higher at 90, leading me to average around 86, so choosing 87 seems
fair.

I also need to fill out the JSON carefully, ensuring that the assessment
summary is a concise paragraph of about 800-1100 words and omitting any
author names or external knowledge outside the provided PDF. The
confidence intervals will match the midpoint of 86. For claims and
evidence, I'm thinking a score of 79?

\textbf{Setting scores and producing JSON details}

I've decided to set the score of 78 for overall assessment, with
specifics for other categories: 73 for methods, 92 for advancing
knowledge, 84 for logic communication, 62 for open science, and 94 for
global relevance. For tier should, I'll use a score of 4.4, and for tier
will, I'll set it at 4.8, all while keeping consistency in mind with
overall scores at 86, advancing at 92, and global relevance at 94.

The JSON needs to include an assessment summary of around 1,000 words
and metrics for each score calculated with midpoint, lower, and upper
bounds. I'll also avoid naming authors or their institutions, but I can
reference previous analyses without using specific names or citation
numbers.

\textbf{Crafting the summary points}

I'm going to paraphrase the training dataset description, ensuring I
accurately reference the analysis of natural regrowth from 2000 to 2016
without including citation numbers. I should say, ``The study builds on
an existing pantropical remote sensing analysis that mapped natural
regrowth and distinguished it from plantations.''

Next, I want to finalize the summary by outlining key points. For data,
I'll mention potential mapping across tropical forests at a 30 m
resolution, calculating weighted areas for 215 Mha, and noting
variations across ecoregions.

For methods, I'll detail the balanced random forest approach, variable
selection, the training dataset derived from regrowth mapping, and the
overall accuracy. I'll also highlight potential issues like omissions in
the training dataset and domain definition relative to thresholds.

\textbf{Assessing model limitations}

I need to go over some aspects of the model's predictive capabilities.
First, I see that there's a lack of probability calibration in the
balanced training; the predicted probabilities aren't calibrated, which
might lead to bias in the weighted area approach and unrealistic
confidence intervals.

Next, there's the issue of carbon overlay and uncertainty propagation;
downscaling seems to be done naively. The coarser covariate resolution
could degrade the 30 m predictions, causing blockiness. Socio-economic
variables are omitted in terms of feasibility versus potential, and
there's a failure to integrate dynamic processes like climate change
impacts.

I should also note that accuracy in Southeastern Asia is poor, and the
code's availability is limited, impacting replicability due to the heavy
dataset.

Lastly, I'll mention that the additional risk of reclearance isn't
accounted for in the carbon measurements. I need to finalize this while
including suggestions and discussing uncertainties qualitatively as we
maintain strict JSON formatting.

\end{tcolorbox}

ewpage



\end{document}
