% Options for packages loaded elsewhere
\PassOptionsToPackage{unicode}{hyperref}
\PassOptionsToPackage{hyphens}{url}
\PassOptionsToPackage{dvipsnames,svgnames,x11names}{xcolor}
%
\documentclass[
  11pt,
  letterpaper,
  DIV=11,
  numbers=noendperiod]{scrreprt}

\usepackage{amsmath,amssymb}
\usepackage{iftex}
\ifPDFTeX
  \usepackage[T1]{fontenc}
  \usepackage[utf8]{inputenc}
  \usepackage{textcomp} % provide euro and other symbols
\else % if luatex or xetex
  \usepackage{unicode-math}
  \defaultfontfeatures{Scale=MatchLowercase}
  \defaultfontfeatures[\rmfamily]{Ligatures=TeX,Scale=1}
\fi
\usepackage{lmodern}
\ifPDFTeX\else  
    % xetex/luatex font selection
\fi
% Use upquote if available, for straight quotes in verbatim environments
\IfFileExists{upquote.sty}{\usepackage{upquote}}{}
\IfFileExists{microtype.sty}{% use microtype if available
  \usepackage[]{microtype}
  \UseMicrotypeSet[protrusion]{basicmath} % disable protrusion for tt fonts
}{}
\makeatletter
\@ifundefined{KOMAClassName}{% if non-KOMA class
  \IfFileExists{parskip.sty}{%
    \usepackage{parskip}
  }{% else
    \setlength{\parindent}{0pt}
    \setlength{\parskip}{6pt plus 2pt minus 1pt}}
}{% if KOMA class
  \KOMAoptions{parskip=half}}
\makeatother
\usepackage{xcolor}
\usepackage[margin=1in]{geometry}
\setlength{\emergencystretch}{3em} % prevent overfull lines
\setcounter{secnumdepth}{5}
% Make \paragraph and \subparagraph free-standing
\ifx\paragraph\undefined\else
  \let\oldparagraph\paragraph
  \renewcommand{\paragraph}[1]{\oldparagraph{#1}\mbox{}}
\fi
\ifx\subparagraph\undefined\else
  \let\oldsubparagraph\subparagraph
  \renewcommand{\subparagraph}[1]{\oldsubparagraph{#1}\mbox{}}
\fi


\providecommand{\tightlist}{%
  \setlength{\itemsep}{0pt}\setlength{\parskip}{0pt}}\usepackage{longtable,booktabs,array}
\usepackage{calc} % for calculating minipage widths
% Correct order of tables after \paragraph or \subparagraph
\usepackage{etoolbox}
\makeatletter
\patchcmd\longtable{\par}{\if@noskipsec\mbox{}\fi\par}{}{}
\makeatother
% Allow footnotes in longtable head/foot
\IfFileExists{footnotehyper.sty}{\usepackage{footnotehyper}}{\usepackage{footnote}}
\makesavenoteenv{longtable}
\usepackage{graphicx}
\makeatletter
\def\maxwidth{\ifdim\Gin@nat@width>\linewidth\linewidth\else\Gin@nat@width\fi}
\def\maxheight{\ifdim\Gin@nat@height>\textheight\textheight\else\Gin@nat@height\fi}
\makeatother
% Scale images if necessary, so that they will not overflow the page
% margins by default, and it is still possible to overwrite the defaults
% using explicit options in \includegraphics[width, height, ...]{}
\setkeys{Gin}{width=\maxwidth,height=\maxheight,keepaspectratio}
% Set default figure placement to htbp
\makeatletter
\def\fps@figure{htbp}
\makeatother
% definitions for citeproc citations
\NewDocumentCommand\citeproctext{}{}
\NewDocumentCommand\citeproc{mm}{%
  \begingroup\def\citeproctext{#2}\cite{#1}\endgroup}
\makeatletter
 % allow citations to break across lines
 \let\@cite@ofmt\@firstofone
 % avoid brackets around text for \cite:
 \def\@biblabel#1{}
 \def\@cite#1#2{{#1\if@tempswa , #2\fi}}
\makeatother
\newlength{\cslhangindent}
\setlength{\cslhangindent}{1.5em}
\newlength{\csllabelwidth}
\setlength{\csllabelwidth}{3em}
\newenvironment{CSLReferences}[2] % #1 hanging-indent, #2 entry-spacing
 {\begin{list}{}{%
  \setlength{\itemindent}{0pt}
  \setlength{\leftmargin}{0pt}
  \setlength{\parsep}{0pt}
  % turn on hanging indent if param 1 is 1
  \ifodd #1
   \setlength{\leftmargin}{\cslhangindent}
   \setlength{\itemindent}{-1\cslhangindent}
  \fi
  % set entry spacing
  \setlength{\itemsep}{#2\baselineskip}}}
 {\end{list}}
\usepackage{calc}
\newcommand{\CSLBlock}[1]{\hfill\break\parbox[t]{\linewidth}{\strut\ignorespaces#1\strut}}
\newcommand{\CSLLeftMargin}[1]{\parbox[t]{\csllabelwidth}{\strut#1\strut}}
\newcommand{\CSLRightInline}[1]{\parbox[t]{\linewidth - \csllabelwidth}{\strut#1\strut}}
\newcommand{\CSLIndent}[1]{\hspace{\cslhangindent}#1}

\KOMAoption{captions}{tableheading,figureheading}
\makeatletter
\@ifpackageloaded{tcolorbox}{}{\usepackage[skins,breakable]{tcolorbox}}
\@ifpackageloaded{fontawesome5}{}{\usepackage{fontawesome5}}
\definecolor{quarto-callout-color}{HTML}{909090}
\definecolor{quarto-callout-note-color}{HTML}{0758E5}
\definecolor{quarto-callout-important-color}{HTML}{CC1914}
\definecolor{quarto-callout-warning-color}{HTML}{EB9113}
\definecolor{quarto-callout-tip-color}{HTML}{00A047}
\definecolor{quarto-callout-caution-color}{HTML}{FC5300}
\definecolor{quarto-callout-color-frame}{HTML}{acacac}
\definecolor{quarto-callout-note-color-frame}{HTML}{4582ec}
\definecolor{quarto-callout-important-color-frame}{HTML}{d9534f}
\definecolor{quarto-callout-warning-color-frame}{HTML}{f0ad4e}
\definecolor{quarto-callout-tip-color-frame}{HTML}{02b875}
\definecolor{quarto-callout-caution-color-frame}{HTML}{fd7e14}
\makeatother
\makeatletter
\@ifpackageloaded{bookmark}{}{\usepackage{bookmark}}
\makeatother
\makeatletter
\@ifpackageloaded{caption}{}{\usepackage{caption}}
\AtBeginDocument{%
\ifdefined\contentsname
  \renewcommand*\contentsname{Table of contents}
\else
  \newcommand\contentsname{Table of contents}
\fi
\ifdefined\listfigurename
  \renewcommand*\listfigurename{List of Figures}
\else
  \newcommand\listfigurename{List of Figures}
\fi
\ifdefined\listtablename
  \renewcommand*\listtablename{List of Tables}
\else
  \newcommand\listtablename{List of Tables}
\fi
\ifdefined\figurename
  \renewcommand*\figurename{Figure}
\else
  \newcommand\figurename{Figure}
\fi
\ifdefined\tablename
  \renewcommand*\tablename{Table}
\else
  \newcommand\tablename{Table}
\fi
}
\@ifpackageloaded{float}{}{\usepackage{float}}
\floatstyle{ruled}
\@ifundefined{c@chapter}{\newfloat{codelisting}{h}{lop}}{\newfloat{codelisting}{h}{lop}[chapter]}
\floatname{codelisting}{Listing}
\newcommand*\listoflistings{\listof{codelisting}{List of Listings}}
\makeatother
\makeatletter
\makeatother
\makeatletter
\@ifpackageloaded{caption}{}{\usepackage{caption}}
\@ifpackageloaded{subcaption}{}{\usepackage{subcaption}}
\makeatother
\ifLuaTeX
  \usepackage{selnolig}  % disable illegal ligatures
\fi
\usepackage{bookmark}

\IfFileExists{xurl.sty}{\usepackage{xurl}}{} % add URL line breaks if available
\urlstyle{same} % disable monospaced font for URLs
\hypersetup{
  pdftitle={Comparing LLM and human reviews of social science research using data from Unjournal.org},
  pdfauthor={David Reinstein; Valentin Klotzbücher; Tianmai Michael Zhang},
  colorlinks=true,
  linkcolor={blue},
  filecolor={Maroon},
  citecolor={Blue},
  urlcolor={Blue},
  pdfcreator={LaTeX via pandoc}}

\title{Comparing LLM and human reviews of social science research using
data from Unjournal.org}
\author{David Reinstein \and Valentin Klotzbücher \and Tianmai Michael
Zhang}
\date{2025-11-30}

\begin{document}
\maketitle
\begin{abstract}
We will build and refine LLM tools to generate peer-reviews and ratings
of impactful research, and compare these with human experts' work
(esp.~from Unjournal.org): to benchmark performance, understand AI's
research taste, and develop tools to improve research evaluation and
dissemination.
\end{abstract}

\renewcommand*\contentsname{Table of contents}
{
\hypersetup{linkcolor=}
\setcounter{tocdepth}{2}
\tableofcontents
}
\bookmarksetup{startatroot}

\chapter{Introduction}\label{introduction}

\begin{tcolorbox}[enhanced jigsaw, title=\textcolor{quarto-callout-warning-color}{\faExclamationTriangle}\hspace{0.5em}{Work in progress}, titlerule=0mm, colbacktitle=quarto-callout-warning-color!10!white, coltitle=black, opacitybacktitle=0.6, bottomrule=.15mm, left=2mm, toptitle=1mm, colback=white, arc=.35mm, colframe=quarto-callout-warning-color-frame, toprule=.15mm, opacityback=0, bottomtitle=1mm, breakable, rightrule=.15mm, leftrule=.75mm]

Pages, metrics, and comparisons are under active development. Expect
rough edges and frequent updates.

\end{tcolorbox}

Is AI good at peer-reviewing? Does it offer useful and valid feedback?
Can it predict how human experts will rate research across a range of
categories? How can it help academics do this ``thankless'' task better?
Is it particularly good at spotting errors? Are there specific
categories, e.g.~spotting math errors or judging real-world relevance,
where it does surprisingly well or poorly? How does its ``research
taste'' compare to humans?

If AI research-evaluation works it could free up a lot of scientific
resources -- perhaps \$1.5 billion/year in the US alone Aczel, Szaszi,
and Holcombe (\citeproc{ref-aczel2021billion}{2021})) -- and offer more
continual and detailed review, helping improve research. It could also
help characterize methodological strengths/weaknesses across papers,
aiding training and research direction-setting. Furthermore, a key
promise of AI is to directly improve science and research. Understanding
how AI engages with research evaluations may provide a window into its
values, abilities, and limitations.

In this project, we are testing the capabilities of current large
language models (LLMs), illustrating whether they can generate research
paper evaluations comparable to expert human reviews. The Unjournal
systematically prioritizes `impactful' research and pays for
high-quality human evaluations, structured quantified ratings, claim
identification and assessment, and predictions. In this project, we use
an AI (OpenAI's \texttt{GPT-5\ Pro} model) to review social science
research papers under the same criteria used by human reviewers for The
Unjournal.

Each paper is assessed on specific dimensions -- for example, the
strength of its evidence, rigor of methods, clarity of communication,
openness/reproducibility, relevance to global priorities, and overall
quality. The LLM will provide quantitative scores (with uncertainty
intervals) on these criteria and produce a written evaluation

Our initial dataset will include the 5 research papers that have
existing Unjournal human evaluations. For each paper, the AI will
generate: (1) numeric ratings on the defined criteria, (2)
identification of the paper's key claims, and (3) a detailed review
discussing the paper's contributions and weaknesses. We will then
compare the AI-generated evaluations to the published human evaluations.

In the next phase, we will focus on papers currently under evaluation,
i.e., where no human evaluation has been made public, to allow us to
rule out any contamination.

\subsection{Our work in context}\label{our-work-in-context}

Luo et al. (\citeproc{ref-Luo2025}{2025}) survey LLM roles from idea
generation to peer review, including experiment planning and automated
scientific writing. They highlight opportunities (productivity, coverage
of long documents) alongside governance needs (provenance, detection of
LLM-generated content, standardizing tooling) and call for reliable
evaluation frameworks.

Eger et al. (\citeproc{ref-eger2025}{2025}) provide a broad review of
LLMs in science and a focused discussion of AI‑assisted peer review.
They argue: (i) peer‑review data is scarce and concentrated in
CS/OpenReview venues; (ii) targeted assistance that preserves human
autonomy is preferable to end‑to‑end reviewing; and (iii) ethics and
governance (bias, provenance, detection of AI‑generated text) are
first‑class constraints.

Zhang and Abernethy (\citeproc{ref-Zhang2025}{2025}) propose deploying
LLMs as quality checkers to surface critical problems instead of
generating full narrative reviews. Using papers from WITHDRARXIV and an
automatic evaluation framework that leverages ``LLM-as-judge,'' they
find the best performance from top reasoning models but still recommend
human oversight.

Pataranutaporn et al. (\citeproc{ref-Pataranutaporn2025}{2025}) asked
four nearly state-of-the-art LLM models (GPT-4o mini, Claude 3.5 Haiku,
Gemma 3 27B, and LLaMA 3.3 70B) to consider 1220 unique papers ``drawn
from 110 economics journals excluded from the training data of current
LLMs''. They prompted the models to act ``in your capacity as a reviewer
for {[}a top-5 economics journal{]}'' and make a publication
recommendation using a 6-point scale ranging from ``1 = Definite
Reject\ldots{}'' to ``6. Accept As Is\ldots{}''. They asked it to
evaluate each paper on a 10-point scale for originality, rigor, scope,
impact, and whether it was `written by AI'. They also (separately) had
LLMs rate 330 papers with the authors' identities removed, or replacing
the names with fake male/female names and real elite or non-elite
institutions (check this) or with prominent male or female economists
attached.

They compare the LLMs' ratings with the RePEC rankings for the journals
the papers were published in, finding general alignment. They find mixed
results on detecting AI-generated papers. In the names/institutions
comparisons, they also find the LLMs show biases towards named
high-prestige male authors relative to high-prestige female authors, as
well as biases towards elite institutions and US/UK universities.
(Doublecheck the details here).

There have been several other empirical benchmarking projects, including
work covered in LLM4SR: A Survey on Large Language Models for Scientific
Research and \href{https://arxiv.org/abs/2502.05151}{Transforming
Science with Large Language Models: A Survey on AI-assisted Scientific
Discovery, Experimentation, Content Generation, and Evaluation}. (We
will discuss these here.)

Zhang et al. (\citeproc{ref-Zhang2025Replication}{2025})

\begin{itemize}
\item
  AI conference paper data
\item
  ``employs LLM agents to perform pairwise comparisons among
  manuscripts''
\item
  ``significantly outperforms traditional rating-based methods in
  identifying high-impact papers'' {[}by citation metrics{]}
\item
  Some evidence of biases/\textasciitilde statistical discrimination
  based on characteristics like `papers from established research
  institutions'
\end{itemize}

Our project distinguishes itself in its use of \emph{actual} human
evaluations of research in economics and adjacent fields, past and
\emph{prospective}, including both reports, ratings, and
predictions.\footnote{Other work has relied on collections of research
  and grant reviews, including NLPEER, SubstanReview, and the Swiss
  National Science Foundation. That data has a heavy focus on
  computer-science adjacent fields, and iss less representative of
  mainstream research peer review practices in older, established
  academic fields. Note that The Unjournal commissions the evaluation of
  impactful research, often from high-prestige working paper archives
  like NBER, and makes all evaluations public, even if they are highly
  critical of the paper.} The Unjournal's 50+ evaluation packages enable
us to train and benchmark the models. Their pipeline of future
evaluations allow for clean out-of-training-data predictions and
evaluation. Their detailed written reports and multi-dimensional ratings
also allows us to compare the `taste', priorities, and comparative
ratings of humans relative to AI models across the different criteria
and domains. The `journal tier prediction' outcomes also provides an
external ground-truth\footnote{About verifiable publication outcomes,
  not about the `true quality' of the paper of course.} enabling a
human-vs-LLM horse race. We are also planning multi-armed trials on
these human evaluations (cf.~Brodeur et al, 2025 and Qazi et al, 2025)
to understand the potential for \emph{hybrid} human-AI evaluation in
this context.

Footnote, a fancier way to say this, from a grant application? Or from
chatGPT?\footnote{Our approach differs from prior work by (i) focusing
  on structured, percentile-based quantitative ratings with credible
  intervals that map to decision-relevant dimensions used by The
  Unjournal; (ii) comparing those ratings to published human evaluations
  rather than using LLM-as-judge; and (iii) curating contamination-aware
  inputs (paper text extraction with reference-section removal and token
  caps), with a roadmap to add multi-modal checks when we score figure-
  or table-dependent criteria.}

\bookmarksetup{startatroot}

\chapter{Data and methods}\label{data-and-methods}

We draw on two main sources:

\begin{enumerate}
\def\labelenumi{\arabic{enumi})}
\tightlist
\item
  Human evaluations from
  \href{https://unjournal.github.io/unjournaldata/index.html}{The
  Unjournal's public evaluation data} (PubPub reports and the Coda
  evaluation form export).\\
\item
  LLM‑generated evaluations using a structured JSON‑schema prompt with
  \texttt{gpt-5-pro-2025-10-06} (knowledge cut-off: 30 September 2024).
\end{enumerate}

\section{Unjournal.org evaluations}\label{unjournal.org-evaluations}

We use The Unjournal's public data for a baseline comparison. At The
Unjournal each paper is typically evaluated (aka `reviewed') by two
expert evaluators\footnote{Occasionally they use 1 or 3 evaluators.} who
provide quantitative ratings on a 0--100 percentile scale for each of
seven criteria (with 90\% credible intervals),\footnote{See their
  guidelines
  \href{https://globalimpact.gitbook.io/the-unjournal-project-and-communication-space/policies-projects-evaluation-workflow/evaluation/guidelines-for-evaluators\#quantitative-metrics}{here};
  these criteria include ``Overall assessment'', ``Claims, strength and
  characterization of evidence'', ``Methods: Justification,
  reasonableness, validity, robustness'', ``Advancing knowledge and
  practice'', ``Logic and communication'', ``Open, collaborative,
  replicable science'', and ``Relevance to global priorities, usefulness
  for practitioners''} two ``journal tier'' ratings on a 0.0 - 5.0
scale,\footnote{``a normative judgment about `how well the research
  should publish'\,'' and ``a prediction about where the research will
  be published''} a written evaluation (resembling a referee report for
a journal), and identification and assessment of the paper's ``main
claim''. For our initial analysis, we extracted these human ratings and
aggregated them, taking the average score per criterion across
evaluators (and noting the range of individual scores).

All papers have completed The Unjournal's evaluation process (meaning
the authors received a full evaluation on the Unjournal platform, which
has been publicly posted at unjournal.pubpub.org). The sample includes
papers spanning 2017--2025 working papers in development economics,
growth, health policy, environmental economics, and related fields that
The Unjournal identified as high-impact. Each of these papers has
quantitative scores from at least one human evaluator, and many have
multiple (2-3) human ratings.

\section{LLM-based evaluation}\label{llm-based-evaluation}

\subsection{Quantitative ratings and journal-ranking
tiers}\label{quantitative-ratings-and-journal-ranking-tiers}

Following The Unjournal's
\href{https://globalimpact.gitbook.io/the-unjournal-project-and-communication-space/policies-projects-evaluation-workflow/evaluation/guidelines-for-evaluators}{standard
guidelines for evaluators} and their
\href{https://coda.io/form/Unjournal-Evaluation-form-academic-stream-Coda-updated-version_dGjfMZ1yXME}{academic
evaluation form}, evaluators are asked to consider each paper along the
following dimensions: \textbf{claims \& evidence}, \textbf{methods},
\textbf{logic \& communication}, \textbf{open science}, \textbf{global
relevance}, and an \textbf{overall} assessment. Ratings are interpreted
as percentiles relative to serious recent work in the same area. For
each metric, evaluators are asked for the midpoint of their beliefs and
their 90\% credible interval, to communicate their uncertainty. For the
journal rankings measure, we ask both ``what journal ranking tier should
this work be published in? (0.0-5.0)'' and ``what journal ranking tier
will this work be published in? (0.0-5.0)'', with some further
explanation.The full prompt can be seen in the code below -- essentially
copied from the Unjournal's guidelines page.

We captured the versions of each paper that was evaluated by The
Unjournal's human evaluators, downloading from the links provided in The
Unjournal's Coda database.

We evaluate each paper by passing the PDF directly to the model and
requiring a strict, machine‑readable JSON output. This keeps the
assessment tied to the document the authors wrote. Direct ingestion
preserves tables, figures, equations, and sectioning, which ad‑hoc text
scraping can mangle. It also avoids silent trimming or segmentation
choices that would bias what the model sees.

We enforce a JSON Schema for the results. The model must return one
object for each of the named criteria including a midpoint rating and a
90\% interval for each rating. This guarantees that every paper is
scored on the same fields with the same types and bounds. It makes the
analysis reproducible and comparisons clean.

We request credible intervals (as we do for human evaluators) to allow
the model to communicate its uncertainty rather than suggest false
precision; these can also be incorporated into our metrics, penalizing a
model's inaccuracy more when it's stated with high confidence.

Relying on GPT-5 Pro, we use a single‑step call with a reasoning model
that supports file input. One step avoids hand‑offs and summary loss
from a separate ``ingestion'' stage. The model reads the whole PDF and
produces the JSON defined above. We do not retrieve external sources or
cross‑paper material for these scores; the evaluation is anchored in the
manuscript itself.

The Python pipeline uploads each PDF once and caches the returned file
id keyed by path, size, and modification time. We submit one background
job per PDF to the OpenAI Responses API with ``high'' reasoning effort
and server‑side JSON‑Schema enforcement. Submissions record the response
id, model id, file id, status, and timestamps.

We then polls job status and, for each completed job, retrieve the raw
JSON object, and write the responses to disk.

\bookmarksetup{startatroot}

\chapter{Results}\label{results}

Here we present preliminary results, starting with a comparison of the
LLM‑generated quantitative ratings (model: \texttt{gpt-5-pro}, see
the\href{methods.qmd}{(previous section}) with human evaluations across
\href{https://globalimpact.gitbook.io/the-unjournal-project-and-communication-space/policies-projects-evaluation-workflow/evaluation/guidelines-for-evaluators\#undefined-1}{the
Unjournal's criteria}.

\section{Quantitative comparison: human vs.~GPT‑5
Pro}\label{quantitative-comparison-human-vs.-gpt5-pro}

We first use the earlier GPT‑5 Pro evaluation run that covered all
papers in our Unjournal sample with a simpler JSON‑schema prompt.
Figure~\ref{fig-forest-overall} shows the overall percentile ratings
from this initial run, averaged across human evaluators and compared to
the LLM's ``overall'' scores for each paper.

\begin{figure}

\caption{\label{fig-forest-overall}Comparison of Human vs LLM overall
percentile ratings}

\includegraphics{results_files/figure-pdf/fig-forest-overall-1.pdf}

\end{figure}%

\begin{figure}

\caption{\label{fig-forest-tiers-should}Comparison of Human vs LLM
journal tier ratings (should be published in)}

\includegraphics{results_files/figure-pdf/fig-forest-tiers-should-1.pdf}

\end{figure}%

Figure~\ref{fig-heatmap-human-minus-llm} shows a heatmap of the
differences between human and LLM mean ratings across all evaluation
criteria. Positive values (in green) indicate that humans rated the
paper higher than the LLM, while negative values (in orange) indicate
the opposite.

\begin{figure}

\caption{\label{fig-heatmap-human-minus-llm}Heatmap of Human minus LLM
mean ratings across evaluation criteria}

\includegraphics{results_files/figure-pdf/fig-heatmap-human-minus-llm-1.pdf}

\end{figure}%

\clearpage

\section{Qualitative comparison: detailed GPT‑5 Pro
evaluations}\label{qualitative-comparison-detailed-gpt5-pro-evaluations}

To understand what GPT‑5 Pro is actually responding to, we re‑ran the
model on four focal papers (\citeproc{ref-Adena2024}{Adena and Hager
2024}; \citeproc{ref-Peterman2024}{Peterman et al. 2024};
\citeproc{ref-Williams2024}{Williams et al. 2024};
\citeproc{ref-Green2025}{Green, Smith, and Mathur 2025}) using a refined
prompt.

This second run keeps the same quantitative metrics but additionally
requires a diagnostic summary of about 1,000 words and high‑effort
reasoning, with the full reasoning trace returned by the ``thinking''
model. For each paper we can therefore inspect:

\begin{itemize}
\tightlist
\item
  the LLM's quantitative scores and journal‑tier predictions,
\item
  the hidden reasoning steps used to arrive at those scores, and
\item
  the token usage and approximate API cost of the evaluation.
\end{itemize}

Table~\ref{tbl-llm-token-cost-summary} summarizes the token usage and
estimated cost of each of these inferences.

We start by examining the Williams et al.~(2024) evaluation in detail
and then show the analogous summaries for the other four focal papers.
In the next step we will juxtapose these LLM assessments with the human
evaluators' written reports.

\begin{longtable}[]{@{}
  >{\raggedright\arraybackslash}p{(\columnwidth - 10\tabcolsep) * \real{0.2234}}
  >{\raggedleft\arraybackslash}p{(\columnwidth - 10\tabcolsep) * \real{0.1383}}
  >{\raggedleft\arraybackslash}p{(\columnwidth - 10\tabcolsep) * \real{0.1489}}
  >{\raggedleft\arraybackslash}p{(\columnwidth - 10\tabcolsep) * \real{0.1809}}
  >{\raggedleft\arraybackslash}p{(\columnwidth - 10\tabcolsep) * \real{0.1383}}
  >{\raggedleft\arraybackslash}p{(\columnwidth - 10\tabcolsep) * \real{0.1702}}@{}}

\caption{\label{tbl-llm-token-cost-summary}Estimated token usage and
cost of GPT-5 Pro evaluations}

\tabularnewline

\toprule\noalign{}
\begin{minipage}[b]{\linewidth}\raggedright
Paper
\end{minipage} & \begin{minipage}[b]{\linewidth}\raggedleft
Input tokens
\end{minipage} & \begin{minipage}[b]{\linewidth}\raggedleft
Output tokens
\end{minipage} & \begin{minipage}[b]{\linewidth}\raggedleft
Reasoning tokens
\end{minipage} & \begin{minipage}[b]{\linewidth}\raggedleft
Total tokens
\end{minipage} & \begin{minipage}[b]{\linewidth}\raggedleft
Est. cost (USD)
\end{minipage} \\
\midrule\noalign{}
\endhead
\bottomrule\noalign{}
\endlastfoot
Williams et al.~2024 & 28890 & 6795 & 5056 & 35685 & 1.86 \\
Green et al.~2025 & 23090 & 6938 & 5440 & 30028 & 1.83 \\
Adena and Hager 2024 & 24420 & 6211 & 4544 & 30631 & 1.66 \\
Peterman et al.~2025 & 18948 & 6091 & 4480 & 25039 & 1.55 \\

\end{longtable}

\subsection{Qualitative comparison: Williams et
al.~(2024)}\label{qualitative-comparison-williams-et-al.-2024}

In the refined run, GPT‑5 Pro reads about 28,704 input tokens and
produces 6,327 visible output tokens plus 5,120 reasoning tokens.\\
At current API prices this evaluation costs roughly \$1.80.

The table below shows the model's percentile ratings and 90\% credible
intervals for the Unjournal criteria.

\begin{longtable}[t]{lccc}

\caption{\label{tbl-llm-williams-metrics}GPT-5 Pro percentile ratings
for Williams et al.~(2024)}

\tabularnewline

\toprule
Criterion & Midpoint & Lower 90\% & Upper 90\%\\
\midrule
Overall assessment & 86 & 86 & 86\\
Claims \& evidence & 78 & 78 & 78\\
Methods & 74 & 74 & 74\\
Advancing knowledge and practice & 92 & 92 & 92\\
Logic and communication & 84 & 84 & 84\\
\addlinespace
Open, collaborative, replicable science & 63 & 63 & 63\\
Relevance to global priorities & 94 & 94 & 94\\
\bottomrule

\end{longtable}

\begin{longtable}[t]{lccc}

\caption{\label{tbl-llm-williams-tiers}GPT-5 Pro journal tier ratings
for Williams et al.~(2024)}

\tabularnewline

\toprule
Measure & Score & Lower 90\% & Upper 90\%\\
\midrule
Deserved journal tier (should) & 4.4 & 4.4 & 4.4\\
Predicted journal tier (will) & 4.8 & 4.8 & 4.8\\
\bottomrule

\end{longtable}

\bookmarksetup{startatroot}

\chapter{Discussion}\label{discussion}

\section{Limitations}\label{limitations}

\emph{Sample size and representativeness:} We evaluated on only
\textasciitilde40--50 papers, all in the social sciences and
specifically chosen by The Unjournal for evaluation (which means they
were considered high-impact or interesting). This is not a random sample
of research literature. The papers also skew toward empirical and
policy-relevant topics. The AI's performance and alignment might differ
in other fields (e.g., pure theory, biology) or on less polished papers.

\emph{Human agreement as a moving target:} The Unjournal human
evaluations themselves are not a single ground truth. As evidence of
this, we note substantial variability between reviewers.

\emph{Potential AI knowledge contamination:} We attempted to prevent
giving the AI any information about the human evaluations, but we cannot
be 100\% sure that the model's training data didn't include some
fragment of these papers, related discussions, or even The Unournal
evaluations. We will be able to exclude this for the evaluations of
\emph{future} Unjournal evaluations.

Model limitations and ``alignment'' issues: While powerful, is not a
domain expert with judgment honed by years of experience. It might be
overly influenced by how a paper is written (fluency) or by irrelevant
sections. It also tends to avoid extremely harsh language or low scores
unless there is a clear reason, due to its alignment training to be
helpful/polite -- this could explain the general score inflation we
observed. The model might fail to catch subtle methodological flaws that
a field expert would notice, or conversely it might ``hallucinate'' a
concern that isn't actually a problem. Without ground truth about a
paper's actual quality, we used human consensus as a proxy; if the
humans overlooked something, the AI could appear to ``disagree'' but
possibly be pointing to a real issue.

(There is also evidence, e.g. Pataranutaporn et al.
(\citeproc{ref-Pataranutaporn2025}{2025}) that LLMs show biases towards
more prestigious author names, institutions, and towards male
prestigious authors. We will provide further evidence on this in the
next iterations, de-identifying the work under LLM evaluation.)

Scoring calibration: The AI was prompted to use the 0--100 percentile
scale, but calibrating that is hard. Humans likely had some calibration
from guidelines or community norms (e.g.~perhaps very few papers should
get above 90). The AI might have been more liberal in using the high end
of the scale (hence higher means). In future, a different prompt or
examples could calibrate it to match the distribution of human scores
more closely. We also only took one run from the AI for each paper; LLM
outputs can have randomness, so a different run might vary slightly. (To
do: aggregate across multiple runs.)

Small differences and rounding: Our analysis treated the AI's numeric
outputs at face value. Small differences (e.g.~AI 85 vs human 82) might
not be meaningful in practice -- both indicate a similar qualitative
assessment (``very good''). Some of our metrics (like kappa) penalize
any difference, even if minor. Thus, the ``low agreement'' statistics
might sound worse than the reality where in many cases AI and humans
were only off by a few points. We intend to analyze the distribution of
absolute differences: a large portion might be within say ±5 points
which could be considered essentially agreement in practice. The
credible intervals add another layer: sometimes an AI's score fell
outside a human's interval, but overlapping intervals could still mean
they agree within uncertainty. We did observe that AI's intervals were
often narrower than humans' (LLM tended to be confident, giving
\textasciitilde10-point spreads, whereas some human evaluators gave
20-point or left some intervals blank), which is another aspect of
calibration.

\bookmarksetup{startatroot}

\chapter*{References}\label{references}
\addcontentsline{toc}{chapter}{References}

\markboth{References}{References}

\phantomsection\label{refs}
\begin{CSLReferences}{1}{0}
\bibitem[\citeproctext]{ref-aczel2021billion}
Aczel, Balazs, Barnabas Szaszi, and Alex O Holcombe, {``A billion-dollar
donation: Estimating the cost of researchers' time spent on peer
review,''} \emph{Research integrity and peer review}, 6 (2021), 1--8
(Springer).

\bibitem[\citeproctext]{ref-Adena2024}
Adena, Maja, and Anselm Hager, {``Does online fundraising increase
charitable giving? A nationwide field experiment on facebook,''} 2024
(WZB Discussion Paper Discussion Paper SP II 2020--302r2 (2nd revision
February 2024)).

\bibitem[\citeproctext]{ref-eger2025}
Eger, Steffen, Yong Cao, Jennifer D'Souza, Andreas Geiger, Christian
Greisinger, Stephanie Gross, Yufang Hou, Brigitte Krenn, Anne Lauscher,
Yizhi Li, Chenghua Lin, Nafise Sadat Moosavi, Wei Zhao, and Tristan
Miller, {``\href{https://arxiv.org/abs/2502.05151}{Transforming science
with large language models: A survey on AI-assisted scientific
discovery, experimentation, content generation, and evaluation},''}
\emph{arXiv preprint arXiv:2505.05151}, (2025).

\bibitem[\citeproctext]{ref-Green2025}
Green, Seth Ariel, Benny Smith, and Maya Mathur,
{``\href{https://Elsevier}{Meaningfully reducing consumption of meat and
animal products is an unsolved problem: A meta-analysis},''}
\emph{Appetite}, (2025), 108233.

\bibitem[\citeproctext]{ref-Luo2025}
Luo, Ziming, Zonglin Yang, Zexin Xu, Wei Yang, and Xinya Du,
{``\href{https://arxiv.org/abs/2501.04306}{LLM4SR: A survey on large
language models for scientific research},''} \emph{arXiv preprint
arXiv:2501.04306}, (2025).

\bibitem[\citeproctext]{ref-Pataranutaporn2025}
Pataranutaporn, Pat, Nattavudh Powdthavee, Chayapatr Achiwaranguprok,
and Pattie Maes,
{``\href{https://api.semanticscholar.org/CorpusID:277510314}{Can AI
solve the peer review crisis? A large scale cross model experiment of
LLMs' performance and biases in evaluating over 1000 economics
papers},''} 2025.

\bibitem[\citeproctext]{ref-Peterman2024}
Peterman, Amber, Jingying Wang, Kevin Kamto Sonke, and Janina Steinert,
{``Social safety nets, women's economic achievements and agency: A
systematic review and meta-analysis,''} \emph{Center for Global
Development Working Papers}, (2024).

\bibitem[\citeproctext]{ref-Williams2024}
Williams, Brooke A, Hawthorne L Beyer, Matthew E Fagan, Robin L Chazdon,
Marina Schmoeller, Starry Sprenkle-Hyppolite, Bronson W Griscom, James
EM Watson, Anazélia M Tedesco, Mariano Gonzalez-Roglich, and others,
{``\href{https://doi.org/10.1038/s41586-024-08106-4}{Global potential
for natural regeneration in deforested tropical regions},''}
\emph{Nature}, 636 (2024), 131--137.

\bibitem[\citeproctext]{ref-Zhang2025}
Zhang, Tianmai M, and Neil F Abernethy,
{``\href{https://arxiv.org/abs/2505.23824v2}{Reviewing scientific papers
for critical problems with reasoning LLMs: Baseline approaches and
automatic evaluation},''} \emph{arXiv preprint arXiv:2505.23824},
(2025).

\bibitem[\citeproctext]{ref-Zhang2025Replication}
Zhang, Yaohui, Haijing Zhang, Wenlong Ji, Tianyu Hua, Nick Haber,
Hancheng Cao, and Weixin Liang,
{``\href{https://arxiv.org/abs/2506.11343v1}{From replication to
redesign: Exploring pairwise comparisons for LLM-based peer review},''}
\emph{arXiv preprint arXiv:2506.11343}, (2025).

\end{CSLReferences}

\cleardoublepage
\phantomsection
\addcontentsline{toc}{part}{Appendices}
\appendix

\chapter{LLM evaluation summaries and reasoning
traces}\label{llm-evaluation-summaries-and-reasoning-traces}

\renewcommand\normalsize{\small}

\section{Adena and Hager 2024}\label{adena-and-hager-2024}

\begin{tcolorbox}[enhanced jigsaw, title=\textcolor{quarto-callout-note-color}{\faInfo}\hspace{0.5em}{Model assessment summary}, titlerule=0mm, colbacktitle=quarto-callout-note-color!10!white, coltitle=black, opacitybacktitle=0.6, bottomrule=.15mm, left=2mm, toptitle=1mm, colback=white, arc=.35mm, colframe=quarto-callout-note-color-frame, toprule=.15mm, opacityback=0, bottomtitle=1mm, breakable, rightrule=.15mm, leftrule=.75mm]

This is a large, well-executed geo-randomized natural field experiment
that evaluates whether a nationwide Facebook video campaign increases
charitable giving for one major NGO, while also exploring spillovers to
competitors and mechanisms. The core identification strategy---random
assignment of almost all German postal codes (PLZs) to exposure or
control, with additional 2x2 variation in video content (empathy vs
effectiveness) and impression allocation (Facebook free allocation vs
fixed PLZ budgets)---is strong. Outcomes are measured as total donations
to the focal charity at the PLZ-day level across all channels, directly
addressing common problems in ad-effectiveness studies (click-based
selection, channel substitution, and unobserved offline giving). The
study reports positive effects on donation frequency in both short and
long horizons and a smaller but positive long-term effect on donation
revenue; estimated immediate ROI is \textgreater1 and plausibly
\textgreater2 in the longer run given assumptions about donor lifetime
value. The design was preregistered, balance is shown, and several
robustness checks (including Fisherian randomization inference and
sensitivity to post-period length) are provided.

The most important methodological limitations concern exposure
heterogeneity and spillovers. Treatment is assigned at the PLZ level,
but impressions are probabilistic and sparse (roughly one in ten
Facebook users in treated PLZs received at least one impression), so the
estimates are ITT and likely attenuated relative to the effect of
actually seeing the ad; the TOT is not estimated. The allocation
strategy partly allows Facebook to endogenously concentrate impressions,
creating within-treatment variation in exposure that is not exploited
for causal TOT analysis (e.g., using randomized budgets as an instrument
in a dose--response framework). Spillovers across PLZs are plausible
(algorithmic leakage of geotargeting and social diffusion). The authors
document positive ``share of treated neighbors'' effects and argue the
main estimates are lower bounds, but the neighbor-treatment share is not
itself randomized, and spatial correlation or common shocks could
inflate these coefficients; the spillover analysis should be interpreted
cautiously. Robustness to spatial correlation in errors is only partly
addressed by robust standard errors and randomization inference;
alternative SEs (e.g., spatial HAC or clustering at larger
administrative units) and placebo geographies would further strengthen
inference.

Data construction choices appear reasonable but introduce some judgment
calls. Winsorizing PLZ-day donations at €1,000 reduces variance from
heavy tails; the authors show that results are directionally robust, but
precision trades off. Outcomes are scaled per million inhabitants per
day, which largely equalizes variance across PLZs, but no explicit
heteroskedastic weights are applied. The pre-period balance tables show
small, nonsignificant differences; preferred specifications include
lagged outcomes and block FEs, and difference-in-differences checks
yield similar conclusions. Given the randomized design, these
adjustments mainly improve precision. Reporting both ANCOVA and DiD
estimates as main tables is helpful; preregistration notes some
deviations (adding the allocation-strategy factor and including reach in
blocking), which should be clearly tracked against pre-analysis plans
for transparency; these deviations plausibly increase power but can
raise concerns about researcher degrees of freedom.

The competitor/crowding-out analysis is an important contribution but
faces data and identification constraints. The ``23-charity alliance''
dataset includes only online donations, not offline, and exhibits
pretreatment imbalances, addressed with ANCOVA and DiD; effects on
revenue are negative and sometimes significant, frequency effects are
weaker. The Betterplace platform data exclude the focal charity but
involve overlapping entities with the alliance and cover a broader set
of projects; here the short-run interaction (children-related
categories) is negative and significant on revenue while long-run
results weaken. Overlap between sources and partial observability of the
sector complicate aggregation; the paper correctly refrains from summing
these effects. Nonetheless, the evidence is consistent with meaningful
crowd-out among similar causes. Clarifying the base rates and providing
standardized effect sizes (e.g., as a percent of baseline giving per
PLZ-day) would aid interpretation. A placebo category less plausibly
affected by the campaign (e.g., sports or animals) is used indirectly
via the non-children grouping; sharper falsification tests (e.g.,
categories orthogonal to child/international relief) would further
bolster the crowd-out claim.

Mechanism and design-variant results are well documented. Empathy videos
generate more short-run attention and immediate donations, but long-run
donation effects equalize with effectiveness-oriented videos,
highlighting the danger of optimizing to clicks or short-watch metrics.
Letting Facebook freely allocate impressions seems to perform at least
as well as fixed PLZ budgets in both intermediate and longer-run
outcomes; differences are not statistically large. These findings are
valuable for practitioners but should be interpreted as specific to this
campaign's creative, audience, and season.

The profitability analysis is transparent yet assumption-sensitive. The
immediate ROI calculation is straightforward, but the long-run
multiplier of 1.75 relies on external retention estimates and assumes
similar carryover for new and existing donors; it also excludes
creative/overhead costs beyond media spend. Confidence intervals on
monetary impacts are wide, and the point estimates rely on p\textless0.1
for revenue; presenting ROI distributions via parametric or bootstrap
uncertainty propagation (combining outcome and cost variance) would
improve decision relevance. Comparison to a hypothetical direct mail
benchmark is informative, though based on stylized assumptions about
response rates.

External validity is high in several dimensions: nationwide scope,
minimal individual targeting, and use of total donations across
channels. Still, the context (Germany, late-year giving season, specific
charity and creatives) limits portability. Heterogeneity analyses
suggest stronger effects in urban areas and in PLZs with higher
employment, more children, and more Catholics, and with higher predicted
giving potential. These patterns can guide targeting but also indicate
that the ITT estimates average over meaningful heterogeneity. Reporting
treatment-on-potential-exposure effects (e.g., dose--response by
observed impressions per capita in the fixed-budget arm) could add
nuance.

Open-science practices are mixed. The design was preregistered on
EGAP/OSF and the appendix is publicly hosted; however, code and analysis
data are not provided (likely due to confidentiality), limiting
reproducibility. The description of blocking, assignment, and all
outcomes is detailed enough that a qualified team could replicate
analyses if granted data access, but turnkey replication is not
possible.

Overall, the paper credibly demonstrates that a broad-reach social media
fundraising campaign can increase total donations to a large charity and
likely crowds out giving to similar causes. The empirical strategy is
sound; the analyses are careful and transparent; and the practical
lessons---avoid relying on clicks, expect competition effects, and
consider algorithmic allocation---are valuable for practitioners. Main
caveats relate to exposure heterogeneity, spatial spillover
identification, sector-wide measurement for competitors, and ROI
assumptions. The paper would benefit from additional spatially robust
SEs, clearer pre-analysis plan deviation tracking, a formal uncertainty
propagation for ROI, and stronger placebo tests in the spillover and
crowd-out sections. Despite these limits, this is a strong and
policy-relevant contribution to the literatures on charitable giving and
digital advertising measurement.

\end{tcolorbox}

\begin{tcolorbox}[enhanced jigsaw, title=\textcolor{quarto-callout-tip-color}{\faLightbulb}\hspace{0.5em}{Model reasoning trace}, titlerule=0mm, colbacktitle=quarto-callout-tip-color!10!white, coltitle=black, opacitybacktitle=0.6, bottomrule=.15mm, left=2mm, toptitle=1mm, colback=white, arc=.35mm, colframe=quarto-callout-tip-color-frame, toprule=.15mm, opacityback=0, bottomtitle=1mm, breakable, rightrule=.15mm, leftrule=.75mm]

\textbf{Analyzing donation outcomes}

I'm using the timeframe from October 10 to December 31, specifically
looking at the 31 days before, 14 days of treatment from November 10 to
November 23, and then 38 days post-treatment. I'm also gathering
competitor data from 23 similar charities and examining various
categories. The results indicate a short-term increase in donation
frequency during the campaign and some long-term effects that show
significance, though the revenue estimates aren't as strong.

\textbf{Evaluating profitability and competition}

I'm calculating profitability, finding an immediate return of 1.45 euros
per euro spent, and longer-term, with a lifetime value multiplier of
1.75, yielding 2.53 euros. I've tested for channel substitution within
Save the Children, concluding no significant changes, as benefits came
from both new and repeat donors. However, competition data shows a
negative impact from Save the Children on others, with some notable
short-term revenue losses in children's projects, while empathy-driven
videos increased attention and immediate donations.

\textbf{Analyzing donation allocation}

I'm looking into how letting Facebook allocate donations more freely
leads to higher metrics compared to fixed budgets, especially in the
short term. While the differences aren't statistically significant,
there are consistent intermediate results. I've also introduced a
variable for the share of treated neighbors within 30km and found
positive impacts on both revenue and frequency, suggesting spillovers.
They compute the total effect to be around 170 euros, though I need to
verify some coefficients and ensure robustness, as placebo tests
indicate the effects aren't just chance.

\textbf{Examining competitor dataset and methodology}

I'm looking into competitor dataset \#1, which shows pre-existing
differences that they correct using DiD. They've winsorized at 1000
euros and claim that the results remain robust even without this
adjustment. There's a discussion of potential noise in Facebook's
location assignments, but they still analyze treatment effects. By
interacting with various characteristics like children share and
Catholic share, they find more positive effects in certain urban
settings. They mention an open science approach with preregistration and
discuss data confidentiality, making replication challenging.

\textbf{Clarifying methodological details}

They provide sufficient methodological details for reproduction if the
data were available, including daily donation data aggregated at the
postal code level. There's a scheduled plan to share all the details,
along with an OSF link for preregistration, but I'm unsure about the
availability of the code and data. There are limitations, notably that
exposures aren't measured at the postal code level, with randomization
ensuring unbiased estimation. They present effect sizes based on
millions of residents, with calculations matching scalar values for
additional donations.

\textbf{Reviewing ROI calculations}

I see that the effect is analyzed over both treatment and post periods,
totaling 52 days. For the ROI calculation, they use a long-term donor
multiplier of 1.75, applying it to both new and existing donors, which
raises some questions. They consider both immediate and long-term
additional donations in this figure. They end up with an estimated 2.53
euros per euro spent, but I'm unsure due to a wide 90\% confidence
interval. Also, it's important to note that their cost figure only
includes direct ad costs without accounting for other expenses, like
research and creative costs.

\textbf{Assessing cost, threats, and validity}

They report a direct cost of 33,700 euros but also include a disclaimer.
There are potential threats from contamination and spillovers across
postal codes, which they attempt to adjust for, yet can't fully
identify. This free allocation could lead to cross-targeting; the
increased neighbor share might also underestimate the control group.
They exclude extremes in reach distribution, limiting the external
validity of their findings but claim they consider this a lower bound.
Additionally, they note some baseline differences in variables, and the
inclusion of certain specifications, like winsorization, might introduce
measurement issues.

\textbf{Evaluating donation measurement and errors}

They're measuring Save the Children's donations across various channels,
which is great. However, the competitor data only includes online
giving, leaving the estimates for general equilibrium effects
potentially conservative or unbalanced. It seems competitor effects
might appear larger, possibly due to a bigger base, but there's the
issue of potential double counting within the data sources. For
analysis, they normalized zeros at the aggregated postal code level and
scaled donations per million inhabitants, but they didn't mention
weighting by population, which raises concerns about measurement error
across postal codes and the handling of heteroskedasticity.

\textbf{Analyzing statistical considerations}

They used robust standard errors, which is solid, but at the postal code
level, spatial correlation from neighboring areas could affect results.
They attempt to address this through neighbor shares, though robust
standard errors might not fully account for spatial correlation. There's
no presentation of cluster-robust errors at the county level. They also
use randomization inference, which helps, but I'm wondering if they
adjusted for multiple hypothesis testing. They test various interactions
and mention weak significance, relying on p-values below 0.1, which
could raise concerns about p-hacking. Adjusting for ad-serving to the
control group from Save the Children seems not accounted for either.

\textbf{Considering randomization and ad exposure}

They randomize at the postal code level, but there's a chance that
``control'' areas may still receive Save the Children ads due to
geofencing errors. I'm curious if they restricted Save the Children from
running other ad campaigns during the 14-day period. It sounds plausible
that they had concurrent marketing channels. They maintain equal
exposure across both groups, which is good, even if they can't guarantee
no offline or competitor exposures. They exclude the lowest 5\% and
highest 1\% of postal codes, which seems reasonable, ending up with a
final sample of 7,686, nice! They also apply gradient boosting to
predict donation potential using past donations and other covariates,
which sounds like a solid approach.

\textbf{Evaluating methodology and modeling choices}

They use estimated potential multiplied by Facebook reach to sort and
block areas, assigning six consecutive postal codes to each block---two
for control and four for treatment. This 2:4 ratio keeps a third as
control, which seems balanced. They specify no further targeting beyond
the PLZ level, improving external validity but potentially reducing
power, which is a smart move.

I notice they apply winsorization at 1,000 euros, though it's not
pre-registered, which is a modeling choice they mention as common in
literature. They justify this decision since only a small fraction (68
out of 11,140 PLZ-days) exceeds that threshold. Normalizing by
population and period length to present outcomes per million inhabitants
per day is also a solid approach. They report pretreatment baseline
imbalances in revenue and frequency, which appear not significant.

\textbf{Examining randomization and bias concerns}

They continue to include baseline revenue before campaign (RBC), which
is interesting. Their randomization inference uses 5,000 permutations,
which is a solid approach.

I have some concerns about potential algorithmic bias in ad assignments
that target cheaper audiences. However, they suggest that this fear is
overshadowed by more favorable outcomes from free allocation, which
seems reasonable.

I'm curious about the exposure share of Facebook users in the treated
postal codes---it's about one in ten. I wonder if they applied treatment
on the treated (TOT) versus intention-to-treat (ITT) methodologies here?

\textbf{Evaluating analysis methods}

They present a simple difference-of-means at the postal code level,
which follows the intention-to-treat (ITT) approach---this seems fine.

There's a noted lack of a pre-specified significance threshold for ROI,
but that doesn't seem necessary here. They also mention a lack of
covariate adjustments, yet they present robust findings.

I'm curious about the handling of the competitor dataset before and
after the campaign using a difference-in-differences (DiD) approach.
Additionally, I'm a bit puzzled about their classification of long-term
effects---are they trying to capture any potential carryover effects,
extending beyond the immediate timeframe of 38 days post-treatment?

\textbf{Crafting the assessment summary}

I want to begin with an overview of the assessment.

For major strengths, they georandomized at a national scale and used
real donation outcomes across channels. It's good that they
pre-registered their study and included competitor effects, utilizing
both true experimental treatment (TOT) and incremental results, along
with robust inference and consideration of assimilation.

However, there are notable weaknesses. External exposures are a concern
since individual-level ad exposures can't be measured, and variations in
exposures across postal codes might arise from algorithmic allocation.
The intention-to-treat (ITT) effect is small, and the TOT effect remains
unknown; they flagged an algorithm that suggests a problem.

For the ROI calculation, the assumptions regarding lifetime value (LTV)
are questionable, and I worry about excluding overhead costs and
weighting.

Additionally, while they report a positive neighbor-share coefficient,
the identification of spillovers seems weak. The correlation of treated
neighbors with other covariates complicates things, and without specific
randomization of neighbor share, it remains observational. Nonetheless,
randomization at the postal code level may help keep the distribution of
neighbor share exogenous across PLZ.

\textbf{Setting research metrics}

I need to establish metrics around benchmarking this study against
serious research in field experiments on charitable giving and digital
advertising measurement from the last three years (2022--2025). This
study stands out above average due to its design clarity and external
validity; it employs national georandomization and produces credible
results with direct measurement of base outcomes, including competitor
spillovers. While some results rely on p\textless0.1, and ROI depends on
assumptions, overall, I consider this a strong applied paper, placing it
around the 80--88 percentile. I might calibrate the claims evidence to
range from 78 to 85.

\textbf{Evaluating study metrics}

This study presents clear claims, and the evidence is robust, so I'd
rate it around 82. The methods are credible, using randomization and OLS
with day-level aggregation, also at 82. Since it addresses the
significant question of whether online fundraising boosts giving for
charities, I'd rate its contribution to advancing knowledge at 75--83.
The logic and communication are solid, resulting in a score of 86.

However, while it has pre-registration, the lack of shared code and data
brings down its open science score to 58. For global relevance, I'd lean
towards 70, but considering its matured state, I'd give it a 65.
Overall, I recommend a weighted score of 82, with confidence intervals
reflecting uncertainty at about 75--88.

\textbf{Calculating publication expectations}

I need to compute ``tier\_should'' and ``tier\_will,'' which refers to
normative expectations about where this paper should publish. The scale
goes from 0 to 5, with 5 being the top general A-journals like
Econometrica or QJE. Since this is a strong field-experimental paper, I
think it should rank in the top B-journals or top field journals,
possibly around 3.5 to 4.2. I'd estimate it should be about 3.8, with
uncertainty ranging from 3.2 to 4.3.

For ``tier\_will,'' given its status as a WZB discussion paper and the
need to disregard actual publication knowledge, I might predict it will
land around 3.2 to 4.0.

\textbf{Setting up the JSON schema}

I need to follow the JSON schema for this paper assessment. It requires
creating an object called ``paper\_assessment\_with\_tiers\_v2'' that
includes keys for ``assessment\_summary'' (a string) and ``metrics.''
The metrics will consist of objects like ``overall,'' each containing a
midpoint, lower\_bound, and upper\_bound numbers alongside
``tier\_should'' and ``tier\_will,'' which need a score and confidence
intervals.

Importantly, the numeric constraints must be set correctly. For
percentile metrics, I have to ensure they're on a 0--100 scale, with
lower values being less than midpoints, which must be less than upper
bounds. Similarly, for the tier scores, I should ensure they fall within
a 0--5 range where the confidence interval's lower value is less than
the score and upper value is greater.

Confidence intervals are set as 90\% credible intervals, and I'll
provide all numbers with one decimal point. The summary is suggested to
be around 1000 words, so I'll aim for that word count while structuring
it into about 7--10 paragraphs in order to deliver a thorough critique
alongside useful details.

\textbf{Identifying threats}

I need to focus on the potential threats to the study. They used
Winsorization, which is one aspect worth noting. However, there were
issues like unmeasured ad exposures, making the TOT unclear. There's
also variation in exposures across PLZ in the treatment due to free
allocation.

Competitor results depend solely on online donations, which may impact
the findings. I'm supposed to craft a summary without mentioning
external vantage packaging, so I must navigate that carefully.

Now, regarding the metrics, for the overall, I'm suggesting a midpoint
of 84 with a lower bound of 78 and an upper bound of 90. But, hmm, I
think I might be overestimating that.

\textbf{Clarifying the assessment metrics}

I need to finalize the assessment metrics. For claims evidence, I'm
setting it at 82, with bounds of 75 and 88. The methods score should be
83, ranging from 76 to 89. For advancing knowledge, I'm placing it at
77, with lower and upper bounds of 70 and 84.

Logic communication scores high at 87, while open science sits at a
lower 56, with bounds of 40 and 70. Global relevance is set at 67, with
a range of 58 to 76.

I also have disclaimers to consider while crafting the assessment
summary.

Now, for the summary, I'll aim for around 900 to 1100 words. I plan to
kick it off by discussing the paper's value and then move on to mention
georandomization and its overall robust design. Time to fill in the
JSON!

\textbf{Finalizing the summary structure}

In the summary, I'm organizing the content into distinct paragraphs.
Paragraph 3 will cover exposures, focusing on measuring donation
outcomes across channels while controlling for substitution. I note the
lack of observed exposures to ads and the unknown TOT, which isn't
unusual.

In Paragraph 4, I'll discuss the estimation approach, emphasizing OLS
and difference-of-means while using robust standard errors. Paragraph 5
will tackle pre-treatment imbalances, while Paragraph 6 will touch on
spillover effects.

By Paragraph 7, I'll consider competitor impacts, followed by ROI in
Paragraph 8. I'll also investigate mechanisms and generalization,
replicability, and open science in subsequent paragraphs.

I'll ensure the JSON is strictly formatted and check if the overall
midpoint should be adjusted from 82 to 81, providing appropriate
increments. Lastly, I'll aim for around 900 to 1000 words for a crisp,
engaging summary. Time to produce!

\end{tcolorbox}

ewpage

\section{Green et al.~2025}\label{green-et-al.-2025}

\begin{tcolorbox}[enhanced jigsaw, title=\textcolor{quarto-callout-note-color}{\faInfo}\hspace{0.5em}{Model assessment summary}, titlerule=0mm, colbacktitle=quarto-callout-note-color!10!white, coltitle=black, opacitybacktitle=0.6, bottomrule=.15mm, left=2mm, toptitle=1mm, colback=white, arc=.35mm, colframe=quarto-callout-note-color-frame, toprule=.15mm, opacityback=0, bottomtitle=1mm, breakable, rightrule=.15mm, leftrule=.75mm]

This meta-analysis asks a clear and policy-relevant question: which
randomized interventions meaningfully reduce consumption of meat and
animal products (MAP) at least one day post-intervention? The authors
assemble 35 papers (41 studies; 112 interventions; \textasciitilde87,000
participants) and restrict inclusion to RCTs with minimum sample sizes
and non-immediate behavioral outcomes, then synthesize using robust
variance estimation (RVE). Major strengths include the focus on behavior
(not attitudes/intentions), careful separation of net MAP versus
red/processed meat (RPM) outcomes, longest-follow-up extraction to
emphasize persistence, transparent handling of multi-arm studies via RVE
with small-sample correction, publication-bias analyses using both
selection models and a ``worst-case'' non-affirmative subset, and
unusually strong computational reproducibility (Rmarkdown,
containerization, public repository). The headline finding---a small
pooled impact on net MAP consumption (SMD˜0.07, 95\% CI {[}0.02,
0.12{]}) that attenuates further under publication-bias corrections
(˜0.01--0.02)---is consistently and cautiously characterized. Still,
several methodological and reporting choices merit attention. First,
three inclusion decisions were made after data collection began
(creating a separate RPM category; excluding substitution-only
interventions; excluding settings without voluntary choice). These
appear defensible and are disclosed, but they introduce a modest risk of
researcher degrees of freedom and could shift the study mix toward
certain settings or mechanisms; a preregistered protocol would mitigate
this concern in future iterations. Second, the requirement of a ``pure
control'' arm excludes a number of quasi-experimental or active-control
designs common in institutional food environments; this choice improves
internal validity but may lower external validity for policy-relevant
defaults and operational nudges. Third, all data extraction was
conducted by a single coder; while the authors consulted public datasets
or contacted authors when uncertain, the lack of independent
double-coding and inter-rater reliability checks increases the risk of
extraction or coding errors (especially for complex effect-size
conversions and intervention categorization). Fourth, when sufficient
information to compute a standardized mean difference (SMD) was lacking
and the text reported a ``null,'' outcomes were set to an ``unspecified
null'' of 0.01. This imputation is transparent but ad hoc; it could bias
pooled estimates upward (relative to zero) and may not reflect the true
variance of those effects. The manuscript would benefit from sensitivity
checks setting these to 0, excluding them, or modeling them with
conservative variances. Fifth, the chosen primary effect-size metric is
Glass's \textless U+0394\textgreater{} standardized on the control-group
SD (preferably pre-treatment). While often reasonable, many included
RCTs likely lack pre-treatment consumption SDs, and cross-study
variability in control SDs can induce additional heterogeneity; the
paper does not report sensitivity to using Hedges' g with small-sample
correction or to odds-ratio/logit-based metrics for binary outcomes. The
text notes standard conversions for regression-based and discrete
outcomes, but more detail on the exact formulas, handling of cluster
designs, and any small-sample corrections would improve reproducibility
and comparability. Sixth, the RVE meta-analytic approach is appropriate
for dependent effect sizes, but the assumed within-study correlation
(rho) is not reported; typical practice is to vary rho (e.g., 0.2--0.9)
to show robustness. Subgroup and moderator analyses with very small k
(notably choice architecture, k=2 studies, 3 estimates) produce
extremely imprecise estimates; these are presented with wide CIs, but
readers would benefit from explicit caution against overinterpretation.
Seventh, publication-bias methods (Vevea--Hedges selection model and the
non-affirmative subset approach) generally assume independence of
effects; it is not fully clear whether dependence was addressed (e.g.,
by collapsing to one effect per study for these specific analyses). If
not, standard errors may be anti-conservative. Eighth, while the authors
emphasize that many outcomes are self-reported and susceptible to social
desirability bias, the main results do not stratify by measurement type
(objective vs.~self-report) nor systematically examine follow-up
duration as a moderator, despite extracting ``longest follow-up'' per
intervention; both could influence effect sizes and real-world
relevance. Ninth, cluster-randomized trials were included contingent on
=10 clusters, but the extraction section does not detail whether effect
sizes and standard errors were adjusted for clustering when primary
reports did not do so; misalignment here could affect weighting. Tenth,
the intervention taxonomy (choice architecture, persuasion, psychology,
combined) is reasonable, but many interventions span multiple
categories, complicating subgroup interpretation; the authors
appropriately eschew meta-regression by theory, but this leaves open
questions about differential efficacy by mechanism beyond simple
stratification. On interpretation, the paper's central claim---net MAP
reduction remains an unsolved problem---is well justified by both the
small pooled effects and the bias-corrected estimates, together with the
observed decline in effects over time and stronger effects when focusing
on RPM only (with unresolved substitution risk). The discussion
appropriately notes improved methodological trends, gaps in evaluated
strategies (e.g., pricing, defaults with longer follow-up, contact with
animals, disgust-based messaging), and the need to measure compensatory
behaviors across meals or days. Presentation and logic are clear;
figures and tables support the narrative; minor textual redundancies and
a few typographical repetitions do not impede comprehension.
Open-science practices are a notable strength: a DOI-linked repository,
documented code and data, and containerization substantially lower
replication barriers, though inclusion of a data dictionary for all
coded variables, explicit mapping of each effect-size transformation,
and a machine-readable PRISMA log would further enhance reuse. Overall,
the work persuasively reframes expectations about behavioral
MAP-reduction interventions, carefully distinguishes RPM-focused
outcomes from net MAP, and shows commendable transparency. The main
limitations---single-coder extraction, ad hoc handling of ``unspecified
nulls,'' incomplete reporting of some meta-analytic assumptions, and
limited moderator detail on measurement/follow-up---temper but do not
overturn the conclusions. The paper provides a reliable, policy-relevant
synthesis that is likely to influence both research prioritization and
practice, especially in encouraging larger, longer, and more
behaviorally rigorous trials with objective outcomes and full accounting
for substitution and compensation.

\end{tcolorbox}

\begin{tcolorbox}[enhanced jigsaw, title=\textcolor{quarto-callout-tip-color}{\faLightbulb}\hspace{0.5em}{Model reasoning trace}, titlerule=0mm, colbacktitle=quarto-callout-tip-color!10!white, coltitle=black, opacitybacktitle=0.6, bottomrule=.15mm, left=2mm, toptitle=1mm, colback=white, arc=.35mm, colframe=quarto-callout-tip-color-frame, toprule=.15mm, opacityback=0, bottomtitle=1mm, breakable, rightrule=.15mm, leftrule=.75mm]

\textbf{Analyzing meta-analysis details}

I need to incorporate details from a paper to justify the scores
clearly. This means summarizing issues like internal threats, data
limitations, and measurement methods. Specifically, the meta-analysis
includes 35 papers, 41 studies, and around 87,000 subjects with some
restrictions, focusing on RCTs measuring behavior at least a day
post-intervention. It employs robust variance estimation and utilizes
Glass's Delta for effect size estimation, although using a null of 0.01
seems questionable and is worth critiquing. We should also consider
publication bias and selection models.

\textbf{Evaluating analysis methods}

The authors use a significance funnel plot and categorize interventions
into four theories: choice architecture, persuasion, psychology, and a
combination of persuasion and psychology. They conduct subgroup analyses
and some meta-regressions, avoiding meta-regression for theoretical
approaches. They present pooled estimates by theory and note that effect
sizes decline over decades.

They distinguish RPM from MAP with SMDs of 0.25 and 0.07, respectively,
and apply a robust check with additional studies. They provide open
science resources, including a DOI and mention using Rmarkdown for
ensuring reproducibility.

\textbf{Reviewing supplementary details}

I provided supplementary details and data, including the PRISMA diagram
that shows relevant numbers. They include tables of results and
moderators, but there are some potential issues.

Firstly, their selection of studies is limited to RCTs with at least 25
participants per arm and measurements taken one day later, which could
bias results by excluding certain interventions.

For data extraction, recording unspecified nulls as 0.01 can introduce
imprecision. It's unclear if they properly assigned variances, which
might bias results positively. Finally, I wonder if all studies included
were RCTs with pre-test measurements.

\textbf{Assessing measurement methods}

Some consumption measurements were taken only post-intervention, and
they mention using the ``pre-treatment control group standard
deviation,'' which makes me wonder if they used baseline data. Many RCTs
might not have a baseline measurement, so it's possible they're relying
on the control group's post-measurement SD. This mixing of pre and post
data is a bit ambiguous.

On a positive note, they used robust variance estimation for multi-arm
studies, which is good. However, regarding publication bias, while they
assumed selection based on p-values, I need to check if they adjusted
for dependence when using metafor with cluster robust RVE.

\textbf{Examining selection models and methods}

It's not entirely clear if the selection model requires independent
effects, especially since they have multiple effect sizes per study.
They used robumeta as their main method, but it's not well-explained if
the effect sizes are treated independently, which could introduce bias.

While small-sample corrections for robust variance estimation (RVE) are
good, they included 41 studies, and robumeta needs more than 10
clusters, which works. Their theoretical categories seem plausible, but
pooling effects within categories might lead to double-counting. The
overlap across categories complicates the inference.

They also used Glass's Delta instead of Hedges' g without discussing
corrections for small sample bias.

\textbf{Analyzing effect size methods}

Using Glass's delta for effect sizes can lead to bias and may not be
comparable across studies, especially with high heterogeneity in control
standard deviations. Typically, a meta-analysis uses the standardized
mean difference, Hedges' g, which might be more appropriate since it
accounts for those issues. While Glass's delta could work if
heteroskedasticity differs, it complicates replicability.

Regarding dichotomous outcomes, they calculated the difference in
proportions from Bernoulli draws but didn't specify how they convert
these to standardized effect sizes. They may use formulas for
standardized mean difference, referencing Paluck et al., which is a
decent approach.

\textbf{Critiquing effect size metrics}

For the 2021 difference of proportions with Bernoulli, I see some issues
with mixing effect size metrics like standardized mean difference (SMD)
and log odds ratio. While it's possible, they need to maintain a
consistent scale, and their lack of details could lead to replicability
problems.

They report a main heterogeneity of 0.082 SMD and interpret proportions
of 0.1 and 0.2, which is good. Their classification of ``marginal
studies'' as part of the supplementary dataset leads to a significant
pooled effect of SMD 0.2 (CI 0.09 to 0.31) with a p-value indicating
strong significance.

\textbf{Evaluating study methodologies}

Mixing immediate-outcome studies with non-random or low-powered studies
may introduce upward bias, and I think their sensitivity check might
just reintroduce that bias. Still, they seem careful in their
interpretations.

They acknowledge three decisions made during study inclusion after data
collection began, which could indicate data-dependent PRISMA issues.
Although these post-hoc decisions might bias the meta-analysis, they
appear to be reasonable, such as excluding certain types of
manipulations.

They're extracting an effect size for the outcome with the longest
follow-up, which is good for avoiding multiplicity. However, their
``pure control'' requirement excludes ``active control,'' which could be
a limitation.

Their examination of publication bias has some flaws, particularly with
reliance on naive standard errors for non-affirmative results, but I can
accept this. While the median analyzed sample size is 132, with many
being small, they state that most included RCTs measure consumption
outcomes after at least one day, addressing social desirability bias,
which is a good point.

Lastly, I appreciate their emphasis on open science, offering replicable
code and links. Providing a pre-analysis plan would also be good.

\textbf{Reviewing study details}

They've used R packages, which is good to see, and their references list
is quite extensive. However, I noticed minor text errors, like the
repeated phrase about the unclear effects on different forms of MAP. The
PRISMA figure accurately identifies 41 included studies, which is a
positive aspect.

With 87,000 subjects in the sample, it seems there might be some large
RCTs involved. The introduction mentions normative aspects, and I
appreciate their transparency in discussing limitations.

Now, focusing on the evaluation metrics: they claim that interventions
yield a small effect and provide evidence to support this. Although,
after considering publication bias, the effect is nearly zero. They
justify this well and also address the substitution effect, handling RPM
versus MAP effectively while interpreting heterogeneity, which is
commendable!

\textbf{Assessing methodology and findings}

There are some issues to note, like how they set ``unspecified null'' to
0.01, which could be problematic. Also, the classification of
``persuasion categories'' isn't fully specified; I'd grade it moderate
to high, around 70.

Their methods seem solid, focusing on RCTs and CLS, utilizing RVE, which
is great. However, I have concerns about conversions, the Glass delta,
and selection model independence. I didn't see a mention of a
meta-analysis pre-registration either, which raises questions. I'd rate
the methods around 68 due to these limitations.

In terms of ``logic\_communication,'' the paper is clear and coherent,
presenting tables well, although there are minor typos and duplications.
I would score this section an 85.

Regarding ``open\_science,'' they provide useful elements like code,
data, containerization, and DOIs, enhancing replicability. Overall, it
looks good!

\textbf{Evaluating methodology and relevance}

There are some concerns with effect size extraction, as they only relied
on the first author and didn't have independent double-coding. While
they utilized dynamic methods and provided a PRISMA flowchart, the
details on linking included and excluded studies seemed lacking. They
acknowledged using an ``AI search tool,'' which raises some doubts but
also offered a well-documented dataset. I'd rate this section around 83.

Regarding ``advancing knowledge,'' they address significant issues like
stricter RCT-only meta-analysis and correcting for publication bias,
which is relevant for global priorities such as climate and health, so
I'd score it 78.

Their ``global relevance'' is high, providing actionable insights for
policymakers, although they didn't compute cost-effectiveness. I'll rate
this part 75.

Overall, considering quality and impact, I might set the overall score
around 78, but given recent serious research in behavior change
meta-analyses, perhaps 75 is more appropriate.

\textbf{Estimating publication tiers}

When thinking about publication tiers, I suggest providing credible
intervals, with the lower bound at 65 and the upper bound at 85.

For the normative publication rating, considering it's an
interdisciplinary meta-analysis, I think it could belong to top field
journals in environmental psychology. My guess for the normative tier is
around 3.8 to 4.0, and I might say it's a ``Marginal A-Journal/Top field
journal'' with a score of 4, giving a confidence interval of 3.3 to 4.5.

Now, estimating where it will actually be published is challenging. I'd
lean toward 3.6 with a CI of 3.0 to 4.2, while being cautious to not let
knowledge of existing ratings or authors influence my judgment.

As I craft the 1000-word summary, I want to highlight points of
replicability: they employed robumeta, a widely-cited method,
single-coded data, and standardized effect sizes. They used a mixture of
interventions and the longest follow-up, showing some heterogeneity, but
nothing overwhelmingly significant.

\textbf{Identifying potential issues}

There are a few potential issues I see regarding the analysis. The use
of RVE for dependent effect sizes is fine, but robumeta employs
degrees-of-freedom adjustments. With 41 clusters, it appears acceptable,
yet the subset analyses with small sample sizes, like choice
architecture with n=2, might lead to unstable small-sample corrections
and extremely wide confidence intervals.

Additionally, using ``Glass's delta with control SD pre-treatment''
could create inconsistencies across studies, potentially skewing
significance, although the standardized mean difference (SMD) of 0.07
remains robust.

Setting the ``unspecified null'' to 0.01 raises concerns since it might
inadvertently bias results upward by treating minimal effects as
positive, which is problematic. I'm also curious about their handling of
cluster RCTs and whether they adjusted for the intracluster correlation
coefficient (ICC).

\textbf{Clarifying effect size computation}

The report doesn't clearly describe the effect size computations from
group-level results, and the use of reported standard errors raises
questions. Since robust variance estimation in meta-analysis operates at
the effect size level, the computation of standard error should reflect
the cluster-level data. If they utilized published results without a
clustering adjustment, this could lead to inaccuracies. They did include
a cluster-assigned sample with at least 10 clusters, but the details are
sparse, so it's worth flagging.

Additionally, they've included some inexpensive ``nonprofit white
papers,'' which may limit the quality. While they list their search
steps, the actual search terms and chronology aren't provided, which
affects replicability. On a positive note, they did supply accompanying
code and a data repository for full documentation.

\textbf{Examining publication bias and follow-up length}

The study addresses publication bias by implementing selection models,
which is good, but these models depend on having knowledge of effect
sizes and standard errors. I notice that the robust variance approach
isn't fully integrated with these models; however, using them as a
separate check seems reasonable.

Another point is the mention of ``longest follow-up time after
intervention.'' If there are significant variations in follow-up
durations, this could introduce bias, as they didn't adjust for the
differing lengths. Presenting the overall effect without considering
this heterogeneity might impact comparability.

\textbf{Reviewing study methodology and details}

They could have conducted a meta-regression based on follow-up length,
but it seems they didn't include this in their analysis, even though it
may be mentioned in the dataset.

I see that they restricted their focus to measurements taken on the
``first day after treatment begins,'' and while they noted that
``immediate measurement at a single lunch event'' might be excluded,
this approach seems sound.

They've chosen to measure only one outcome per intervention to minimize
multiplicity, which is a good decision. Evaluating ``evidence of
substitution to other forms of MAP'' not being measured is also
reasonable.

The note about the ``lack of long-term follow-ups'' is valid; many
studies only measured outcomes over a matter of weeks or days. Their use
of dynamic normative signage is mentioned as an illusion, which is good
to acknowledge.

They also clearly disclaim that they did not treat PRISMA classification
as unnatural, which is an important clarification. Lastly, the mention
of ``eight disclaimers seldom expedite interplay'' stands out; I'll need
to make sure we produce about 1000 words, ensuring the final JSON
includes an ``assessment\_summary'' that fits the 900-1100 word range.

\textbf{Drafting assessment summary}

I'm starting with a concise paragraph that identifies key issues,
summarizing contributions but emphasizing specific concerns: dataset
selection, risk of bias, effect size conversions, and the ``unspecified
null'' fudge. There's also the mixture of outcomes, the need for cluster
adjustment, and independence assumptions in selection models.

While their approach seems robust regarding RVE, there's no clear
pre-registration, and replicability is an important resource, especially
since they have good open science practices like code containerization.
I'll also note that replicability could be improved with two-coder
extraction and alignment with MOOSE/PRISMA standards.

They mentioned using PRISMA, which is good, but it doesn't appear to
have been pre-registered. Although the user didn't specifically ask for
suggestions, the ``assessment\_summary'' could include some proposed
improvements, such as pre-registering the meta-analysis, employing a
3-level meta-analysis, using robust synergy, applying Tipton
corrections, and including small-sample corrections while noting
potential illusions.

\textbf{Calibrating manuscript assessment}

The instructions are clear: I need to provide a compact paragraph that
identifies important issues in the manuscript while being specific,
neutral, and concrete. It's essential to highlight both the issues and
merits, and I might include some suggestions too. But a paragraph with
1000 words feels quite long---although I suppose it's manageable.

Now, I need to compute metrics by calibrating to a percentile ranking
based on serious research within the last three years. This area focuses
on meta-analyses of behavioral interventions aimed at reducing meat
consumption. Among those, this meta-analysis appears rigorous, so I'll
rank it quite high. However, I want to avoid making extreme leaps in my
ranking, so I think an overall score of 83 makes sense, but I might
settle on 78 after further calibration.

\textbf{Calibrating contour scores}

I'm noticing that the methods in the research are strong, and the
commitment to open science is impressive. Overall, I'd consider a score
of 82, but I think I'll keep it at 80, with a confidence interval of
72-88. For the claims, I'm thinking maybe a score of 78 with a range of
70-85 seems fitting.

The methods might score around 76 (68-83) and the advancing knowledge
piece at 79 (70-87). For logic and communication, I'm leaning towards 86
(80-92) and open science at 90 (82-97). The global relevance could be 85
(78-92).

I want to ensure I'm consistent with my earlier numbers. They found an
effect near zero, which is significant. I believe the meta-analysis
executes well, so I'm considering moving open science up to 88 due to
containerization and dataset.

I need to remember that the upper bound can't exceed 100, and the lower
bound should be less than the midpoint, which has me thinking about
calibrating the ``tier\_should'' and ``tier\_will.'' Given the high
quality, perhaps the normative ``should'' score is between 4.2 to 4.6,
while ``will'' may range from 3.8 to 4.2. But, I must stick to the
instruction not to consider where it's published.

\textbf{Finalizing numeric values}

I need to propose some numeric values along with their confidence
intervals. For the tier\_should score, I'm considering 4.3, with a lower
CI of 3.6 and an upper CI of 4.8. For the tier\_will score, I think 3.9
would work, with a lower CI of 3.2 and an upper CI of 4.5.

Now, it's time to craft the summary. It should discuss the search
strategy, which includes a multi-step process covering 156 prior
reviews, but only nine were included. They used Google Scholar,
crowd-sourcing, an AI tool, grey literature, and databases from
nonprofits, while also noting the PRISMA figure, which is good.

For data extraction, I'll mention that they focused on one effect per
intervention at the longest follow-up; there's also the issue of the
``unspecified null'' fudge and the use of conversion formulas. Many
results are self-reported, and the risk of bias wasn't formally
assessed.

Regarding the analysis, they used Robust Variance Estimation (RVE) with
robumeta, incorporating small-sample adjustments that are robust to
dependencies. I'm curious if they specified their handling of
``within-study correlation,'' which is typically set at about 0.8, and
whether they used ``rho'' by default.

\textbf{Evaluating methodology and sensitivity}

I noticed that robumeta RVE requires users to specify the rho value,
which defaults to 0.8, but they don't report any sensitivity to it.
That's a concern, as variations between 0.2 to 0.9 rarely change the
results, and they should definitely show some sensitivity analysis. I'll
critique this as it's an important issue regarding replicability.

Now, regarding their analysis, they treat the ``difference in
proportions estimator,'' which is good, and they use ``Glass's
\textless U+0394\textgreater{}'' formula. However, that might not apply
to binary outcomes, so they convert these separately.

While they convert effect sizes from regression well, I've also noticed
they utilize p \textless{} 0.05 with an aggregator. I see they mention
``two choice architecture,'' which seems overshadowed, while ``longest
follow-up'' also casts some illusions.

I appreciate their use of ``one day after treatment begins'' for RCTs,
but there's a lack of cost-effectiveness discussion or generalizable
translation to policy. Additionally, they don't specify classification
for theoretical categories, but overall, the quality appears high.

I'll make sure to keep a neutral tone in my summary while producing the
JSON output.

\textbf{Detailing methodological issues}

First, I need to highlight some issues in the manuscript. There might be
potential selection bias due to restrictions, like requiring a pure
control group, which could exclude significant policy changes.
Additionally, post hoc inclusion decisions after data collection began
may pose problems, particularly regarding RPM and forced-intake
restrictions.

They also have single-coder extraction, which increases the risk of
errors; double-coding was absent. The ``unspecified null = 0.01''
appears unnatural and could introduce upward bias without variance
imputation. Using Glass's \textless U+0394\textgreater{} alongside SMD
may mix results without proper Hedges g correction, hampering
comparability. Testing sensitivity to Hedges g or log OR is important.

Furthermore, while they used RVE, robumeta needs an assumed within-study
effect correlation rho that they didn't report, along with no
sensitivity analysis described. They addressed publication bias with
selection models, but independence assumptions may not hold.

Regarding heterogeneity, they reported t and assessed true effects but
assumed a normal distribution. They noted many outcomes were
self-reported but didn't meta-regress effects by measurement type, which
could lead to bias.

Lastly, they provided a dataset and a DOI link, which is fantastic for
replicability and meets open science standards!

\textbf{Finalizing metrics and scores}

I'm working on finalizing the metrics for the assessment. Here are the
numbers I've chosen:

For the overall score, I'm thinking a midpoint of 82, with a lower bound
of 74 and an upper bound of 89. Claims evidence could sit at a midpoint
of 80, with a range of 72 to 87. For methods, a midpoint of 75 feels
right, while advancing knowledge can be 80. Logic and communication
could be high at 87, and open science can sit at 90. Global relevance
should be at 85.

For tier should, I'm suggesting 4.3, and tier will at 3.9. However,
after considering the moderate-high ``methods'' score, I feel
``overall'' might be better set at 78 to maintain balance, especially
since ``open science'' and ``logic'' are strong.

I'll return a structured JSON, making sure it aligns with the ``applied
and policy stream.''

\textbf{Finalizing JSON and Summary}

I'm putting together the final JSON for the assessment. I want to make
sure it adheres to the format ``paper\_assessment\_with\_tiers\_v2''.

First, I'll confirm the provided metrics: they include numerical values
for ``overall,'' ``claims\_evidence,'' ``methods,''
``advancing\_knowledge,'' ``logic\_communication,'' ``open\_science,''
and ``global\_relevance.''

Next, I need to include ``tier\_should'' and ``tier\_will,'' ensuring
the top-level object has only two keys. I must be cautious that all
values remain numbers, not strings, and they satisfy numeric
constraints.

In the summary, I'll also mention the importance of replicability,
noting the lack of pre-registration. Then, I'll create the final JSON!

\end{tcolorbox}

ewpage

\section{Peterman et al.~2025}\label{peterman-et-al.-2025}

\begin{tcolorbox}[enhanced jigsaw, title=\textcolor{quarto-callout-note-color}{\faInfo}\hspace{0.5em}{Model assessment summary}, titlerule=0mm, colbacktitle=quarto-callout-note-color!10!white, coltitle=black, opacitybacktitle=0.6, bottomrule=.15mm, left=2mm, toptitle=1mm, colback=white, arc=.35mm, colframe=quarto-callout-note-color-frame, toprule=.15mm, opacityback=0, bottomtitle=1mm, breakable, rightrule=.15mm, leftrule=.75mm]

This is a carefully executed and policy-relevant systematic review and
meta-analysis synthesizing 1,307 effect sizes from 93 experimental
studies (115 publications) on social safety nets (SSNs) and women's
economic achievements and agency in LMICs. The work follows a registered
protocol (PROSPERO) and PRISMA, uses robust variance estimation (RVE)
for dependent effects with small-sample adjustments, triangulates with a
multilevel random-effects model, and provides extensive descriptive
characterization of studies, interventions, and outcomes. However,
several important issues warrant attention. First, heterogeneity is very
high (I2 mostly 80--95\%, tau2 \textgreater{} 0.02), yet pooled averages
are emphasized as ``robust'' without providing prediction intervals;
this means that while the mean effect is precisely estimated, the
expected range of effects in a new setting is likely wide and often
uncertain, which matters for external validity and for decision-makers
comparing modalities and contexts. Relatedly, meta-regressions explain
little of the heterogeneity and have limited degrees of freedom (df just
above 4 threshold in some cases), making moderator inferences fragile;
the paper acknowledges low power and heterogeneity but could more
explicitly temper generalizations about modality differences (e.g., UCT
\textgreater{} CCT) and the practical significance of pooled estimates.
Second, outcome harmonization raises comparability concerns. The review
collapses conceptually diverse constructs (e.g., labor force
participation, savings, assets, expenditures; voice, autonomy,
decision-making) into standardized mean differences (Hedges' g). Many
outcomes are binary (e.g., LFP), some are indices, and others
continuous; the paper does not describe in the main text how binary
outcomes and indices were transformed to SMDs, nor how differing
variances and measurement properties across constructs were handled.
Without clear transformation formulas and sensitivity to alternative
metrics (e.g., log odds ratios for binaries, standardized indices only),
comparability and interpretability of pooled Hedges' g across such
heterogeneous measures is limited. Third, dependence and weighting
deserve more transparency. While RVE addresses within-study dependence,
the manuscript does not detail the assumed correlation structure or
\textless U+03C1\textgreater{} choice used for weights (and whether
results are sensitive to \textless U+03C1\textgreater), nor quantify the
extent to which large multi-arm or multi-outcome studies contribute
disproportionate leverage. The Cook's D influence check is helpful, but
the criterion for flagging and the effect on pooled estimates should be
documented quantitatively. Fourth, there are minor internal
inconsistencies/typos that need correction: public works' pooled effect
is reported as 0.127 in one place and 0.273 elsewhere; a decision-making
CI appears as {[}0.036--0.37{]}, which likely omits a leading zero
(0.137). Such inconsistencies, while not undermining the core
conclusions, should be corrected for clarity and credibility. Fifth, the
paper includes only experimental (RCT) evidence published 2003--2024 and
searched in English/French/Spanish, but ultimately notes that no French
or Spanish studies were included. It is unclear whether this reflects a
true paucity or limitations in search strategy/screening; the exclusion
of high-quality quasi-experimental studies and the effective
English-only inclusion can constrain generalizability, particularly for
Latin America where CCTs were historically prominent. Sixth,
risk-of-bias/quality appraisal is not transparently summarized in the
main text. The meta-regressions include an indicator for ``quality
assurance (medium quality)'' with little explanation, and there is no
synthesis of randomization integrity, attrition, spillovers, or outcome
measurement bias across studies; readers would benefit from a clear
figure/table summarizing risk-of-bias domains and any sensitivity
analyses that down-weight or exclude high-risk studies. Seventh,
moderator coding could be more granular for design features often
posited to affect women's outcomes (e.g., payment to whom; transfer
frequency; delivery modality; conditionality enforcement intensity;
childcare or reduction-of-time-cost components; norms context;
plus-component content and dosage). The paper codes ``plus'' components
broadly and finds limited moderator effects, but heterogeneity within
categories (especially ``training or information'' and
``gender-sensitive plus'') is substantial; more detailed taxonomy or
hierarchical models distinguishing content, intensity, and timing would
have more power to detect design--effect relationships. Eighth, the
cost--benefit section is informative but methodologically thin for
meta-analytic purposes. Measures (BCR, IRR, cost-effectiveness) vary in
perspective, time horizon, discount rate, benefit definitions, and
inclusion/exclusion of women-specific benefits; there is no
standardization or meta-analytic synthesis, and comparability is
limited. The claim that reported BCRs are ``likely lower bounds''
because women's benefits are undercounted is plausible but remains
speculative without re-estimation under common assumptions; a structured
re-calculation for a subset with microdata would strengthen this
section. Ninth, the orientation of ``improvements'' to ensure positive
signs is sensible, but more detail is needed for outcomes where
improvements imply decreases (e.g., debt, care-work intensity);
explicitly stating for each indicator whether a decrease was coded as
positive would help replication. Tenth, time dynamics are underexplored:
intervention duration and follow-up are dichotomized at 12 months, but
impacts of SSNs plausibly evolve; modeling time since last payment,
cumulative value received, or longer-term follow-up with continuous
functions or splines could be more informative. Eleventh, publication
bias assessment is only referenced as ``little evidence,'' with details
presumably in the supplement; funnel plots, small-study tests adapted
for dependent effects, and p-curve/p-uniform analyses could be
summarized in the main text to reassure readers. Twelfth, outcome
selection could be subject to within-study reporting biases (e.g.,
studies more likely to report gendered outcomes when expecting positive
effects); this is mentioned but not probed---pre-specification checks
against original protocols or balance tests of outcome families would
help. Finally, while the paper's conclusions are appropriately cautious
in places, some policy implications (e.g., advocating unconditionality
over conditionality) could be more explicitly framed as suggestive given
unresolved confounding of modality with geography, age of programs,
enforcement intensity, and measurement choices; the authors note these
limitations but the abstract and headline statements might overstate
certainty. On transparency and reproducibility, replication files are
said to be available as supplementary materials and to be deposited
publicly upon acceptance; the data extraction form is not public, which
is a notable barrier to full reproducibility because effect-level coding
decisions (outcome mapping, sign conventions, index composition) are
central to the results. Overall, this is a timely, thorough, and careful
synthesis with clearly positive contributions; addressing the above
issues---especially clarifying outcome transformations, correcting
inconsistencies, adding prediction intervals, summarizing risk-of-bias,
and strengthening the moderator taxonomy---would materially enhance
credibility and practical usefulness for policy design.

\end{tcolorbox}

\begin{tcolorbox}[enhanced jigsaw, title=\textcolor{quarto-callout-tip-color}{\faLightbulb}\hspace{0.5em}{Model reasoning trace}, titlerule=0mm, colbacktitle=quarto-callout-tip-color!10!white, coltitle=black, opacitybacktitle=0.6, bottomrule=.15mm, left=2mm, toptitle=1mm, colback=white, arc=.35mm, colframe=quarto-callout-tip-color-frame, toprule=.15mm, opacityback=0, bottomtitle=1mm, breakable, rightrule=.15mm, leftrule=.75mm]

\textbf{Evaluating RVE Methodology}

I'm thinking about how they used RVE to handle dependent estimates, but
it seems like they still double-count tests. They possibly took a
``study-level cluster'' approach that may weigh each effect size by the
number of effects per cluster. It concerns me that if some studies have
many effect sizes, they could dominate the results. They didn't include
risk of bias ratings in the main text and seem to rely on a quality
assessment score, which feels incomplete.

\textbf{Exploring Publication Bias and Cost-Benefit Analysis}

I'm considering how they approached ``publication bias.'' They mention
``little evidence'' but don't provide a funnel plot, which makes me
wonder about the replicability since they could have used Egger or
p-curve methods. Their qualitative summary of cost-benefit reveals that
few women's outcomes were included, and while they conclude BCRs are
generally positive, it's only a plausible assertion without thorough
evaluation. There are also issues like the lack of standardization
across currency and time, which limits quantitative comparability. Plus,
they need to clarify the mapping of ``agency'' constructs used.

\textbf{Evaluating RCT Inclusion and Statistical Details}

I'm analyzing how they included only RCTs from ``published and grey
literature'' between 2003 and December 2024, but there are likely
earlier RCTs of CCTs in Latin America. They interpreted the difference
between UCT and CCT as conditionality restricting women's choices, but
they found smaller effect sizes for CCTs, which raises confounding
concerns. They mentioned using correlated-effects RVE, but I'm puzzled
about whether they specified the \textless U+03C1\textgreater{}
parameter for the assumed correlation, which is crucial. It feels like
that detail is missing.

\textbf{Analyzing Weighting and Outcome Measures}

I'm assessing their approach in robumeta, particularly how they use
\textless U+03C1\textgreater{} to compute weights. There's some concern
about variations in \textless U+03C1\textgreater{} since they didn't
address sensitivity, which I think should be flagged. For dichotomous
outcomes, they used standardized mean difference (SMD), but I'm left
wondering if they converted odds ratios (OR) to SMD because that wasn't
mentioned. For outcomes like labor force participation, it seems they
could have used log risk ratios or log ORs. These transformations are a
bit unclear, limiting replicability.

\textbf{Identifying Inconsistencies and Errors}

I'm noticing some internal inconsistencies regarding p-value thresholds
and credible significance. They mention a ``pooled coefficient on public
works'' of g 0.273 at p\textless0.05, but earlier it was listed as
0.127---so that seems off. Additionally, their report on
``decision-making hedges'' shows g 0.087 with a 95\% CI that indicates
0.37, but I suspect it should actually be 0.137 due to typographical
errors. Lastly, they define ``care work participation'' as having
insufficient power, but there's a 1\% effect level in Panel C.

\textbf{Considering Statistical Methods and Replicability Issues}

I'm looking at how they handle the data, mentioning ``N=1,307 effects
across 93 studies,'' but they use a low degrees-of-freedom threshold of
\textgreater4 for the robust variance estimation. This could lead to
unstable significance results, although they seem to correct for that.
They mention trimming influential effects but don't specify the
threshold used. I find it concerning that details in the Supplementary
Material aren't accessible, making replicability weak. They also note a
need for further research on intensity and dosage in their domain
classification.

\textbf{Assessing Evidence and Statistical Concerns}

I'm looking into their claims about ``care work participation'' and
``leadership'' but they say there's a lack of evidence for calculations.
They mentioned searching for studies in French and Spanish, but I'm
wondering if that search was thorough enough. There's a note on
confidence intervals and robust methods, yet they didn't compute
prediction intervals for expected effects, which seems necessary given
their high heterogeneity. They mention ``p-hacking'' but say it's not
relevant, and there's some confusion regarding Egger's test for
publication bias. Lastly, they seem to have introduced ``units'' like
Hedges g without thorough explanations, which makes assigning
quantitative metrics tricky.

\textbf{Reviewing Systematic Analysis and Findings}

I think it's fair to rate this systematic review and RVE meta-analysis
in the mid-percentile range; they did a solid job covering 115
publications across 93 RCTs. The effect sizes are small but robust, and
while the heterogeneity is high, their writing is careful and logical.
It's noteworthy that this is the first meta-analysis aggregating across
SSN modalities. Their open science efforts are moderate, as they provide
replication files but won't deposit them publicly until acceptance. They
also indicate that the data extraction form isn't available right now.
Overall, I'd rate it around the 82nd percentile.

\textbf{Summarizing Ratings and Observations}

I'm thinking about my overall rating for this analysis, which seems to
be around 82.

For methods, they did a thorough job using robust variance estimation
(RVE), but there are some missing details, resulting in a score of 78.

The advancing knowledge is solid---providing broad evidence at an 85.
Their logic and communication are good, albeit with a few typos, rating
around 80.

In terms of open science, while they provided some replication code, the
data extraction isn't public, so I'd rate that at 70.

Finally, the global relevance is high at 90, considering the focus on
women's empowerment and social safety nets. Overall, I'd say a midpoint
score of 84 feels right, but I might round it up to 85!

\textbf{Crafting the Ratings Summary}

I'm considering aligning my scores with previous assessments. Overall,
I'd set the score at 84, with a lower bound of 76 and an upper of 92.

For claims evidence, I'd give an 83, with a range of 75-90.

Methods would be 78, ranging from 68-86. Advancing knowledge gets an 86,
with a 78-93 range.

The logic and communication would score an 81, from 72-89.

Open science might drop to 67, with a range of 55-78 since it mentions
replication but hasn't deposited the files.

For global relevance, I'd aim for 91, slightly up from 90, so with a
range of 85-96.

Lastly, I'd consider Tiers, assigning `should' a score of 4.3 and `will'
a 4.1. I need to ensure that all numbers fit the criteria of lower
bounds being less than the midpoint, which is crucial for fleshing out
my 1000-word assessment summary.

\textbf{Summarizing Key Issues}

I'm aiming for about 900-1100 words in a compact summary that covers
major concerns.

First, the scope is limited to RCTs, and there are language
restrictions, with no studies found in French or Spanish, affecting
generalizability.

Regarding outcomes, there are issues with standardization and mixing
constructs, relying on standardized mean difference (SMD) which raises
comparability concerns.

Heterogeneity is extremely high, with insufficient moderators explained,
and they don't compute prediction intervals, although this is discussed.

Additionally, the within-study dependence should be clearer, especially
regarding \textless U+03C1\textgreater{} sensitivity and weighting. Data
extraction is not public, yet they promise replicability by deposit.

I noticed the assessment of publication bias was not shown, which is a
concern. Typos and inconsistencies exist in their data, like
discrepancies in public work estimates.

The cost-benefit summary lacks comparability and fails to include
intangible benefits. The risk of bias was not thoroughly presented,
highlighting inconsistent quality indicators.

Lastly, there's potential ambiguity surrounding the analysis of economic
improvements versus debt reduction, and the time post-intervention
classification could be refined. The heavy reliance on p-values for
significance warrants a closer look.

\textbf{Drafting the Assessment Summary}

I'll need to carefully craft the assessment summary as a JSON format
that includes `assessment\_summary', `metrics', and nested float
numbers. The guidelines say to avoid heavy formatting and keep it in
plain text, while also allowing for bullet lists if needed. We're aiming
for around 900-1100 words, even though the request was for a more
compact paragraph.

Looking back at their presentation, they outline their PRISMA diagram:
5,120 hits, 1,215 duplicates, 3,905 screened, 3,568 excluded, 345
assessed, 230 excluded, ultimately including 115 publications and 93
studies---resulting in 1,307 effects.

We should highlight the details in Tables S6 to S8 and note the absence
of mosaic representation. It's also important to mention the lack of
pre-registration aside from their PROSPERO registration with ID \#,
which is a positive point.

Regarding primary results, they report a pooled g of 0.107, indicating
variation across modalities that continues to affirm their analysis.
Let's put this together as we draft the summary!

\textbf{Crafting the Assessment Summary}

I need to ensure we highlight that ``N is obviously high.'' It's crucial
to address potential data illusions, especially regarding ``confidence
intervals.'' Now, let's set the metrics and make sure to match the JSON
structure in ``paper\_assessment\_with\_tiers\_v2.''

I'll create the JSON format like this:

\{ ``assessment\_summary'': '' \ldots{} text \ldots{} ``,''metrics'': \{
``overall'': \{``midpoint'': 84, ``lower\_bound'': 76, ``upper\_bound'':
92\}, ``tier\_should'': \{``score'': 4.3, ``ci\_lower'': 3.7,
``ci\_upper'': 4.8\}, ``tier\_will'': \{``score'': 4.1, ``ci\_lower'':
3.3, ``ci\_upper'': 4.7\} \} \}

It's important that this JSON structure strictly adheres to the schema,
ensuring ``tier\_should'' and ``tier\_will'' are appropriately nested
under metrics. I'll include all necessary properties, making sure
numeric fields are floats and credible intervals are present for
everything. Now, I'm ready to draft the actual ``assessment\_summary,''
incorporating both strengths and criticisms of the paper. Let's get to
it!

\textbf{Identifying Key Issues in the Manuscript}

I need to focus on not just general ``issues'' but the most important
ones identified in the manuscript itself. The instructions are clear: I
should specify these issues in a neutral and concrete manner, aiming for
about 1000 words. I'll touch on intangible concerns, like the ``lack of
multiple definitions'' and mention gaps in definitions.

To start my summary, I'll summarize that this is a systematic review
examining social safety nets (SSNs) related to women's economic
achievements and agency in low- and middle-income countries (LMICs). It
covers data from 93 RCTs and revolves around RVE pooling with 1,307
effects. They report small positive effects, with variation across
modalities, and confirm robustness to multi-level models.

However, there are major issues: high heterogeneity, limited
specification, mixing of measurements, risk of bias, replicability
concerns, potential publication bias, and challenges with effect size
comparability and classification. There's also a significant ``lack of
evidence on some aspects.'' I need to include that they claim compliance
with PROSPERO, indicating the study is pre-registered, which is
important actually. Let's go ahead and craft this summary!

\end{tcolorbox}

ewpage

\section{Williams et al.~2024}\label{williams-et-al.-2024}

\begin{tcolorbox}[enhanced jigsaw, title=\textcolor{quarto-callout-note-color}{\faInfo}\hspace{0.5em}{Model assessment summary}, titlerule=0mm, colbacktitle=quarto-callout-note-color!10!white, coltitle=black, opacitybacktitle=0.6, bottomrule=.15mm, left=2mm, toptitle=1mm, colback=white, arc=.35mm, colframe=quarto-callout-note-color-frame, toprule=.15mm, opacityback=0, bottomtitle=1mm, breakable, rightrule=.15mm, leftrule=.75mm]

This study assembles a pantropical, 30 m resolution model of the
potential for natural forest regeneration (NR) on deforested lands
within tropical forest biomes, trained on locations of observed natural
regrowth (2000--2016) distinguished from plantations, and predicts a
continuous probability of NR that is translated to expected area and
carbon accumulation potential. The work is conceptually strong and
practically useful, but several methodological and interpretive issues
temper confidence in some quantitative claims. The dependent variable
(NR presence) comes from a prior mapping product that the authors note
has substantial omission error in humid biomes at the area-estimation
level (producer's accuracy \textasciitilde18.7\% when area-weighted,
despite high user's accuracy for detected patches). Because the model is
trained on mapped occurrences, systematic omission of true NR areas---in
regions where NR is common---can bias both model fitting and spatial
extrapolation (e.g., underweighting conditions conducive to NR in wet
tropics), potentially distorting variable effects and spatial
predictions. The authors emphasize mapped accuracy and argue the
distributional estimates are robust, but they do not propagate this
training-data uncertainty into their area and carbon estimates.

The authors fit random forest models on \textasciitilde6 million points,
balanced between NR and non-NR, and ultimately predict with a
biophysical-only model after showing little accuracy difference from
models including socioeconomic predictors. This choice aligns with
mapping ``biophysical potential,'' but it conflates potential with
likelihood under human pressures. Excluding socioeconomic covariates
increases risk that predictions will overestimate practically achievable
NR in landscapes with high clearing pressure, tenure insecurity, or
near-term market drivers. The paper is careful to frame results as
potential rather than realized outcomes and discusses leakage,
additionality, and persistence, but these distinctions are not reflected
in the headline global area and carbon numbers.

Validation and uncertainty quantification warrant caution. Reported
global accuracy is 87.9\% using a large, randomly sampled,
class-balanced validation set. The authors attempt to examine spatial
autocorrelation by stratifying validation by distance from training
data; accuracy dips to \textasciitilde81\% at 2--2.5 km then rises, with
lower accuracies in Southeast Asia. However, the main accuracy figure
still derives from spatially random validation that is known to be
optimistic for spatial ecological models. A proper spatial block
cross-validation, or geographically structured k-fold with held-out
regions, would likely yield lower accuracy and provide more realistic
uncertainty for end users. The model outputs are treated as
probabilities, and the expected area is computed by summing pixel area
multiplied by predicted probability. Because random-forest
``probabilities'' learned from class-balanced data are generally not
calibrated to true prevalence, the interpretation of these scores as
probabilities is questionable. No calibration (e.g., Platt/isotonic
scaling on an unbiased validation set) is reported, and prevalence
correction is not addressed. This undermines the probabilistic
interpretation and the expected-area calculation. The discrepancy
between weighted expected area (215 Mha) and the
\textgreater0.5-threshold area (263 Mha) highlights sensitivity to how
probabilities are used; yet the study presents an unrealistically tight
``CI'' for the 215 Mha (214.78--215.22 Mha) that appears to reflect only
numerical aggregation, not genuine model or data uncertainty. Similarly,
the 23.4 Gt C (21.1--25.7 Gt) carbon range seems to derive from carbon
accumulation layers, not from uncertainty in the NR mapping and
modeling, and therefore understates true uncertainty substantially.

Predictors and resolution mismatches are transparently described. Key
drivers---distance to forest, local forest density, soils, bioclimatic
axes---behave as expected and agree with ecological understanding.
Nevertheless, the prediction grid is 30 m while several inputs are 300 m
to 1 km or coarser, so many pixel-level differences reflect coarse
inputs; the authors acknowledge this. The decision to use 2018 tree
cover and 2015 land cover to represent ``present (2015) and near-future
(2030)'' while assuming overall conditions from 2000--2016 creates some
temporal inconsistency; no explicit 2030 scenario for climate or
socioeconomic change is modeled, so ``2030'' should be interpreted as a
static extrapolation under recent conditions rather than a forecast.

The study domain is defined liberally: within forest biomes (±25°) the
non-NR area includes croplands, pastures, grasslands, shrublands, etc.,
excluding water/urban/bare and current forest. This choice reduces
subjective screening but risks including some areas that are
ecologically marginal for closed-forest recovery; the model is asked to
down-weight them. The authors excluded explicitly non-forest biomes,
which limits risks of afforestation of true savannas, yet biome
boundaries are coarse and the ESA CCI class uncertainties are
nontrivial. Results show strong concentration of potential near existing
forests, which is ecologically plausible and policy-relevant.

The carbon accounting combines predicted NR potential with published 1
km carbon accumulation rates over 30 years, downscaled to 30 m and
aggregated. Important limitations are acknowledged: belowground biomass
is omitted (noted), permanence and additionality are uncertain, and
leakage can erode gains. However, no attempt is made to adjust area or
carbon for mapped model error, or to provide scenario ranges reflecting
plausible policy, price, or disturbance regimes (fire/drought).
Consequently, the quantitative claims about carbon are best read as
biophysical maxima conditional on establishment and persistence, not as
realizable mitigation potential.

On reproducibility, the underlying datasets are public and the 30 m
probability rasters are openly shared via Zenodo as tiled products,
which is valuable for practitioners. However, the analysis code is
available only on request. Given the complexity of data preprocessing,
variable selection, spatial sampling, and model fitting, public code and
a full computational environment would materially improve
reproducibility and enable independent re-estimation and sensitivity
analysis.

Communication is generally clear. Figures effectively convey spatial
patterns and key partial dependencies; the paper repeatedly cautions
that maps indicate potential and that local decision-making, equity, and
socioeconomics matter. Nonetheless, some statements risk
over-interpretation: the use of ``probability'' for random forest scores
learned under balanced sampling, the very narrow ``confidence
intervals'' for area, and the lack of propagated uncertainty in carbon
estimates could mislead nontechnical readers. The Discussion
thoughtfully addresses persistence risks, perverse incentives,
certification gaps, and equity, which enhances practical relevance.

Priority methodological improvements that would strengthen the work
include: (1) spatially explicit cross-validation with geographic blocks
and reporting of region-specific performance; (2) probability
calibration and prevalence correction, with alternative area estimates
based on calibrated probabilities and multiple thresholds; (3) explicit
uncertainty propagation from training data error, model error, and
predictor uncertainty into area and carbon estimates; (4) side-by-side
predictions with and without socioeconomic variables to bound
``biophysical potential'' versus ``near-term likelihood''; (5)
area-of-applicability diagnostics to flag extrapolations; and (6) full
public release of code and workflows. Even with these caveats, the
dataset fills a significant gap for restoration planning by mapping
where natural regeneration is most biophysically favored at fine spatial
resolution and by linking this to indicative carbon potential.
Practitioners should treat the numbers as upper bounds and combine them
with local data on land tenure, pressures, and costs when setting
targets or designing incentives.

\end{tcolorbox}

\begin{tcolorbox}[enhanced jigsaw, title=\textcolor{quarto-callout-tip-color}{\faLightbulb}\hspace{0.5em}{Model reasoning trace}, titlerule=0mm, colbacktitle=quarto-callout-tip-color!10!white, coltitle=black, opacitybacktitle=0.6, bottomrule=.15mm, left=2mm, toptitle=1mm, colback=white, arc=.35mm, colframe=quarto-callout-tip-color-frame, toprule=.15mm, opacityback=0, bottomtitle=1mm, breakable, rightrule=.15mm, leftrule=.75mm]

\textbf{Analyzing regeneration potential}

I'm exploring the potential for natural regeneration in deforested
tropical areas using machine learning. The study employs random forest
models with biophysical and socio-economic variables and decides on a
biophysical-only model due to similar accuracy results. Training
involved 6 million random points, balancing between regenerated and
non-regenerated areas, aiming to predict potential across the tropics at
a 30m scale. They've quantified an area of 215 million hectares for
regeneration and performed sensitivity analysis, noting some
calculations produce larger areas than expected.

\textbf{Clarifying regeneration thresholds}

I'm looking into the discrepancy between the weighted area and expected
area for regeneration. The threshold area is reported as 263 million
hectares, while the expected area is 215 million hectares. It's
interesting to note that the threshold can exceed the expected area,
especially when the average probability for those pixels is below one.
They estimate above-ground carbon sequestration potential at 23.4 Gt C
over 30 years but don't include below-ground estimates. Additionally,
they address risks and uncertainties in their methodology and provide an
open dataset for predictions.

\textbf{Reviewing accuracy details}

I'm examining the accuracy findings of the model, which shows an 87.9\%
validation with an independent set of 4.87 million balanced random
points. There's an autocorrelation effect where accuracy is high near
training points but drops to 81.4\% at a distance of 2-2.5 km. They note
differences in mapping accuracy, especially across Southeastern Asia,
and provide an out-of-bag accuracy of 87.8\%. Socioeconomic variables
were excluded for predictions, opting for biophysical variables instead,
citing their stability and resolution. The final model utilizes ten
biophysical variables that appear significant, like forest density and
distance to the forest.

\textbf{Examining variable selection}

I'm looking into the variable selection process, which indicates that
adding variables improved model accuracy up to a limit of ten variables.
However, there are concerns regarding the dependent variable training
dataset from Fagan 2022. It seems that this dataset omitted many natural
regrowth patches in humid biomes due to conservative mapping methods,
leading to underrepresentation that could bias the model results. This
is particularly concerning as the producer's accuracy in the humid biome
is only 18.7\% based on estimated areas.

\textbf{Evaluating model assumptions and variables}

I'm considering the emphasis on higher mapped estimates in the model.
The training sample of presence points is based on mapped patches, which
have omission errors that could lead to misleading interpretations about
regrowth conditions. They trained on 5.4 million hectares of regrowth
and aimed for a balanced sample.

They decided to exclude socioeconomic variables, claiming similar
accuracy, but these factors can significantly affect regrowth potential.
Their method leans towards ``biophysical potential'' rather than actual
outcomes, which may not represent local decisions well. They also use
2018 tree cover data, assuming conditions from 2000 to 2016 will hold
true for future predictions.

\textbf{Clarifying model scenarios and accuracy}

I'm finding it a bit odd that the model uses 2018 tree cover data
alongside 2015 land cover but doesn't include projections for 2030. This
makes it seem like they treat the future as static, which creates some
confusion about how they define the ``present'' and ``near future.''

Regarding spatial autocorrelation, they validated their findings with
random splits and also evaluated accuracy by distance, noting a drop at
2-2.5 km. Yet, they still report an overall accuracy of 87.9\%, which
they compare to an out-of-bag accuracy of 87.8\%, highlighting low
accuracy in Southeastern Asia.

\textbf{Analyzing model accuracy and sampling issues}

I'm noticing that while they present accuracy from balanced samples and
bootstrapping with 50 replicates, they don't provide aggregated accuracy
for distances over 5 km. A more robust spatial cross-validation might
mitigate the influence of nearby points, potentially leading to lower
accuracy and a risk of being overly optimistic.

Another concern is their balanced sampling, which sets prevalence at
50/50. This could lead to uncertainty in real-world scenarios with
smaller prevalences. They didn't appear to calibrate predicted
probabilities to actual prevalence, raising questions about whether they
assessed probability calibration properly. Their methodology may lack
the needed adjustments, like Platt scaling or isotonic regression, which
could skew area calculations if probabilities are not calibrated
accurately.

\textbf{Evaluating confidence intervals}

I'm seeing that they report an estimate of 215 million hectares with an
extremely narrow confidence interval (CI) of 214.78--215.22 million
hectares. This seems unrealistic for a model-based estimate. It appears
they might have computed the CI through Monte Carlo methods, but they
don't clarify the specific process for calculating it. I wonder if the
narrow range reflects a lack of variability in their model? This could
raise questions about the robustness of their estimates.

\textbf{Questioning confidence intervals and carbon estimates}

I'm noticing that the confidence interval (CI) of 215.22 million
hectares is very tight, suggesting it's more about rounding than
reflecting true uncertainty. It seems they derived the CI based on
weighting rather than proper modeling, which makes it questionable. A
credible CI should be wider to account for classification errors.

On carbon sequestration, they multiply area potentials by carbon
accumulation rates over 30 years, projecting a 14.3\% increase in global
carbon uptake. However, they could be overpredicting since they don't
factor in persistence or reclearance. They do acknowledge that realized
potential is likely lower.

\textbf{Considering regeneration and uncertainty factors}

I'm noticing they haven't included the time needed for natural
regeneration after conversion, and they also overlook land tenure or
opportunity costs. They mention potential costs but don't provide
estimates.

They claim they can sequester 23.4 Gt of carbon, with a range of 21.1 to
25.7 Gt, but it's unclear where that range comes from. It seems based on
uncertainties in the Cook-Patton estimates, without considering mapping
uncertainties or propagating classification errors.

Additionally, their datasets mix different resolutions, leading to
degradation in pixel-level predictions, and they define a broad domain
for non-regeneration that might still include various land types, like
savannas.

\textbf{Examining biome restrictions}

I see they've limited their focus to three forest biomes: dry broadleaf,
moist broadleaf, and coniferous. But does this also include natural
savannas? Many savanna ecoregions fall within ``tropical and subtropical
grasslands, savannas, and shrublands,'' which aren't included in their
model. Additionally, by restricting the domain to within ±25°, they
might be excluding northern edges. It seems the domain could still
encompass non-forested ecosystems within those forest biomes. It's a bit
confusing!

\textbf{Analyzing regeneration and prediction methods}

The researchers use local land cover classes that don't allow for forest
regeneration, which excludes sparse vegetation but includes grassland
and shrubland within forest biomes. This may extend their model's
domain, risking the afforestation of non-forest ecosystems if
predictions yield non-zero probabilities. They note they don't define
specific restoration sites, but such mapping might lead to unnatural
forestation in savanna patches. Additionally, they validate using 4.87
million random points but lack spatial cross-validation. Their
conclusions about potential regeneration need more precise definitions
of additionality regarding deforested lands too.

\textbf{Examining regrowth and carbon estimates}

The baseline scenario may not encompass regrowth, as the global carbon
flux estimate already accounts for regrowth across all lands. They
assert that these values are ``additional'' from currently deforested
lands, but this is a bit ambiguous. The global estimate includes removal
in primary and secondary forests, so if regrowth happens on deforested
lands that would naturally regrow anyway, it could lead to double
counting. Also, I wonder if there are confounding factors related to
misclassifications between natural regrowth and plantations.

\textbf{Analyzing classification accuracy and variable selection}

They reference Fagan 2022, noting a 90.6\% overall accuracy in
three-class classification. However, the producer's accuracy in humid
biomes is very low. They use the ``user's accuracy 85\%'' as robust,
which seems questionable to me.

Additionally, their use of variable selection in the random forest model
shows potential bias due to correlated predictors. They used PCA for
climate factors but not all correlated features. Their approach for the
``biome'' variable may lead to differing predictions by biome, and the
impact of accuracy at various distances is also worth noting.

\textbf{Evaluating sample imbalance and contributions}

I wonder if the inflated accuracy is due to sample imbalance across
distance classes. They did attempt a second evaluation by balancing the
samples in each interval, but the vertical lines they present indicate
ranges across replicates without aggregating the data. They still depend
on the 87.9\% general accuracy.

On the positive side, this paper provides a 30m global map dataset of
natural regeneration potential in the tropics, which is based on actual
regrowth occurrences. It's an open dataset, which is a significant
advantage for practitioners, though they do include many disclaimers.
However, the lack of open code is a downside.

\textbf{Assessing data reliance and clarity}

The authors rely on publicly accessible remote-sensing data and provide
large prediction files along with training datasets and methods.
However, without the code for replicating variable selection and many
pre-processing steps, replicability could be challenging. They do
describe their general approach, which is helpful, but calling code ``on
request'' may hinder full replication.

Visual clarity is strong with good maps and variable importance
presented in the Extended Data, though the explanation of confidence
intervals is somewhat awkward. Their claims about carbon benefits are
carefully moderated, excluding non-forested ecosystems like savannas to
avoid unintended afforestation. Still, the random forest probability
calibration raises some concerns.

\textbf{Exploring predictions and assessment}

The predictions are made at a 30 m resolution, but many input datasets
are much coarser, creating potential illusions in the results. There are
high and low carbon accumulation rates present, yet no error bars are
provided, which could be misleading---and that should really be
disclosed.

I'm thinking of crafting a 1000-word summary that highlights both the
positive aspects and criticisms of their approach. This piece has a
solid quality, leveraging a robust dataset and aligning well with
remote-sensing classification methods. Strengths include using actual
regrowth data and providing an open dataset, while also extending
important disclaimers.

\textbf{Evaluating methods and scores}

I'm planning to downscore some methods due to calibration issues,
internal illusions, and uncertainties. The use of ``CI'' seems unnatural
for their claims, and although they customize their ``biophysical-only
model,'' it's not fully transparent, which hampers replicability.
Overall, I think this is an advanced piece. I'd place it in the 80 to 88
percentile range, maybe around 84 with a 90\% confidence interval of
74-92.

For specific metrics, I think ``claims evidence'' could sit at 83,
``methods'' at 78, ``advancing knowledge'' at 87, ``logic
communication'' at 86, and finally ``open science'' at 68, as the data
is open, but the code isn't fully available, making replicability
tricky. However, the overall relevance is quite high, landing around 90.

\textbf{Setting performance scores and metrics}

I'm looking at the overall performance score, which I've placed around
84. For the ``tier\_should'' and ``tier\_will'' scores on a 0-5 scale,
I'm thinking they should be between 4.0 to 4.5. Without referencing the
actual publication, I'd suggest a normative score of 4.4 and predict 4.5
with a 90\% confidence range of 3.7 to 5.0. However, I want to be
cautious about known issues. So, I'll set ``should'' at 4.4 (3.8 - 4.9)
and ``will'' at 4.5 (3.7 - 5.0).

Now, it's time to produce the JSON with all the metrics, including
``overall,'' ``claims\_evidence,'' ``methods,''
``advancing\_knowledge,'' ``logic\_communication,'' ``open\_science,''
``global\_relevance,'' ``tier\_should,'' and ``tier\_will.'' Each will
include midpoint, lower, upper percentile metrics, and ``score,''
``ci\_lower,'' ``ci\_upper'' for the tiers. I'll also work on that
1000-word assessment summary.

\textbf{Creating a Diagnostic Summary}

I want to start by focusing on the project that defines ``potential for
natural regeneration'' based on machine learning from observed regrowth
between 2000 and 2016, differentiating between biophysical and
socioeconomic factors.

Next, there are some data quality issues. The Fagan 2022 dataset they're
using has significant omissions in humid biomes, which might bias
training and underestimate nature.

Their sampling seems unbalanced, with random forest probabilities
treated as accurate even though they aren't calibrated correctly. The
expected area isn't credible, resulting in a narrow confidence interval.

They need to address spatial autocorrelation and validation since their
overall accuracy of 87.9\% could be misleading without proper
cross-validation, especially with low accuracy in Southeastern Asia.

Moreover, the dataset's resolution mismatch is concerning, as they've
got 30 m predictions versus 300 m or 1 km inputs, and the mixed data
from 2018 versus 2015 also degrades their findings.

They exclude socioeconomic variables while labeling their projections
for 2030 as near-future without proper context. When discussing carbon
potential assimilation, only above-ground data is provided, neglecting
below-ground figures, which raises additionality concerns.

The project articulates risks surrounding reclearance and ephemeral
situations well.

As for implementation, they supply a 30 m map by tile and make their
code available upon request, but releasing the full pipeline would
improve replicability.

For improvement, I suggest several actions:

\begin{enumerate}
\def\labelenumi{\alph{enumi})}
\item
  They should conduct spatial block cross-validation and probability
  calibration while providing location-specific confusion metrics and
  alternatives for the correct prevalence threshold.
\item
  They can refine their methodology by considering Southeastern Asia
  more closely and ensuring better calibration.
\item
  It makes sense to present predictions including socioeconomic
  variables to provide a different scenario for comparison.
\item
  Address ``lack of data'' concerns explicitly.
\item
  An analysis of classification error predictions using weighted
  accuracy metrics could adjust area and carbon predictions.
\item
  Finally, addressing any illusions in their claims will enhance
  transparency.
\end{enumerate}

\textbf{Finalizing Credible Intervals}

Now, I'm working on setting credible intervals. I think we can propose
an overall score of 84 with a range of 75 to 91, but we might have to
adjust that; I'll set the lower bound to 76 and the upper to 91 for
clarity.

Next, I need to fill out the final JSON metrics, keeping in mind the
numbers have to fall between 0 and 100:

\begin{itemize}
\tightlist
\item
  Overall: mid 84; LB 74; UB 92.
\item
  Claims evidence: 83 {[}72, 90{]}
\item
  Methods: 77 {[}66, 86{]}
\item
  Advancing knowledge: 89 {[}78, 96{]}
\item
  Logic communication: 86 {[}77, 93{]}
\item
  Open science: I think I'll set it to 68 {[}55, 80{]}, as the code is
  still withheld, which affects replicability.
\item
  Global relevance: after reconsideration, I'll set this to 92 {[}84,
  97{]}.
\end{itemize}

Lastly, we should also determine the ``tier\_should'' as 4.4 {[}3.8,
4.8{]} and ``tier\_will'' as 4.5 {[}3.7, 5.0{]}. But what about the
applied and policy implications?

\textbf{Finalizing the Assessment Summary}

I'm putting together the final touches of the assessment summary. It
looks like I have to address the applied and policy aspects in a
friendly manner, while also noting that the reference group includes
serious research in remote sensing and regeneration.

I need to draft approximately 900 words. The schema requires paragraphs
instead of bullets, so I'll structure it accordingly. I should be
cautious about the confidence intervals for the area of 214.78 to 215.22
million hectares; they seem unrealistically narrow. I'll provide
calibration suggestions and ensure to include all necessary disclaimers.

Finally, I'll prepare the JSON string for
``paper\_assessment\_with\_tiers\_v2'' and make sure to craft a
comprehensive ``assessment\_summary.'' Let's get this all implemented!

\end{tcolorbox}

ewpage



\end{document}
