% Options for packages loaded elsewhere
\PassOptionsToPackage{unicode}{hyperref}
\PassOptionsToPackage{hyphens}{url}
\PassOptionsToPackage{dvipsnames,svgnames,x11names}{xcolor}
%
\documentclass[
  11pt,
  letterpaper,
  DIV=11,
  numbers=noendperiod]{scrreprt}

\usepackage{amsmath,amssymb}
\usepackage{iftex}
\ifPDFTeX
  \usepackage[T1]{fontenc}
  \usepackage[utf8]{inputenc}
  \usepackage{textcomp} % provide euro and other symbols
\else % if luatex or xetex
  \usepackage{unicode-math}
  \defaultfontfeatures{Scale=MatchLowercase}
  \defaultfontfeatures[\rmfamily]{Ligatures=TeX,Scale=1}
\fi
\usepackage{lmodern}
\ifPDFTeX\else  
    % xetex/luatex font selection
\fi
% Use upquote if available, for straight quotes in verbatim environments
\IfFileExists{upquote.sty}{\usepackage{upquote}}{}
\IfFileExists{microtype.sty}{% use microtype if available
  \usepackage[]{microtype}
  \UseMicrotypeSet[protrusion]{basicmath} % disable protrusion for tt fonts
}{}
\makeatletter
\@ifundefined{KOMAClassName}{% if non-KOMA class
  \IfFileExists{parskip.sty}{%
    \usepackage{parskip}
  }{% else
    \setlength{\parindent}{0pt}
    \setlength{\parskip}{6pt plus 2pt minus 1pt}}
}{% if KOMA class
  \KOMAoptions{parskip=half}}
\makeatother
\usepackage{xcolor}
\usepackage[margin=1in]{geometry}
\setlength{\emergencystretch}{3em} % prevent overfull lines
\setcounter{secnumdepth}{5}
% Make \paragraph and \subparagraph free-standing
\ifx\paragraph\undefined\else
  \let\oldparagraph\paragraph
  \renewcommand{\paragraph}[1]{\oldparagraph{#1}\mbox{}}
\fi
\ifx\subparagraph\undefined\else
  \let\oldsubparagraph\subparagraph
  \renewcommand{\subparagraph}[1]{\oldsubparagraph{#1}\mbox{}}
\fi


\providecommand{\tightlist}{%
  \setlength{\itemsep}{0pt}\setlength{\parskip}{0pt}}\usepackage{longtable,booktabs,array}
\usepackage{calc} % for calculating minipage widths
% Correct order of tables after \paragraph or \subparagraph
\usepackage{etoolbox}
\makeatletter
\patchcmd\longtable{\par}{\if@noskipsec\mbox{}\fi\par}{}{}
\makeatother
% Allow footnotes in longtable head/foot
\IfFileExists{footnotehyper.sty}{\usepackage{footnotehyper}}{\usepackage{footnote}}
\makesavenoteenv{longtable}
\usepackage{graphicx}
\makeatletter
\def\maxwidth{\ifdim\Gin@nat@width>\linewidth\linewidth\else\Gin@nat@width\fi}
\def\maxheight{\ifdim\Gin@nat@height>\textheight\textheight\else\Gin@nat@height\fi}
\makeatother
% Scale images if necessary, so that they will not overflow the page
% margins by default, and it is still possible to overwrite the defaults
% using explicit options in \includegraphics[width, height, ...]{}
\setkeys{Gin}{width=\maxwidth,height=\maxheight,keepaspectratio}
% Set default figure placement to htbp
\makeatletter
\def\fps@figure{htbp}
\makeatother
% definitions for citeproc citations
\NewDocumentCommand\citeproctext{}{}
\NewDocumentCommand\citeproc{mm}{%
  \begingroup\def\citeproctext{#2}\cite{#1}\endgroup}
\makeatletter
 % allow citations to break across lines
 \let\@cite@ofmt\@firstofone
 % avoid brackets around text for \cite:
 \def\@biblabel#1{}
 \def\@cite#1#2{{#1\if@tempswa , #2\fi}}
\makeatother
\newlength{\cslhangindent}
\setlength{\cslhangindent}{1.5em}
\newlength{\csllabelwidth}
\setlength{\csllabelwidth}{3em}
\newenvironment{CSLReferences}[2] % #1 hanging-indent, #2 entry-spacing
 {\begin{list}{}{%
  \setlength{\itemindent}{0pt}
  \setlength{\leftmargin}{0pt}
  \setlength{\parsep}{0pt}
  % turn on hanging indent if param 1 is 1
  \ifodd #1
   \setlength{\leftmargin}{\cslhangindent}
   \setlength{\itemindent}{-1\cslhangindent}
  \fi
  % set entry spacing
  \setlength{\itemsep}{#2\baselineskip}}}
 {\end{list}}
\usepackage{calc}
\newcommand{\CSLBlock}[1]{\hfill\break\parbox[t]{\linewidth}{\strut\ignorespaces#1\strut}}
\newcommand{\CSLLeftMargin}[1]{\parbox[t]{\csllabelwidth}{\strut#1\strut}}
\newcommand{\CSLRightInline}[1]{\parbox[t]{\linewidth - \csllabelwidth}{\strut#1\strut}}
\newcommand{\CSLIndent}[1]{\hspace{\cslhangindent}#1}

% hyperref-colors.tex
\usepackage{xcolor}
\usepackage{hyperref}

% Unjournal colours
\definecolor{unjournalgreen}{HTML}{99BB66}
\definecolor{unjournalorange}{HTML}{F19E4B}

\hypersetup{
  colorlinks = true,
  linkcolor  = unjournalgreen,   % internal links (TOC, sections, refs)
  citecolor  = unjournalorange,  % citations
  urlcolor   = unjournalorange   % URLs
}
\KOMAoption{captions}{tableheading,figureheading}
\makeatletter
\@ifpackageloaded{tcolorbox}{}{\usepackage[skins,breakable]{tcolorbox}}
\@ifpackageloaded{fontawesome5}{}{\usepackage{fontawesome5}}
\definecolor{quarto-callout-color}{HTML}{909090}
\definecolor{quarto-callout-note-color}{HTML}{0758E5}
\definecolor{quarto-callout-important-color}{HTML}{CC1914}
\definecolor{quarto-callout-warning-color}{HTML}{EB9113}
\definecolor{quarto-callout-tip-color}{HTML}{00A047}
\definecolor{quarto-callout-caution-color}{HTML}{FC5300}
\definecolor{quarto-callout-color-frame}{HTML}{acacac}
\definecolor{quarto-callout-note-color-frame}{HTML}{4582ec}
\definecolor{quarto-callout-important-color-frame}{HTML}{d9534f}
\definecolor{quarto-callout-warning-color-frame}{HTML}{f0ad4e}
\definecolor{quarto-callout-tip-color-frame}{HTML}{02b875}
\definecolor{quarto-callout-caution-color-frame}{HTML}{fd7e14}
\makeatother
\makeatletter
\@ifpackageloaded{bookmark}{}{\usepackage{bookmark}}
\makeatother
\makeatletter
\@ifpackageloaded{caption}{}{\usepackage{caption}}
\AtBeginDocument{%
\ifdefined\contentsname
  \renewcommand*\contentsname{Table of contents}
\else
  \newcommand\contentsname{Table of contents}
\fi
\ifdefined\listfigurename
  \renewcommand*\listfigurename{List of Figures}
\else
  \newcommand\listfigurename{List of Figures}
\fi
\ifdefined\listtablename
  \renewcommand*\listtablename{List of Tables}
\else
  \newcommand\listtablename{List of Tables}
\fi
\ifdefined\figurename
  \renewcommand*\figurename{Figure}
\else
  \newcommand\figurename{Figure}
\fi
\ifdefined\tablename
  \renewcommand*\tablename{Table}
\else
  \newcommand\tablename{Table}
\fi
}
\@ifpackageloaded{float}{}{\usepackage{float}}
\floatstyle{ruled}
\@ifundefined{c@chapter}{\newfloat{codelisting}{h}{lop}}{\newfloat{codelisting}{h}{lop}[chapter]}
\floatname{codelisting}{Listing}
\newcommand*\listoflistings{\listof{codelisting}{List of Listings}}
\makeatother
\makeatletter
\makeatother
\makeatletter
\@ifpackageloaded{caption}{}{\usepackage{caption}}
\@ifpackageloaded{subcaption}{}{\usepackage{subcaption}}
\makeatother
\ifLuaTeX
  \usepackage{selnolig}  % disable illegal ligatures
\fi
\usepackage{bookmark}

\IfFileExists{xurl.sty}{\usepackage{xurl}}{} % add URL line breaks if available
\urlstyle{same} % disable monospaced font for URLs
\hypersetup{
  pdftitle={Comparing LLM and human reviews of social science research using data from Unjournal.org},
  pdfauthor={Valentin Klotzbücher; David Reinstein; Tianmai Michael Zhang},
  colorlinks=true,
  linkcolor={blue},
  filecolor={Maroon},
  citecolor={Blue},
  urlcolor={Blue},
  pdfcreator={LaTeX via pandoc}}

\title{Comparing LLM and human reviews of social science research using
data from Unjournal.org}
\author{Valentin Klotzbücher \and David Reinstein \and Tianmai Michael
Zhang}
\date{2025-12-13}

\begin{document}
\maketitle
\begin{abstract}
We study how a frontier language model evaluates social‑science research
compared to expert human reviewers in The Unjournal. Using the same
structured rubric as human evaluators, we ask GPT‑5 Pro to rate papers
on overall quality, methods, evidence, communication, openness, and
global relevance, and to produce a narrative assessment anchored in the
PDF of each paper. We first compare its quantitative ratings to
aggregated human scores across The Unjournal's existing evaluations,
then take a closer qualitative look at a small set of focal papers,
including a high‑profile mapping study of natural regeneration. For
these focal cases, we examine where the model's written review overlaps
with and diverges from the human reports, and how both sides describe
the main strengths, weaknesses, and policy relevance of the work. So
far, the model reliably identifies many of the same methodological and
interpretive issues that human experts emphasize, but it tends to
translate these into more generous numerical ratings and narrower
uncertainty intervals. We view this as an initial, work‑in‑progress
probe of LLM‑based peer review in a high‑stakes, policy‑relevant domain,
and as the first step toward a broader benchmark and set of tools for
comparing and combining human and AI research evaluations.
\end{abstract}

\renewcommand*\contentsname{Table of contents}
{
\hypersetup{linkcolor=}
\setcounter{tocdepth}{2}
\tableofcontents
}
\bookmarksetup{startatroot}

\chapter{Introduction}\label{introduction}

\definecolor{unjournalgreen}{HTML}{99BB66}
\definecolor{unjournalorange}{HTML}{F19E4B}

\hypersetup{
  colorlinks = true,
  linkcolor  = unjournalgreen,  
  citecolor  = unjournalorange,  
  urlcolor   = unjournalorange  
}

Is AI good at peer-reviewing? Does it offer useful and valid feedback?
Can it predict how human experts will rate research across a range of
categories? How can it help academics do this ``thankless'' task better?
Is it particularly good at spotting errors? Are there specific
categories, e.g.~spotting math errors or judging real-world relevance,
where it does surprisingly well or poorly? How does its ``research
taste'' compare to humans?

If AI research-evaluation works it could free up a lot of scientific
resources -- perhaps \$1.5 billion/year in the US alone
(\citeproc{ref-aczel2021billion}{Aczel, Szaszi, and Holcombe 2021}) --
and offer more continual and detailed review, helping improve research.
It could also help characterize methodological strengths/weaknesses
across papers, aiding training and research direction-setting.
Furthermore, a key promise of AI is to directly improve science and
research. Understanding how AI engages with research evaluations may
provide a window into its values, abilities, and limitations.

In this project, we test whether current large language models (LLMs)
can generate research evaluations that are comparable, in structure and
content, to expert human reviews. The Unjournal systematically
prioritizes ``impactful'' research and pays for high‑quality human
evaluations, including structured numeric ratings with credible
intervals, claim identification and assessment, predictions, and
detailed narrative reports. We use a frontier LLM (OpenAI's GPT‑5 Pro)
to review the same social‑science and policy‑relevant working papers
under essentially the same rubric.

For a first pass we focus on papers that already have completed
Unjournal evaluation packages. For each of 47 such papers, the model
reads the PDF that human evaluators saw and returns: (i) percentile
ratings and 90\% credible intervals on The Unjournal's seven criteria,
and (ii) two 0--5 journal‑tier scores (``should'' and ``will'' be
published). In an additional, richer run on a small set of focal papers,
we keep these quantitative outputs but also require a long diagnostic
summary and high‑effort reasoning trace. We then compare the model's
ratings, journal‑tier predictions, and qualitative assessments to the
existing human evaluations.

Future iterations will extend this design to papers that are still in
The Unjournal's pipeline, where no human evaluations are yet public.
This will let us study out‑of‑sample prediction, reduce the risk of
model contamination from published evaluations, and test LLMs as tools
for triaging and prioritising new work.

\textbf{Our work in context}

Luo et al. (\citeproc{ref-Luo2025}{2025}) survey LLM roles from idea
generation to peer review, including experiment planning and automated
scientific writing. They highlight opportunities (productivity, coverage
of long documents) alongside governance needs (provenance, detection of
LLM-generated content, standardizing tooling) and call for reliable
evaluation frameworks.

Eger et al. (\citeproc{ref-eger2025}{2025}) provide a broad review of
LLMs in science and a focused discussion of AI‑assisted peer review.
They argue: (i) peer‑review data is scarce and concentrated in
CS/OpenReview venues; (ii) targeted assistance that preserves human
autonomy is preferable to end‑to‑end reviewing; and (iii) ethics and
governance (bias, provenance, detection of AI‑generated text) are
first‑class constraints.

Zhang and Abernethy (\citeproc{ref-Zhang2025}{2025}) propose deploying
LLMs as quality checkers to surface critical problems instead of
generating full narrative reviews. Using papers from WITHDRARXIV and an
automatic evaluation framework that leverages ``LLM-as-judge,'' they
find the best performance from top reasoning models but still recommend
human oversight.

Pataranutaporn et al. (\citeproc{ref-Pataranutaporn2025}{2025}) asked
four nearly state-of-the-art LLM models (GPT-4o mini, Claude 3.5 Haiku,
Gemma 3 27B, and LLaMA 3.3 70B) to consider 1220 unique papers ``drawn
from 110 economics journals excluded from the training data of current
LLMs''. They prompted the models to act ``in your capacity as a reviewer
for {[}a top-5 economics journal{]}'' and make a publication
recommendation using a 6-point scale ranging from ``1 = Definite
Reject\ldots{}'' to ``6. Accept As Is\ldots{}''. They asked it to
evaluate each paper on a 10-point scale for originality, rigor, scope,
impact, and whether it was `written by AI'. They also (separately) had
LLMs rate 330 papers with the authors' identities removed, or replacing
the names with fake male/female names and real elite or non-elite
institutions (check this) or with prominent male or female economists
attached.

They compare the LLMs' ratings with the RePEC rankings for the journals
the papers were published in, finding general alignment. They find mixed
results on detecting AI-generated papers. In the names/institutions
comparisons, they also find the LLMs show biases towards named
high-prestige male authors relative to high-prestige female authors, as
well as biases towards elite institutions and US/UK universities.

There have been several other empirical benchmarking projects, including
work covered in LLM4SR: A Survey on Large Language Models for Scientific
Research and \href{https://arxiv.org/abs/2502.05151}{Transforming
Science with Large Language Models: A Survey on AI-assisted Scientific
Discovery, Experimentation, Content Generation, and Evaluation}.

Our project distinguishes itself in its use of \emph{actual} human
evaluations of research in economics and adjacent fields, past and
\emph{prospective}, including both reports, ratings, and
predictions.\footnote{Other work has relied on collections of research
  and grant reviews, including NLPEER, SubstanReview, and the Swiss
  National Science Foundation. That data has a heavy focus on
  computer-science adjacent fields, and iss less representative of
  mainstream research peer review practices in older, established
  academic fields. Note that The Unjournal commissions the evaluation of
  impactful research, often from high-prestige working paper archives
  like NBER, and makes all evaluations public, even if they are highly
  critical of the paper.} The Unjournal's 50+ evaluation packages enable
us to train and benchmark the models. Their pipeline of future
evaluations allow for clean out-of-training-data predictions and
evaluation. Their detailed written reports and multi-dimensional ratings
also allows us to compare the `taste', priorities, and comparative
ratings of humans relative to AI models across the different criteria
and domains. The `journal tier prediction' outcomes also provides an
external ground-truth\footnote{About verifiable publication outcomes,
  not about the `true quality' of the paper of course.} enabling a
human-vs-LLM horse race. We are also planning multi-armed trials on
these human evaluations (\citeproc{ref-Brodeur2025}{Brodeur et al.
2025}) to understand the potential for \emph{hybrid} human-AI evaluation
in this context.

\bookmarksetup{startatroot}

\chapter{Data and methods}\label{data-and-methods}

We draw on two main sources:

\begin{enumerate}
\def\labelenumi{\arabic{enumi})}
\tightlist
\item
  Human evaluations from
  \href{https://unjournal.github.io/unjournaldata/index.html}{The
  Unjournal's public evaluation data} (PubPub reports and the Coda
  evaluation form export).\\
\item
  LLM‑generated evaluations using a structured JSON‑schema prompt with
  \texttt{gpt-5-pro-2025-10-06} (knowledge cut-off: 30 September 2024).
\end{enumerate}

\section{Unjournal.org evaluations}\label{unjournal.org-evaluations}

We use The Unjournal's public data for a baseline comparison. At The
Unjournal each paper is typically evaluated (aka `reviewed') by two
expert evaluators\footnote{Occasionally they use 1 or 3 evaluators.} who
provide quantitative ratings on a 0--100 percentile scale for each of
seven criteria (with 90\% credible intervals),\footnote{See their
  guidelines
  \href{https://globalimpact.gitbook.io/the-unjournal-project-and-communication-space/policies-projects-evaluation-workflow/evaluation/guidelines-for-evaluators\#quantitative-metrics}{here};
  these criteria include ``Overall assessment'', ``Claims, strength and
  characterization of evidence'', ``Methods: Justification,
  reasonableness, validity, robustness'', ``Advancing knowledge and
  practice'', ``Logic and communication'', ``Open, collaborative,
  replicable science'', and ``Relevance to global priorities, usefulness
  for practitioners''} two ``journal tier'' ratings on a 0.0 - 5.0
scale,\footnote{``a normative judgment about `how well the research
  should publish'\,'' and ``a prediction about where the research will
  be published''} a written evaluation (resembling a referee report for
a journal), and identification and assessment of the paper's ``main
claim''. For our initial analysis, we extracted these human ratings and
aggregated them, taking the average score per criterion across
evaluators (and noting the range of individual scores).

All papers have completed The Unjournal's evaluation process (meaning
the authors received a full evaluation on the Unjournal platform, which
has been publicly posted at unjournal.pubpub.org). The sample includes
papers spanning 2017--2025 working papers in development economics,
growth, health policy, environmental economics, and related fields that
The Unjournal identified as high-impact. Each of these papers has
quantitative scores from at least one human evaluator, and many have
multiple (2-3) human ratings.

\section{LLM-based evaluation}\label{llm-based-evaluation}

Following The Unjournal's
\href{https://globalimpact.gitbook.io/the-unjournal-project-and-communication-space/policies-projects-evaluation-workflow/evaluation/guidelines-for-evaluators}{standard
guidelines for evaluators} and their
\href{https://coda.io/form/Unjournal-Evaluation-form-academic-stream-Coda-updated-version_dGjfMZ1yXME}{academic
evaluation form}, evaluators are asked to consider each paper along the
following dimensions: \textbf{claims \& evidence}, \textbf{methods},
\textbf{logic \& communication}, \textbf{open science}, \textbf{global
relevance}, and an \textbf{overall} assessment. Ratings are interpreted
as percentiles relative to serious recent work in the same area. For
each metric, evaluators are asked for the midpoint of their beliefs and
their 90\% credible interval, to communicate their uncertainty. For the
journal rankings measure, we ask both ``what journal ranking tier should
this work be published in? (0.0-5.0)'' and ``what journal ranking tier
will this work be published in? (0.0-5.0)'', with some further
explanation.The full prompt can be seen in the code below -- essentially
copied from the Unjournal's guidelines page.

We captured the versions of each paper that was evaluated by The
Unjournal's human evaluators, downloading from the links provided in The
Unjournal's Coda database.

We evaluate each paper by passing the PDF directly to the model and
requiring a strict, machine‑readable JSON output. This keeps the
assessment tied to the document the authors wrote. Direct ingestion
preserves tables, figures, equations, and sectioning, which ad‑hoc text
scraping can mangle. It also avoids silent trimming or segmentation
choices that would bias what the model sees.

We enforce a JSON Schema for the results. The model must return one
object for each of the named criteria including a midpoint rating and a
90\% interval for each rating. This guarantees that every paper is
scored on the same fields with the same types and bounds. It makes the
analysis reproducible and comparisons clean.

We request credible intervals (as we do for human evaluators) to allow
the model to communicate its uncertainty rather than suggest false
precision; these can also be incorporated into our metrics, penalizing a
model's inaccuracy more when it's stated with high confidence.

Relying on GPT-5 Pro, we use a single‑step call with a reasoning model
that supports file input. One step avoids hand‑offs and summary loss
from a separate ``ingestion'' stage. The model reads the whole PDF and
produces the JSON defined above. We do not retrieve external sources or
cross‑paper material for these scores; the evaluation is anchored in the
manuscript itself.

The Python pipeline uploads each PDF once and caches the returned file
id keyed by path, size, and modification time. We submit one background
job per PDF to the OpenAI Responses API with ``high'' reasoning effort
and server‑side JSON‑Schema enforcement. Submissions record the response
id, model id, file id, status, and timestamps.

We then polls job status and, for each completed job, retrieve the raw
JSON object, and write the responses to disk.

\bookmarksetup{startatroot}

\chapter*{References}\label{references}
\addcontentsline{toc}{chapter}{References}

\markboth{References}{References}

\phantomsection\label{refs}
\begin{CSLReferences}{1}{0}
\bibitem[\citeproctext]{ref-aczel2021billion}
Aczel, Balazs, Barnabas Szaszi, and Alex O Holcombe, {``A billion-dollar
donation: Estimating the cost of researchers' time spent on peer
review,''} \emph{Research integrity and peer review}, 6 (2021), 1--8
(Springer).

\bibitem[\citeproctext]{ref-Brodeur2025}
Brodeur, Abel, David Valenta, Alexandru Marcoci, Juan P Aparicio, Derek
Mikola, Bruno Barbarioli, Rohan Alexander, Lachlan Deer, Tom Stafford,
Lars Vilhuber, and others,
{``\href{https://hdl.handle.net/10419/308508}{Comparing human-only,
AI-assisted, and AI-led teams on assessing research reproducibility in
quantitative social science},''} \emph{Institute for Replication (I4R)
Discussion Paper}, 195 (2025).

\bibitem[\citeproctext]{ref-eger2025}
Eger, Steffen, Yong Cao, Jennifer D'Souza, Andreas Geiger, Christian
Greisinger, Stephanie Gross, Yufang Hou, Brigitte Krenn, Anne Lauscher,
Yizhi Li, Chenghua Lin, Nafise Sadat Moosavi, Wei Zhao, and Tristan
Miller, {``\href{https://arxiv.org/abs/2502.05151}{Transforming science
with large language models: A survey on AI-assisted scientific
discovery, experimentation, content generation, and evaluation},''}
\emph{arXiv preprint arXiv:2505.05151}, (2025).

\bibitem[\citeproctext]{ref-Luo2025}
Luo, Ziming, Zonglin Yang, Zexin Xu, Wei Yang, and Xinya Du,
{``\href{https://arxiv.org/abs/2501.04306}{LLM4SR: A survey on large
language models for scientific research},''} \emph{arXiv preprint
arXiv:2501.04306}, (2025).

\bibitem[\citeproctext]{ref-Pataranutaporn2025}
Pataranutaporn, Pat, Nattavudh Powdthavee, Chayapatr Achiwaranguprok,
and Pattie Maes,
{``\href{https://api.semanticscholar.org/CorpusID:277510314}{Can AI
solve the peer review crisis? A large scale cross model experiment of
LLMs' performance and biases in evaluating over 1000 economics
papers},''} 2025.

\bibitem[\citeproctext]{ref-Zhang2025}
Zhang, Tianmai M, and Neil F Abernethy,
{``\href{https://arxiv.org/abs/2505.23824v2}{Reviewing scientific papers
for critical problems with reasoning LLMs: Baseline approaches and
automatic evaluation},''} \emph{arXiv preprint arXiv:2505.23824},
(2025).

\end{CSLReferences}

\cleardoublepage
\phantomsection
\addcontentsline{toc}{part}{Appendices}
\appendix

\chapter{LLM evaluation summaries and reasoning
traces}\label{llm-evaluation-summaries-and-reasoning-traces}

\begin{tcolorbox}[enhanced jigsaw, left=2mm, opacitybacktitle=0.6, colframe=quarto-callout-important-color-frame, rightrule=.15mm, toprule=.15mm, bottomtitle=1mm, title=\textcolor{quarto-callout-important-color}{\faExclamation}\hspace{0.5em}{Important}, coltitle=black, colback=white, arc=.35mm, bottomrule=.15mm, toptitle=1mm, breakable, colbacktitle=quarto-callout-important-color!10!white, opacityback=0, titlerule=0mm, leftrule=.75mm]

Current testing with GPT-5 Pro to generate detailed reasoning traces and
assessment summaries for various UNRELATED papers. The full outputs are
included below for reference.

\end{tcolorbox}

\renewcommand\normalsize{\small}

\section{1\_PDFsam\_Nature published version 17 Nov
2021}\label{pdfsam_nature-published-version-17-nov-2021}

\begin{tcolorbox}[enhanced jigsaw, left=2mm, opacitybacktitle=0.6, colframe=quarto-callout-note-color-frame, rightrule=.15mm, toprule=.15mm, bottomtitle=1mm, title=\textcolor{quarto-callout-note-color}{\faInfo}\hspace{0.5em}{Model assessment summary}, coltitle=black, colback=white, arc=.35mm, bottomrule=.15mm, toptitle=1mm, breakable, colbacktitle=quarto-callout-note-color!10!white, opacityback=0, titlerule=0mm, leftrule=.75mm]

This paper exploits a novel, high-frequency data source---8 million
calls to 23 helplines in 19 countries---to describe mental health
concerns during COVID-19 and relate call volumes/topics to infection
rates and policies. The design has clear strengths (large scale, daily
observability, cross-context variation, topic tagging, pre-period
coverage for many helplines, and multiple checks) but several
identification and measurement limitations temper causal claims. Data
are heterogeneous: some helplines provide call-level records with topics
and demographics, others provide only daily aggregates, and the US
analysis uses ``routed'' calls rather than answered calls. The
harmonization of diverse topic taxonomies into eight broad categories is
reasonable but introduces non-classical measurement error; category
definitions differ by helpline and may have changed over time, yet no
validation audit or coder consistency checks are shown. Caller
representativeness is unknown and could shift during the pandemic;
although the paper focuses on within-helpline changes (which removes
time-invariant selection), the composition of callers (e.g., first-time
versus repeat, gender/age mix, access to phones/privacy) plausibly
changed with lockdowns and school/work closures. Repeated calls by the
same individual are counted each time, and while the share of
first-time/sporadic callers is reported, dynamics in this mix are not
systematically modeled. A major concern is supply constraints: during
surges, unanswered calls rose and capacity expanded gradually. The paper
argues answered/unanswered status is random with respect to topics, so
composition should be unaffected, but this assumption is untested and
may be unrealistic if call-handling protocols triage acute/suicidal risk
or if patience to wait differs by problem type. For volume analyses,
capacity constraints create a lower bound on demand; for topic shares,
selective answering (even unintentionally) could bias estimates. The
main descriptive result---35\% peak increase in calls about six weeks
post-outbreak---is based on an event-study that includes helpline, year,
week-of-year, and day-of-week fixed effects. However, 4 of 21 helplines
lack 2019 baselines; Figure 1 shows only four weeks of pre-trend,
limiting a visual test of parallel trends; and other contemporaneous
shocks could confound the timing. Topic analyses use a binary
post-outbreak indicator (and later a two-period split), which compresses
rich dynamics; modeling flexible event-time effects by topic would
better show pre-trends and persistence. The U.S. panel leverages
state-week variation in infections, the Oxford stringency index, and an
income-support index with two-way fixed effects. This is informative but
still observational; policy indices are potentially endogenous to
unobserved distress, and infection rates proxy for multiple time-varying
shocks (risk salience, media attention, hospital strain). Week fixed
effects absorb nationwide shocks, but local confounders (unemployment,
closures of mental-health services, vaccination rollouts, media
campaigns, center-specific staffing or technical changes) remain
unaccounted for. The dependent variable for the U.S. is ``routed''
calls, not answered calls; routing and overflow rules, center hours,
wait-time thresholds, or technical outages can change over time and
across states, risking non-stationary measurement error. State-of-origin
based on phone numbers introduces misclassification via mobile numbers
from other states; while likely mean-zero, it adds noise that could
attenuate effects. The choice to cluster standard errors at the
``state--month'' level is unconventional; two-way clustering by state
and week (or Driscoll--Kraay) would better address spatial and temporal
dependence. In the France/Germany topic-specific regressions (daily,
with helpline fixed effects and seasonal controls but not week fixed
effects), identification relies on within-country temporal variation;
with only two helplines, inference about policy elasticities is
vulnerable to coincident trends or specification choices. Reported
elasticities are small and significant (e.g., stricter NPIs associated
with more suicide-related calls; income support with fewer), but causal
interpretation remains tentative. The claim that pandemic-related
concerns ``replaced rather than exacerbated'' underlying anxieties leans
on shares: several non-COVID topics fall in relative frequency. This
does not rule out increases in absolute counts for some categories,
especially when total calls rise; indeed, early-wave suicidality shares
do not significantly change overall (Fig. 2b), though later waves see
decreases in the two-country sample. The paper notes possible
underreporting of violence because callers may be less able to safely
phone during stay-at-home orders, but this is not directly tested; this
could bias the inference that violence became less prevalent. Robustness
is partly addressed through weighted/unweighted models, alternative
outbreak definitions (infection threshold vs.~SIP), period splits, and
supplementary specifications; nonetheless, more explicit falsification
tests (e.g., placebo ``pseudo-outbreak'' dates in 2019; alternative
clustering schemes; inclusion of helpline-specific linear trends;
controlling for local unemployment or mobility; instrumenting policies
with neighboring states' policies or political variables) would
strengthen credibility. Transparency is good for methods and code (Stata
do-files shared), but replication is constrained: most data are
proprietary, require separate agreements, and cross-helpline
compatibility rests on non-public coding manuals. Figures are generally
clear and consistent, and the narrative is careful to couch policy
results as associations, though some language (e.g., ``implies that
financial relief can allay distress'') could be toned down given
identification caveats. Overall, the work convincingly documents time
patterns and topic shifts in helpline activity during COVID-19 and
provides suggestive evidence that policy stringency is associated with
increased distress and that income support is associated with
reductions. The main limitations are potential selection/measurement
changes on both the demand and supply sides of helplines, compressed
treatment timing in topic models, partial baseline coverage, endogeneity
of policies, and replication barriers due to data access. Despite these,
the paper usefully advances the monitoring toolkit for population mental
health and supplies actionable correlational evidence relevant to crisis
policymaking.

\end{tcolorbox}

\begin{tcolorbox}[enhanced jigsaw, left=2mm, opacitybacktitle=0.6, colframe=quarto-callout-tip-color-frame, rightrule=.15mm, toprule=.15mm, bottomtitle=1mm, title=\textcolor{quarto-callout-tip-color}{\faLightbulb}\hspace{0.5em}{Model reasoning trace}, coltitle=black, colback=white, arc=.35mm, bottomrule=.15mm, toptitle=1mm, breakable, colbacktitle=quarto-callout-tip-color!10!white, opacityback=0, titlerule=0mm, leftrule=.75mm]

\textbf{Analyzing helpline data}

I'm looking at the data for 17 helplines from 2019, specifically
focusing on the same days. However, it seems they don't show pre-trends
before week 0, just the 4 weeks prior. They have various methods like
weighted and unweighted data. I noticed they didn't utilize
difference-in-differences with controls but instead used event time with
helpline fixed effects while considering year, week of the year, and day
of the week. There's a lot to unpack here!

\textbf{Evaluating NPI effects and data availability}

They're discussing trends related to the treatment effects of
non-pharmaceutical interventions (NPIs) and income support, but they
rely on Oxford indices, which might have measurement limitations. They
consider policy stringency across states, suggesting that national
policies could skew the scores, reducing within-state variation. The
income support index for the US heavily weighs national policies,
potentially limiting variation. They log data weekly and find small
coefficients, showing positive but not statistically significant effects
in the US. Data is accessible upon permission, and the code is openly
available, which is great!

\textbf{Exploring replication barriers and limitations}

It seems there's a replication barrier due to data sharing restrictions.
They show various specifications in supplementary materials but aren't
robust to heterogeneity among helplines. They account for capacity
constraints, but I wonder if they adjusted for multiple hypotheses ---
it's not mentioned. The data on mental health across 19 countries is
useful, demonstrating the positive impact of income support. However,
generalizing findings to lower-income countries may be limited. There
are also concerns about the variability in call supply capacity and
whether it affects the results---like, are some callers more patient
based on the topic?

\textbf{Questioning assumptions in call patterns}

It seems the data isn't entirely random; they didn't test for call
reasons that remain unaffected by supply. Call topics are based on
answered calls, meaning changes in capacity could alter topic
composition, but they assume call answering is random --- which is
questionable. Additionally, the data aggregates calls rather than unique
individuals, so repeated calls could skew mental health observations.
They mention that a large percentage of calls were from first-time or
sporadic callers, but the range is inconsistent across helplines without
adjustments for repeat callers. There are variations in how suicidal
calls are classified, and mapping categories may be inconsistent due to
changed classification codes during the pandemic. This inconsistency
makes me wonder about their conclusions regarding the relationship
between issues and call prevalence---there seems to be a lack of
clarity!

\textbf{Examining method choices and results}

I noticed that the abstract claims calls related to suicide are less
prevalent than before. In the first wave, the figures show decreased
relationships regarding livelihood and violence, but suicidality
remained unchanged. Overall, they state a decrease across both first and
later waves. Regarding methods, they applied OLS with a transformation
that may be odd for zero counts. It's interesting that they cluster
standard errors at the helpline-week level, which seems unusual,
potentially to account for within-week correlation. For the US weekly
state-level panel, clustering at the state-month level raises questions
about the number of clusters being adequate.

\textbf{Evaluating clustering methods}

I'm considering the clustering methods used, which involve 51 states
over several months from 2019 to 2021. Clustering by state gives at
least 51 clusters, but clustering at state-month would create more. I'm
wondering if two-way clustering is better, especially since they
included both state and week fixed effects. Using cluster at state-month
might not consider cross-state correlation across weeks, even though
week fixed effects account for national shocks. So, I think state-level
clustering would be the safer approach to avoid correlated errors.

\textbf{Assessing methodological choices}

They clustered at the state-month level to address serial correlation,
but it feels a bit unnatural. The variation in non-pharmaceutical
interventions (NPIs) across states appears limited since national
restrictions often overshadow local policies. It's noted that they show
large standard errors without significance, while Germany and France had
significant NPI effects. They share their Stata do-files and source
data, but there's a potential confusion treating ``fear of infection''
as only pandemic-related issues. Also, they might not adequately account
for seasonality and rely on some assumptions that could lead to
misleading conclusions.

\textbf{Analyzing comparability and findings}

It looks like they included week-of-year fixed effects to control for
seasonality, which is good, but it reduces comparability since these
factors are helpline-specific, interacted with ``year,'' ``week,'' and
``day.'' There's also a concern about p-hacking, as they tested eight
categories without adjusting for multiple tests, even though they note
their results are robust with small effect sizes. The claim that
``income support alleviates distress from lockdowns'' might simply
reflect general time patterns. For the US, they didn't find significant
effects from stringency or income support, but they concluded that
suicide-related calls increased with stricter policies and decreased
when income support was provided.

\textbf{Clarifying findings and relevance}

The statement about findings particularly refers to helplines in Germany
and France, while the US data isn't included, which might overshadow the
overall conclusions due to the limited sample size. There's also
uncertainty regarding the interplay with changes in mental healthcare
supply. They mention re-running the analysis as the situation matures.
The global relevance is rated high given the focus on mental health, but
the open science aspect is moderate since the data is restricted,
although they do provide the code and a contact list.

\textbf{Evaluating research quality}

They provide aggregated source data, which is great! The clarity in
writing and figures earns a high score in ``Logic\_communication,''
though there are no sample selection details for all helplines. The
``Claims\_evidence'' is rated moderate-high due to the potential
illusions of substitution, making it a bit debatable. The methods show
moderate-high quality since they used fixed effects, but there's an
observational identification of policy effects. The contribution to
knowledge is high with new cross-country data. Overall, this article is
of high quality, likely earning an 85 percentile.

\textbf{Assessing quality scores}

It's not the absolute best, but it's strong overall. I'd rate
``Claims\_Evidence'' around 78, ``Methods'' at 75, and ``Advancing
Knowledge'' at 80. For ``Logic \& Communication,'' I think an 88 is
fitting. However, ``Open Science'' might drop to about 60 due to data
restrictions, while ``Global Relevance'' sits at 82.

I'll provide credible intervals, maybe ±10 for the overall. For journal
tier predictions, the normative ``should'' might be around 4.2 since
it's in Nature, while ``will'' could range from 3.8 to 4.4. I'll propose
``should'' at 4.1 and ``will'' at 4.3, with intervals.

\textbf{Preparing for journal ranking assessment}

The request is for ``Journal ranking tier `should' and `will'\,'' on a
scale of 0.0 to 5.0, including credible intervals. I think I'll aim for
a midpoint score of 4.0, although that might feel a bit unnatural due to
limitations in open science.

Next, I need to craft a summary of about 1,000 words in one compact
paragraph. They want it to be specific, neutral, and concrete, so I'll
aim for a final word count between 800 to 1100 words to stay on target.

\textbf{Structuring content for analysis}

Let's outline the content for the analysis. I'll start with data and
measurement details, mentioning helplines from 19 countries, with 8
million calls and various topics. I'll highlight strengths, like a
high-frequency sample using JHU and Oxford indices, while noting
limitations such as unknown caller representativeness and measurement
errors.

I plan to address potential biases from repeated callers, the effect of
concurrent events, and the lack of clear causation in conclusions about
income support. I'll provide suggestions for improvements, like
additional event studies, modeling supply constraints, and using placebo
tests. Overall, I want a balanced approach.

\textbf{Producing metrics for assessment}

I need to finalize some metrics for the assessment. Let's go with an
overall score of 84; I noticed I initially thought 85, but it feels
right. For credible intervals, I'll set a lower bound of 74 and an upper
bound of 92.

For specific categories, I'll assess claims evidence at 76, methods at
74, advancing knowledge at 81, and logic communication at 88. Open
science will be 55, and global relevance will be 84. I'll list the
journal tier suggestions as well.

Finally, I must write an assessment summary of about 850-1000 words,
ensuring I'm specific with references like ``Fig. 1 shows a 35\%
increase in calls.''

\textbf{Finalizing assessment metrics}

I need to include some important figures in the assessment. For
instance, ``Fig. 3 shows two helplines over the long run, consistent and
dynamic.'' Then, ``Fig. 4 details the US Lifeline, showing weekly call
volumes with negative elasticity related to infections; however, the
stringency and income support effects aren't significant and not
robust.''

I'll also mention ``Fig. 5, which includes Germany and France's
topic-specific regressions, indicating a positive elasticity for
stringency and a negative effect for income support.''

Now, I'll produce the metrics object and ensure the JSON matches the
specified schema, confirming that the bounds are correctly set, and the
conditions hold true.

\textbf{Crafting the final analysis}

I want to focus on avoiding heavy formatting in this final analysis.
It's important that I don't mention the authors to keep the writing
neutral. I'll aim for approximately 1,000 words, targeting a range of
850 to 1,000 to maintain specificity without overwhelming detail. Now,
let's take the analysis and turn it into the final document. I'll work
on this to ensure it's clear and accessible for the reader.

\end{tcolorbox}

\newpage

\section{Class Warfare Political Exclusion of the Poor and the Roots of
Social Revolutionary Terrorism 1860
1950}\label{class-warfare-political-exclusion-of-the-poor-and-the-roots-of-social-revolutionary-terrorism-1860-1950}

\begin{tcolorbox}[enhanced jigsaw, left=2mm, opacitybacktitle=0.6, colframe=quarto-callout-note-color-frame, rightrule=.15mm, toprule=.15mm, bottomtitle=1mm, title=\textcolor{quarto-callout-note-color}{\faInfo}\hspace{0.5em}{Model assessment summary}, coltitle=black, colback=white, arc=.35mm, bottomrule=.15mm, toptitle=1mm, breakable, colbacktitle=quarto-callout-note-color!10!white, opacityback=0, titlerule=0mm, leftrule=.75mm]

This paper studies whether political exclusion of the poor predicts the
emergence and activity of social-revolutionary terrorist groups
(anarchist/leftist) across 99 countries from 1860--1950. The main
outcome is the annual count of active social-revolutionary groups from a
pre-1950 terrorism dataset; the key regressor is a V-Dem expert-coded
index of political power distribution by socioeconomic position
(sign-reversed so higher values mean greater exclusion). The empirical
core is a two-way fixed-effects OLS design with Driscoll--Kraay standard
errors, supplemented by an IV strategy that instruments local exclusion
with the average exclusion in the same world region (excluding the focal
country). The authors run numerous robustness checks: alternative
dependent variables (per-capita counts, inverse hyperbolic sine, binary
activity, group formation), alternative class-cleavage measures
(civil-liberties discrimination, unequal access to public
services/jobs/business, and a composite index), additional controls
(natural disasters, education, state involvement), nonlinearity tests,
and adjustments to address potential regional trending in the IV setup.
The main finding---a positive association between exclusion of the poor
and social-revolutionary terrorism---is persistent, and no comparable
relationship appears for nationalist, right-wing, or religious
terrorism.

Key strengths: (i) Timely and clearly framed question with plausible
mechanism grounded in rational-choice and grievance/relative deprivation
perspectives. (ii) Rare use of pre-1970 terrorism data, filling a gap
and enabling historical comparisons. (iii) Careful panel specification
with country and year fixed effects, attention to cross-sectional
dependence, and transparent reporting of estimates. (iv) Extensive
robustness work, including multiple operationalizations of both the
outcome and the focal regressor, and comparison across ideologies that
functions as a placebo test. (v) Replication materials and use of public
data sources make reproduction feasible.

Main limitations and identification risks: (1) Measurement of the
dependent variable. The outcome is the number of active groups rather
than attacks or casualties; this may capture organizational
proliferation more than intensity, and differences in how groups are
counted or named. The data are likely sparse with many zeros and small
counts, which raises distributional concerns for linear OLS. The paper
partially addresses this via binary and asinh transformations and a
group-formation onset measure, but does not estimate count models with
fixed effects (e.g., conditional Poisson), nor assess zero-inflation
explicitly. This choice could affect efficiency and possibly inference;
showing that FE Poisson results align would further bolster credibility.

\begin{enumerate}
\def\labelenumi{(\arabic{enumi})}
\setcounter{enumi}{1}
\item
  Measurement of the key regressor. The V-Dem class-power index relies
  on expert coding and item response models; measurement
  error---especially for the 19th century---may be nontrivial. The
  authors acknowledge potential attenuation; the IV approach mitigates
  this but hinges on strong assumptions. Relatedly, life expectancy is
  used as the main development proxy due to data constraints; while
  understandable, it may capture mortality crises and wars as well as
  development, and might mediate the effect of exclusion. Additional
  controls for industrialization/urbanization (only partially
  approximated later) would help reduce omitted variable bias.
\item
  Endogeneity and the IV strategy. Using regional average exclusion as
  an instrument raises exclusion-restriction concerns: cross-border
  diffusion of anarchist/leftist networks, region-wide shocks (empire
  formation/dissolution, war mobilization, decolonization, legal
  changes), and region-level democratization may directly affect local
  terrorism independent of local exclusion. The authors try to
  neutralize this by: removing potentially collinear controls, adding
  regional trends for democracy/autonomy and regional means of several
  covariates, and even including mean regional terrorism. First-stage F
  statistics are strong, and Anderson--Rubin confidence sets support
  non-weak inference. Nonetheless, with a single aggregated instrument,
  residual correlation between regional exclusion and local terrorism
  through unobserved regional shocks remains a live possibility. A
  leave-one-region-out analysis or alternative instruments (e.g.,
  historical franchise reforms, suffrage expansions, or settler
  institutions) would further strengthen causal claims.
\item
  Dynamics and timing. Most specifications use contemporaneous
  exclusion. If political inclusion reforms produce lagged changes in
  group activity, or if terrorism prompts reforms with a delay,
  estimates could be biased. While a lagged dependent variable is tested
  in robustness, lag structures for exclusion are not explored
  systematically (e.g., distributed lags or leads to test pre-trends).
  Event-study style checks around major enfranchisement or labor-law
  reforms would clarify dynamics and mitigate reverse causality
  concerns.
\item
  Controls and potential over/under-control. Democracy and domestic
  autonomy may be conceptually close to political exclusion and could
  lie on the causal pathway; including them risks absorbing part of the
  effect of interest or inducing post-treatment bias, whereas omitting
  them risks omitted variable bias. The paper shows robustness across
  specifications, but a directed acyclic graph (DAG) or explicit
  mediation analysis could clarify which specifications should be
  interpreted as total vs.~direct effects. Additional plausible
  confounders (union density, strike frequency, industrial employment
  share, urbanization rates) are largely unavailable for the full period
  but, where feasible, would be valuable sensitivity checks.
\item
  External validity and scope. Africa is excluded for data reasons,
  colonies and dominions are handled via a domestic autonomy index, and
  classification of groups may vary by archival coverage. These choices
  are well-motivated but limit generalizability and could introduce
  selection if data coverage correlates with both exclusion and
  terrorism. The authors partly address this by including autonomy and
  discrimination indices, but a discussion of coverage bias in the Old
  Terrorism Dataset and V-Dem for early periods would help readers
  calibrate inference.
\item
  Interpretation and magnitudes. The paper focuses on statistical
  significance. Substantive interpretation of effect sizes is limited;
  translating coefficients into probabilities of any group being active
  or into changes in expected group counts for realistic shifts in
  exclusion would help readers assess practical importance.
  Heterogeneity by baseline level of development or regime type is not
  explored beyond nonlinear checks; interactions might reveal where
  exclusion is most consequential.
\end{enumerate}

Communication and transparency are generally strong: the argument is
clear, data sources and construction steps are described, tables and
figures are interpretable, and numerous appendices document robustness.
Replication materials are available via a data repository, which aids
reproducibility. Overall, the paper provides credible, carefully
triangulated evidence that political exclusion of the poor is linked to
social-revolutionary terrorism in 1860--1950. The central identification
step---regional-trends IV---remains the most debatable element, and the
distributional choice for counts is a secondary methodological
limitation. Neither seems fatal; however, future work could solidify
causality with alternative instruments, dynamic designs, and count-model
estimators, and reinforce mechanism tests by examining reform episodes
and labor-movement covariates.

\end{tcolorbox}

\begin{tcolorbox}[enhanced jigsaw, left=2mm, opacitybacktitle=0.6, colframe=quarto-callout-tip-color-frame, rightrule=.15mm, toprule=.15mm, bottomtitle=1mm, title=\textcolor{quarto-callout-tip-color}{\faLightbulb}\hspace{0.5em}{Model reasoning trace}, coltitle=black, colback=white, arc=.35mm, bottomrule=.15mm, toptitle=1mm, breakable, colbacktitle=quarto-callout-tip-color!10!white, opacityback=0, titlerule=0mm, leftrule=.75mm]

\textbf{Creating assessment JSON}

I need to create a JSON that aligns with the schema for paper assessment
with tiers version 2. First, I'll draft an assessment summary of about
1000 words, focusing strictly on the content of the provided PDF.

Next, I'll include metrics like percentile scores and credible intervals
for several categories, which will involve boundaries from 0 to 100 and
maintaining inequalities. I'll also determine tier recommendations based
on the document's content, considering the empirical study's methodology
and its thoroughness, including various robustness checks across a
sample of 99 countries.

\textbf{Reviewing replication data and issues}

I need to provide details about code and data replication. They
mentioned that replication data and code can be found on Mendeley Data,
which supports open science. The dataset comes from Tschantret's 2019
work and V-Dem, and they've included replicable code.

I have to identify potential issues: the dependent variable is the
number of active groups, not attacks, due to data constraints. There are
also questions of omitted variables, reverse causality, and the validity
of the IV instrument relating to regional political exclusion. While
they controlled for regional trends, concerns about exogeneity persist.
The sample only includes 99 countries, omitting Africa and facing
uncertainties with V-Dem expert coding and development proxies.

\textbf{Analyzing dependent variable and methods}

They treat the dependent variable as a count using OLS, though counts
with fixed effects might benefit from Poisson FE due to zero-inflation.
They also explore transformations like inverse hyperbolic sine but don't
use Poisson directly. The range of active groups goes up to 14, and OLS
may be acceptable with robust errors.

However, they didn't address pre-trends or dynamic panel issues, and
there are endogeneity concerns. The OLS coefficient suggests about 0.07
groups per unit of exclusion, but this interpretation isn't properly
scaled. There are measurement issues related to ideology, and risks of
sample selection bias persist due to potential misclassification and
non-random missing data.

\textbf{Examining lag structure and confounding factors}

There might be a lag structure present, but they didn't test for lags of
exclusion, using only some in extensions. They mentioned including a
lagged dependent variable in robustness tests, which is helpful.
However, they didn't consider lags for explanatory variables or
``placebo ideologies.''

They didn't address potential confounding factors like unionization or
labor conflict, and there are no variables for strikes, likely due to
limited data. Their theoretical foundation includes a rational-economic
model and relative deprivation, which is good. While historical, they
briefly touch on current inequalities, suggesting moderate relevance for
policies today, but immediate guidance is limited. I'm glad they provide
open science access via Mendeley for good replicability.

\textbf{Assessing data reliability and methods}

Some data before 1900 relies on V-Dem version 10, which is replicable.
When it comes to logic and communication, the clarity is good, and the
figures, tables, and footnotes are well-structured. They're advancing
knowledge by using a unique dataset focused on that early period and
addressing ideology-specific effects, which is a valuable
contribution---even if it's not surprising, it holds up across various
checks.

Their methods involve OLS with fixed effects and Driscoll-Kraay errors,
which is typical. However, the IV instrument might violate exclusion due
to regional shocks driving terrorism directly. They incorporate year
fixed effects to account for global shocks, but region-specific
confounding potential still exists. While they add controls for regional
democracy trends, strong assumptions underpin their identification.
Another concern is the sparse country-year distribution of the dependent
variable, which may lead to many zeros. Although the fixed effects plus
Driscoll-Kraay might manage this, using a count panel with Poisson fixed
effects could be more efficient; still, they explored a binary dependent
variable for robustness.

\textbf{Evaluating dependent variable and control biases}

The dependent variable is the number of active groups, and a single
country-year can have multiple active groups merged. The theoretical
focus seems to be on activity intensity rather than just group counts,
which is good since they also test group formation as an onset.

However, there's potential for overcontrol bias because they include
life expectancy and democracy, which might act as mediators for
political exclusion. While their goal is to disentangle effects,
controlling for democracy could inadvertently overlap with the political
exclusion variable, even though they aren't the same.

The instrument they use is region-level exclusion, but if transnational
groups are influenced by that, it may compromise the exclusion
restriction. They do try to account for regional trends and local
terrorism, but it might not address everything entirely. They apply
Driscoll-Kraay standard errors for clustering, which is solid, though
missing data from Africa limits generalizability.

\textbf{Assessing functional form and coding issues}

The time fixed effects are applied well. They assume a linear effect,
but don't test for non-linearity regarding exclusion, which might
overlook some important dynamics. They did explore alternative measures,
which is a good step.

The dependent variable coding comes from Tschantret, but they don't
adjust for group size or actual attacks, creating a potential mismatch.
Their sample covers 1860-1950, including many colonies and addressing
domestic autonomy and discrimination, which is commendable. Ethical
considerations aren't applicable here.

I need to craft an assessment summary to highlight these strengths and
weaknesses while providing numeric values. For overall scoring, I'd
suggest around the 72nd percentile with a confidence interval of {[}60,
83{]}.

\textbf{Evaluating assessment scores}

The overall quality seems strong, but not top-tier.

For claims and evidence, they clearly assert that political exclusion
drives social-revolutionary terrorism, providing solid OLS and IV
evidence along with robust checks. I'd rate this around 75 with a
confidence interval of {[}62, 86{]}.

Their methods include good fixed effects OLS with Driscoll-Kraay errors,
although there are some concerns about IV exclusion and DV measurement.
I'd give this about 68, with a CI of {[}55, 80{]}.

The advancing knowledge aspect benefits from a unique historical dataset
bridging the pre-1970 era, so I'd score it at moderate-high 70 with
{[}58, 84{]}.

In terms of logic and communication, it's very well-written, earning an
82 with {[}72, 90{]}.

On the open science front, the data and code are available on Mendeley,
enhancing replicability, so I rate this at 85 with {[}75, 92{]}. I have
to mention that while the dataset for old terrorism is accessible,
careful attention is needed for replication code, which is good.

Lastly, concerning global relevance, it's historically oriented but ties
to current issues of inequality. I'd give it a 55 with {[}40, 70{]}.
Overall, this could potentially reach a top field journal tier.

\textbf{Assessing publication evaluation}

The paper appears in Defence and Peace Economics, which is a field
journal. However, I think we should evaluate it normatively instead of
focusing on publication standards. I'd rate its strengths around 3.4,
classifying it as a ``Top B-journal/Strong field,'' possibly reaching a
4, making my score approximately 3.6 with a confidence interval of
{[}3.0, 4.2{]}.

For predicted publication, considering its typical field relevance, I'd
say 3.2 with a CI of {[}2.6, 3.8{]}. Sure, I need to provide ratings
from 0-5, including decimals, and I'll keep it straightforward as the
instructions suggest.

I aim for a robust summary of about 800-1100 words, ensuring credible
intervals are at 90\%. It's essential to provide the structure in JSON
format, and I won't worry about identifying people in images since it
doesn't apply here.

\textbf{Crafting assessment summary}

Let's summarize the research question and dataset first. The focus is on
how political exclusion may foster social-revolutionary terrorism,
utilizing a unique dataset.

Next, there are several threats to consider, particularly around the
dependent variable (DV) measurement---comparing groups versus attacks,
low counts, and significant zeros. It's noted that they didn't use
Poisson FE or negative binomial models, although results remained robust
across transformations like binary and inverse hyperbolic sine (IHS).

For the independent variable measurement, they reverse the V-Dem
``political power by socio-economic position,'' relying on expert
ratings that might introduce measurement errors, especially for pre-1900
data.

Concerns about endogeneity loom as reverse causality and omitted
variables may play a role. Their instrumental variable (IV) approach,
based on regional political exclusion, raises worries about exclusion
restrictions, particularly with transnational ideological diffusion.

Additionally, they don't test lags on exclusion, which calls into
question the short-run vs long-run effects.

Lastly, while they account for domestic autonomy and regional
discrimination in colonies, measuring the DV becomes complex, especially
regarding terrorist groups and classification.

\textbf{Assessing dataset and methodology}

They used a dataset covering 99 countries, but excluded Africa, which
limits generalizability.

When it comes to alternative measures, they tested additional class
cleavage indicators, reducing the sample size, yet the results generally
hold, except for state business opportunities.

For outliers and influential values, they addressed these using inverse
hyperbolic sine (IHS) and binary measures, but didn't mention jackknife
techniques, which could improve robustness.

Their standard errors, Driscoll-Kraay, seem solid. However, potential
high multicollinearity between exclusion and democracy wasn't discussed,
even though it's likely present.

They did not convert their findings into meaningful effects, like
probabilities of at least one group, which could enhance interpretation.
On a positive note, they briefly show that other ideologies are
unresponsive, which is good.

Their control set includes life expectancy as a development indicator,
but this might act as a mediator or confounder since GDP data is
missing. Urbanization and industrialization considerations were notably
absent, aside from later references to state economic involvement and
access to education.

Reproducibility looks good with data and code links provided, and the
graphical presentation, with Figure 1 showing counts and Figures 2 and 3
included, is well done!

\textbf{Filling the JSON format for metrics}

I need to structure the metrics in JSON format correctly.

Overall score is 72, with a confidence interval ranging from 60 to 82.
For claims and evidence, the score is 75, with a range of 62 to 86. The
methods score 68, with lower and upper bounds of 55 and 80 respectively.
Advancing knowledge gets a score of 70, ranging from 58 to 82. Logic and
communication rates an impressive 84, with a confidence interval of 74
to 92. Open science scores 86, possibly within the range of 76 to 94.
Finally, the global relevance score is 56, with a range of 40 to 68.

As for the tiers, ``should'' proposes a score of 3.6 between 3.0 and
4.2, while ``will'' suggests a score of 3.2, ranging from 2.6 to 3.8. I
need to ensure I'm matching the schema names like ``tier\_should'' and
``tier\_will'' to keep consistency.

Alright, it's time to write up the assessment summary without including
any author names beyond general references. I'm aiming for about 900 to
1100 words, focusing on specific issues while avoiding heavy formatting
--- let's construct that!

\end{tcolorbox}

\newpage

\section{sciadv.adh4030}\label{sciadv.adh4030}

\begin{tcolorbox}[enhanced jigsaw, left=2mm, opacitybacktitle=0.6, colframe=quarto-callout-note-color-frame, rightrule=.15mm, toprule=.15mm, bottomtitle=1mm, title=\textcolor{quarto-callout-note-color}{\faInfo}\hspace{0.5em}{Model assessment summary}, coltitle=black, colback=white, arc=.35mm, bottomrule=.15mm, toptitle=1mm, breakable, colbacktitle=quarto-callout-note-color!10!white, opacityback=0, titlerule=0mm, leftrule=.75mm]

This paper combines quasi-experimental variation in the timing and
duration of COVID-19 school closures across German federal states and
school tracks with (i) nationally representative adolescent mental
health survey data collected near the end of the first lockdown (COPSY,
May--June 2020) and a pre-pandemic benchmark (BELLA, 2015--2017), and
(ii) high-frequency crisis helpline data (2019--2020). The main
identification uses a two-way fixed-effects design leveraging
within-state, across-grade/track and within--grade/track, across-state
differences in mandated weeks without in-person schooling, with age and
sex controls and extensive fixed effects. The authors conclude that each
additional week of school closure worsened health-related quality of
life (HRQoL) and increased psychosomatic, behavioral-emotional, and
depressive symptoms (no detectable effect on anxiety), that effects were
stronger for boys, younger adolescents, and those with limited living
space, and that family problems were the predominant issue revealed in
helpline calls; they further argue school closures largely explain the
overall deterioration of youth mental health observed between the
pre-pandemic and first-wave samples, albeit with wide confidence
intervals.

Key identification threats and data limitations are as follows. First,
the main estimates come from a single cross-section during the first
wave (COPSY) with variation in exposure generated by reopening
schedules. While two-way fixed effects absorb state-level and
grade/track-level differences, the design relies on the assumption that
no other within-state, grade/track-specific shocks correlated with
reopening sequences affected mental health. The logic is plausible since
other policies were not grade-targeted; nevertheless, unobserved,
grade-specific changes in school support, remote learning intensity, or
local implementation that correlate with reopening priority could bias
estimates. The paper mitigates this by adding state-level pandemic
severity and policy stringency controls in a correlated random effects
(Mundlak) framework; results are stable, but those controls cannot rule
out grade-specific factors by state. Second, the pre-pandemic comparison
relies on BELLA measured 2015--2017 in a different mode
(computer-assisted, in-person) versus COPSY online in 2020. Secular
trends and mode effects could confound the pre/post difference. The
authors include time trend and seasonality checks and find little
sensitivity, but the several-year gap and mode shift remain an important
caveat when concluding that closures ``largely explain'' the
deterioration. Relatedly, only two of the mental health scales are
available pre-pandemic; the other three are standardized within COPSY,
which complicates harmonized interpretation across measures.

Third, exposure measurement is imperfect. ``Weeks of closure'' is
defined up to the (unknown) individual COPSY interview date and is
top-coded at the COPSY start date in baseline; an alternative uses the
end date; and another adjusts for holidays; results are robust. However,
the end of closure is defined as ``partial reopening,'' which may not
capture the intensity of in-person schooling (e.g., hybrid schedules
varying across schools/classes), implying classical measurement error
likely attenuating estimates and leaving possible heterogeneity
unobserved. Fourth, inference and clustering: standard errors are
clustered at the treatment level (state×grade×track), which is
appropriate conceptually, but the effective number of clusters may be
moderate; no wild-cluster bootstrap is reported. Fifth, the COPSY
analytic sample is n˜907 adolescents; sampling weights support
representativeness on observables, but nonresponse/selection into online
participation could bias results if correlated with both mental health
and closure exposure within fixed-effect cells.

Sixth, the helpline analyses are suggestive rather than causal. The
linking of callers to exposure uses age as a proxy for grade and the
receiving center's location as the state identifier, which may not
precisely reflect the caller's residence or school context. Topic coding
is counselor-reported and calls considered are those that progressed to
counseling, creating selection that may vary over time. The analysis
shows compelling descriptive patterns (family-problem calls spiking when
schools are closed and diverging between early vs.~late returners),
reinforced by an ARMA model with fortnight dummies; however, other
contemporaneous changes (e.g., local news cycles, staggered reopenings
of out-of-school activities) could contribute, and the median split on
closure weeks is a coarse exposure metric.

Seventh, heterogeneity analyses rely on interactions with discrete
splits (e.g., living space per school-aged child above/below the
median), which are interpretable but may mask continuous dose-response
patterns; measurement of living space and the allocation of space among
children are approximate. Eighth, the absence of an anxiety effect may
reflect the timing (short-run, first wave) or measurement focus
(generalized anxiety), and should be interpreted cautiously. Ninth, the
large per-week effect magnitudes (e.g., \textasciitilde0.11 SD per week
on HRQoL) imply sizable cumulative impacts over the observed exposure
range; while consistent with the pre/post differences, they depend on
linearity over the 4.7--10.1 week window and may not extrapolate to
later, longer closures---as the authors note. Tenth, external validity
is limited: Germany's first-wave context combined low adolescent case
rates, specific reopening priorities by grade/track, and particular
broader restrictions; the authors appropriately caution about
cross-country generalization.

Internal consistency and robustness are otherwise strong: estimates are
stable across multiple specifications (e.g., alternative imputation of
exposure, quadratic terms, parental-reported outcomes, absorbing
state×track fixed effects), and adding state-level pandemic severity and
detailed policy stringency measures does not change the closure effects;
only one of 65 policy coefficients reaches 5\% significance (on
anxiety), limiting concerns about omitted, state-level policy confounds.
Pre-pandemic mental health measures are uncorrelated with mandated
closure duration, supporting quasi-exogeneity of exposure. Communication
is clear and transparent; figures align with text; the logic linking
design to assumptions is explicit; and the heterogeneity and helpline
sections triangulate mechanisms (family stress). Replication barriers
exist: the core microdata (COPSY/BELLA) and helpline data are
restricted-access; the ordinance-derived school-closure dataset and
aggregate policy data are available, but code availability is not
explicitly documented. Overall, this is a careful, policy-relevant study
with a credible identification strategy given available data,
appropriately framed limitations, and strong but not definitive evidence
that school closures directly worsened adolescent mental health in the
short run and likely explain a substantial share of the observed
deterioration during the first pandemic wave.

\end{tcolorbox}

\begin{tcolorbox}[enhanced jigsaw, left=2mm, opacitybacktitle=0.6, colframe=quarto-callout-tip-color-frame, rightrule=.15mm, toprule=.15mm, bottomtitle=1mm, title=\textcolor{quarto-callout-tip-color}{\faLightbulb}\hspace{0.5em}{Model reasoning trace}, coltitle=black, colback=white, arc=.35mm, bottomrule=.15mm, toptitle=1mm, breakable, colbacktitle=quarto-callout-tip-color!10!white, opacityback=0, titlerule=0mm, leftrule=.75mm]

\textbf{Considering mental health trends}

I need to mention selection into tracks while noting the missing
within-track grade-specific pre-trends. It's vital to include
pre-pandemic mental health data from 2015-2017, earlier than 2020. I
should also consider potential secular trends and check time trends with
quarter dummies. Measurement concerns arise from self-reported mental
health, and there might be reporting bias due to different
standardizations. Plus, using sampling weights and being aware of finite
cluster concerns is important for accuracy. I'll look into the survey
start date assumptions and how they handle top-coding and variations.

\textbf{Addressing potential confounding factors}

I see that they control for age and sex, incorporating school
track-specific grade level fixed effects and state fixed effects. They
identify variables across states and grade-tracks, which seems
promising. However, potential confounding arises with additional
state-level policy measures that aren't age- or grade-specific,
especially if families with older kids differ. Despite this, they
include fixed effects for state and grade-by-track while also
considering state-level stringency measures. They interpret effect sizes
per week as -0.107 SD for health-related quality of life, but that seems
significant. Am I interpreting that correctly?

\textbf{Examining effects of school closures}

It seems that over 4 additional weeks would lead to a decline of 0.428
SD, and the total decline they found compared to pre is 0.824 SD. They
suggest that closures largely explain this deterioration but also
caution that there's a wide confidence interval. If mandated closures
average about 4.7 weeks, their per-week effect sizes might be
overstated. The measured overall effect net of closures becomes
insignificant, implying the closures alone may not fully account for the
differences. I'm curious about how they interact closure measures with
COPSY!

\textbf{Analyzing closure effects in the study}

In Equation 3, they include variables like dscx, ct, and the interaction
ct*dscx. Beta3 measures the effect of closures during the pandemic,
while Beta2 captures the COVID-19 effect net of closures. Interestingly,
Beta2 is reported as insignificantly different from zero, suggesting
closure explains only 31\% of the decline in health-related quality of
life. There's a moderate sample size of n=907, and variations in closure
weeks are limited, which could lead to imprecision. Measurement modes
also differ between pre-pandemic and pandemic surveys, contributing to
potential biases.

\textbf{Evaluating anxiety measurement and helpline data}

Anxiety, measured as SCARED, shows no notable effects, which might be
due to measurement limitations tied to generalized anxiety. The crisis
helpline data relies on call volumes, topics, and locations of receiving
centers, but it's not fully random as it captures only calls that
proceeded to counseling. Although confidentiality is maintained, the
observational nature limits causal conclusions. They present moving
averages and ARMA regression with dummies while identifying variations
over short windows. There's a suggestion that family problem calls
decreased after reopening, but linking these effects to closure
exposures is challenging due to potential confounding factors, including
top-coding closure weeks.

\textbf{Exploring grade-specific confounders}

I'm considering the underlying assumption that there are no
grade-specific confounders at the state level. It's possible that
distance learning quality differed across grades or tracks, which ties
back to the closure effects. There's also a concern that the intensity
of remote learning might vary independently from closure weeks. Although
closure weeks align with grade levels, grade-track fixed effects are
included to account for this. However, if certain grades, particularly
higher ones, reopened earlier, it could create a different slope in
mental health changes. This might introduce bias and calls into question
whether we can assume stable grade-specific effects across states. What
if, for instance, ninth graders in some areas had greater access to
remote learning resources?

\textbf{Assessing grade-specific measures and data access}

There's a challenge in considering ``other measures'' that aren't
grade-specific, as resource allocation may vary based on policy and the
reopening of grades. Since reopening timelines might correlate with
additional grade-specific supports, it's tough to exclude those factors.
They work with a small number of states (16), but clustering may yield a
significant number of clusters, potentially over 100. Interpreting the
results as ``school closures largely explain'' the findings seems a bit
strong, and they do provide cautions. Additionally, the dataset isn't
publicly shareable, and access requires contact. Code availability is
unclear, making open science somewhat limited. The study's global
relevance is high for policy discussions, especially regarding closure
tradeoffs, but its generalizability is uncertain, focused only on
Germany's first wave.

\textbf{Analyzing methods and measurement boundaries}

The use of two-way fixed effects seems reasonable, and I like that they
incorporate Chamberlain-Mundlak correlated random effects to include
factors like state-level policy stringency and pandemic severity. They
have a variety of robustness checks, including imputation methods and
using parental reports as outcomes, which is good! However, addressing
pre-trends through within-sample event studies isn't possible with this
cross-sectional dataset. Additionally, there are concerns about
measurement boundaries. Standardizing scales like HBSC-SCL, SDQ, and
SCARED within COPSY without a pre-pandemic baseline limits effect sizes.
For KIDSCREEN-10 and CES-DC, standardization to BELLA makes sense for
comparisons across time, but I wonder if they also standardized for
COPSY-only regressions.

\textbf{Evaluating standardization and measurement concerns}

KIDSCREEN-10 and CES-DC are standardized to a mean of 0 and SD of 1 in
BELLA, and they use these scales in COPSY. If the standardization
remains consistent, the effect sizes would reflect pre-pandemic levels.
However, for other scales used only in COPSY, they standardize within
that context, which feels a bit inconsistent, but they clarify this for
comparison purposes. Sampling weights and clustering SEs at the
treatment level are good practices. There's potential measurement error
regarding closure exposure, as many grade-level reopenings were partial.
This could lead to inaccuracies, but it might also introduce
variability. They assume that other measures don't vary by grade level,
though I'm curious if there were any grade-specific contact
restrictions.

\textbf{Assessing helpline linkage and study evaluation}

The helpline links calls based on the location of the receiving center,
which might not match the caller's actual location. They recognize this
is a regional center receiving calls from various areas, which is
important. They use age as a proxy for grade since the exact grade isn't
known. We should assign quantitative metrics related to fields like
public health, social science, or economics. Overall, it's a strong
paper with credible quasi-experimental design addressing a relevant
policy question. Limitations include its cross-sectional nature, which
makes it hard to fully separate the effects of the pandemic from
closures. The helpline analysis provides suggestive evidence but isn't
completely causal. I'd rate it around the 80th to 88th percentile
compared to top field work.

\textbf{Evaluating overall scores and claims}

I'm setting the overall midpoint at 84, with a confidence interval of
75-90. The major claim that ``school closures largely explain
deterioration'' might be slightly overstated, but it's presented
cautiously, so I'm giving it an 82 (74-88). For methods, the two-way
fixed effects and correlated random effects checks are solid but not
gold standard---I'd rate that an 80 (70-87).

Advancing knowledge has strong policy relevance and adds to the
literature, so maybe an 85 (78-92). However, global generalization
remains uncertain, so I'm setting that at 83 (74-90). The writing and
figures are clear and transparent, so that gets an 88 (80-94).

Open science is limited since the data isn't open, only a compiled
school policy dataset is available. I'm unsure about the code, so I'll
give that a low score of 55 (40-65). For global relevance, it's high
because they provide quant estimates---I'd say 80 (70-90). Considering
the quality, I'd rate the journal tier at 4.3, with a confidence
interval of 3.8-4.7, since it's published in a good-tier journal like
Sci Adv.

\textbf{Clarifying journal tier metrics}

I'm considering the journal tier, which will likely fall between a top
field or A-journal, so I'm thinking of setting that at 4.2 with a
confidence interval of 3.5-4.8. Now, the schema expects numeric entries
for `tier\_should' and `tier\_will' as objects with corresponding scores
and confidence intervals, which are nested under the metrics as
`tier\_should' and `tier\_will'.

I see that these metrics should include score, ci\_lower, and ci\_upper.
Next, I need to ensure that I provide 90\% credible intervals within
0-100 for percentiles, making sure that the lower value is less than the
mid, which is less than the upper.

Additionally, the assessment summary should aim for about 1000 words. I
plan to produce around 800-1100 words and I'll be specific by mentioning
internal coherence checks, sample sizes, and cluster counts. While the
exact cluster count isn't known, I can discuss the possibility of a
limited number of treated cells, as they cluster by state, grade, and
track, suggesting an adequate count but with some caution.

\textbf{Highlighting potential issues and structuring metrics}

I need to mention that the claim of no anxiety effect might stem from
measurement timings early in the pandemic. While we can't rely solely on
external factors, it's a plausible limitation. Another aspect to address
is potential selection bias due to non-response in COPSY. Although
weights are used to ensure representativeness, unobservable selection
might still be an issue.

Additionally, family climate is measured by parents, but there's no
direct mental health effect found on them, which raises the possibility
of measurement errors due to adaptation or underreporting. The paper
claims all necessary data for evaluations are available, but
replicability isn't straightforward since code isn't provided.

To comply with the metrics, I need to produce sub-objects for each
metric: overall, claims\_evidence, methods, advancing\_knowledge,
logic\_communication, open\_science, and global\_relevance. Each one
will need midpoint, lower\_bound, and upper\_bound values, and I should
ensure no extra text beyond the JSON. Valid numeric fields will be
required, using decimals where needed.

\textbf{Creating the assessment summary}

I need to summarize the design as a two-way fixed effects approach,
utilizing exogenous variations from federal states' reopening sequences
by grade and track. The primary data comes from COPSY's wave from May to
June 2020 and pre-pandemic BELLA measurements from 2015-2017, using
various mental health measures while applying weights for robustness.

The findings show that an additional week of closure reduces
health-related quality of life (HRQoL) while increasing psychosomatic
issues, though no anxiety effects were found. There are some threats to
validity, mostly from the cross-sectional design and potential
measurement errors.

I must also note concerns around external validity, the
representativeness of helpline calls, and the lack of longer-term
follow-ups or robust coding for spatial data. Overall, there's internal
consistency across measures, though transparency may be hindered by
access constraints to their data and code.

\textbf{Assigning scores for evaluation}

I'm setting the scores now, starting with ``Open Science.'' Given they
don't provide code and the data is restricted, I think a score of 50
with a range from 35 to 60 makes sense. I previously considered 55, but
50 seems more fitting right now.

Overall, I'm aiming for a midpoint of 84 with lower and upper bounds of
75 and 90. Here's what I'll assign:

\begin{itemize}
\tightlist
\item
  Claims Evidence: 82 (72-88)
\item
  Methods: 80 (70-87)
\item
  Advancing Knowledge: 83 (73-90)
\item
  Logic Communication: 88 (80-94)
\item
  Open Science: 50 (35-60)
\item
  Global Relevance: 80 (68-90)
\end{itemize}

Now I'll produce the `tier\_should' score of 4.3 (CI 3.8-4.7) and
`tier\_will' score of 4.2 (3.5-4.8). I need to ensure this is all valid
JSON, respecting the constraints on percentiles. Also, I should produce
a detailed summary of around 1000 words, keeping in mind the desired
verbosity.

\textbf{Preparing the summary}

I want to craft a summary that falls around 900 to 1050 words. My goal
is to keep it neutral, specific, and concrete, focusing solely on the
content of the PDF. While I can mention the paper's title, I need to
avoid bringing up ``Science Advances'' or the status of the paper as
evidence. Also, to maintain neutrality, I should refrain from mentioning
authors' names. Now, I'll proceed to produce the final JSON.

\end{tcolorbox}

\newpage



\end{document}
